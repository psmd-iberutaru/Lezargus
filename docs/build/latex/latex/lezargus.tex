%% Generated by Sphinx.
\def\sphinxdocclass{report}
\IfFileExists{luatex85.sty}
 {\RequirePackage{luatex85}}
 {\ifdefined\luatexversion\ifnum\luatexversion>84\relax
  \PackageError{sphinx}
  {** With this LuaTeX (\the\luatexversion),Sphinx requires luatex85.sty **}
  {** Add the LaTeX package luatex85 to your TeX installation, and try again **}
  \endinput\fi\fi}
\documentclass[letterpaper,11pt,english]{sphinxmanual}
\ifdefined\pdfpxdimen
   \let\sphinxpxdimen\pdfpxdimen\else\newdimen\sphinxpxdimen
\fi \sphinxpxdimen=.75bp\relax
\ifdefined\pdfimageresolution
    \pdfimageresolution= \numexpr \dimexpr1in\relax/\sphinxpxdimen\relax
\fi
%% let collapsible pdf bookmarks panel have high depth per default
\PassOptionsToPackage{bookmarksdepth=5}{hyperref}
%% turn off hyperref patch of \index as sphinx.xdy xindy module takes care of
%% suitable \hyperpage mark-up, working around hyperref-xindy incompatibility
\PassOptionsToPackage{hyperindex=false}{hyperref}
%% memoir class requires extra handling
\makeatletter\@ifclassloaded{memoir}
{\ifdefined\memhyperindexfalse\memhyperindexfalse\fi}{}\makeatother

\PassOptionsToPackage{booktabs}{sphinx}
\PassOptionsToPackage{colorrows}{sphinx}

\PassOptionsToPackage{warn}{textcomp}

\catcode`^^^^00a0\active\protected\def^^^^00a0{\leavevmode\nobreak\ }
\usepackage{cmap}
\usepackage{fontspec}
\defaultfontfeatures[\rmfamily,\sffamily,\ttfamily]{}
\usepackage{amsmath,amssymb,amstext}
\usepackage{polyglossia}
\setmainlanguage{english}



\setmainfont{FreeSerif}[
  Extension      = .otf,
  UprightFont    = *,
  ItalicFont     = *Italic,
  BoldFont       = *Bold,
  BoldItalicFont = *BoldItalic
]
\setsansfont{FreeSans}[
  Extension      = .otf,
  UprightFont    = *,
  ItalicFont     = *Oblique,
  BoldFont       = *Bold,
  BoldItalicFont = *BoldOblique,
]
\setmonofont{FreeMono}[
  Extension      = .otf,
  UprightFont    = *,
  ItalicFont     = *Oblique,
  BoldFont       = *Bold,
  BoldItalicFont = *BoldOblique,
]



\usepackage[Bjarne]{fncychap}
\usepackage[,numfigreset=1,mathnumfig]{sphinx}

\fvset{fontsize=\small}
\usepackage{geometry}


% Include hyperref last.
\usepackage{hyperref}
% Fix anchor placement for figures with captions.
\usepackage{hypcap}% it must be loaded after hyperref.
% Set up styles of URL: it should be placed after hyperref.
\urlstyle{same}

\addto\captionsenglish{\renewcommand{\contentsname}{Contents:}}

\usepackage{sphinxmessages}
\setcounter{tocdepth}{1}

\usepackage{enumitem}

\title{Lezargus}
\date{Sep 11, 2023}
\release{}
\author{Sparrow}
\newcommand{\sphinxlogo}{\vbox{}}
\renewcommand{\releasename}{}
\usepackage[columns=1]{idxlayout}\makeindex
\begin{document}

\pagestyle{empty}
\sphinxmaketitle
\pagestyle{plain}
\sphinxtableofcontents
\pagestyle{normal}
\phantomsection\label{\detokenize{index::doc}}



\chapter{Code Manual}
\label{\detokenize{index:code-manual}}
\sphinxAtStartPar
The code manual is primarily for software maintainers and other IRTF staff. It
details the software API documentation of Lezargus and its inner workings.
This does not detail any of the design principles of the software but instead
it is the function and class documentation of the software (generated by Sphinx).

\sphinxAtStartPar
\sphinxhref{https://psmd-iberutaru.github.io/Lezargus/build/html/coverage/index.html}{Code coverage}%
\begin{footnote}[1]\sphinxAtStartFootnote
\sphinxnolinkurl{https://psmd-iberutaru.github.io/Lezargus/build/html/coverage/index.html}
%
\end{footnote} is available in html. The code manual (and overall
documentation) is generated via Sphinx and the code coverage is generated by
a different tool. As such, it is manually linked rather than integrated.

\sphinxstepscope


\section{lezargus}
\label{\detokenize{code/modules:lezargus}}\label{\detokenize{code/modules::doc}}
\sphinxstepscope


\subsection{lezargus package}
\label{\detokenize{code/lezargus:lezargus-package}}\label{\detokenize{code/lezargus::doc}}

\subsubsection{Subpackages}
\label{\detokenize{code/lezargus:subpackages}}
\sphinxstepscope


\paragraph{lezargus.container package}
\label{\detokenize{code/lezargus.container:lezargus-container-package}}\label{\detokenize{code/lezargus.container::doc}}

\subparagraph{Submodules}
\label{\detokenize{code/lezargus.container:submodules}}
\sphinxstepscope


\subparagraph{lezargus.container.cube module}
\label{\detokenize{code/lezargus.container.cube:module-lezargus.container.cube}}\label{\detokenize{code/lezargus.container.cube:lezargus-container-cube-module}}\label{\detokenize{code/lezargus.container.cube::doc}}\index{module@\spxentry{module}!lezargus.container.cube@\spxentry{lezargus.container.cube}}\index{lezargus.container.cube@\spxentry{lezargus.container.cube}!module@\spxentry{module}}
\sphinxAtStartPar
Spectral data cube container.

\sphinxAtStartPar
This module and class primarily deals with spectral data cubes containing both
spatial and spectral information.
\index{LezargusCube (class in lezargus.container.cube)@\spxentry{LezargusCube}\spxextra{class in lezargus.container.cube}}

\begin{savenotes}\begin{fulllineitems}
\phantomsection\label{\detokenize{code/lezargus.container.cube:lezargus.container.cube.LezargusCube}}
\pysigstartsignatures
\pysiglinewithargsret{\sphinxbfcode{\sphinxupquote{class\DUrole{w,w}{  }}}\sphinxcode{\sphinxupquote{lezargus.container.cube.}}\sphinxbfcode{\sphinxupquote{LezargusCube}}}{\sphinxparam{\DUrole{n,n}{wavelength}\DUrole{p,p}{:}\DUrole{w,w}{  }\DUrole{n,n}{\sphinxhref{https://numpy.org/doc/stable/reference/generated/numpy.ndarray.html\#numpy.ndarray}{ndarray}%
\begin{footnote}[2]\sphinxAtStartFootnote
\sphinxnolinkurl{https://numpy.org/doc/stable/reference/generated/numpy.ndarray.html\#numpy.ndarray}
%
\end{footnote}}}, \sphinxparam{\DUrole{n,n}{data}\DUrole{p,p}{:}\DUrole{w,w}{  }\DUrole{n,n}{\sphinxhref{https://numpy.org/doc/stable/reference/generated/numpy.ndarray.html\#numpy.ndarray}{ndarray}%
\begin{footnote}[3]\sphinxAtStartFootnote
\sphinxnolinkurl{https://numpy.org/doc/stable/reference/generated/numpy.ndarray.html\#numpy.ndarray}
%
\end{footnote}}}, \sphinxparam{\DUrole{n,n}{uncertainty}\DUrole{p,p}{:}\DUrole{w,w}{  }\DUrole{n,n}{\sphinxhref{https://numpy.org/doc/stable/reference/generated/numpy.ndarray.html\#numpy.ndarray}{ndarray}%
\begin{footnote}[4]\sphinxAtStartFootnote
\sphinxnolinkurl{https://numpy.org/doc/stable/reference/generated/numpy.ndarray.html\#numpy.ndarray}
%
\end{footnote}}\DUrole{w,w}{  }\DUrole{o,o}{=}\DUrole{w,w}{  }\DUrole{default_value}{None}}, \sphinxparam{\DUrole{n,n}{wavelength\_unit}\DUrole{p,p}{:}\DUrole{w,w}{  }\DUrole{n,n}{str\DUrole{w,w}{  }\DUrole{p,p}{|}\DUrole{w,w}{  }\sphinxhref{https://docs.astropy.org/en/stable/api/astropy.units.Unit.html\#astropy.units.Unit}{Unit}%
\begin{footnote}[5]\sphinxAtStartFootnote
\sphinxnolinkurl{https://docs.astropy.org/en/stable/api/astropy.units.Unit.html\#astropy.units.Unit}
%
\end{footnote}\DUrole{w,w}{  }\DUrole{p,p}{|}\DUrole{w,w}{  }None}\DUrole{w,w}{  }\DUrole{o,o}{=}\DUrole{w,w}{  }\DUrole{default_value}{None}}, \sphinxparam{\DUrole{n,n}{data\_unit}\DUrole{p,p}{:}\DUrole{w,w}{  }\DUrole{n,n}{str\DUrole{w,w}{  }\DUrole{p,p}{|}\DUrole{w,w}{  }\sphinxhref{https://docs.astropy.org/en/stable/api/astropy.units.Unit.html\#astropy.units.Unit}{Unit}%
\begin{footnote}[6]\sphinxAtStartFootnote
\sphinxnolinkurl{https://docs.astropy.org/en/stable/api/astropy.units.Unit.html\#astropy.units.Unit}
%
\end{footnote}\DUrole{w,w}{  }\DUrole{p,p}{|}\DUrole{w,w}{  }None}\DUrole{w,w}{  }\DUrole{o,o}{=}\DUrole{w,w}{  }\DUrole{default_value}{None}}, \sphinxparam{\DUrole{n,n}{mask}\DUrole{p,p}{:}\DUrole{w,w}{  }\DUrole{n,n}{\sphinxhref{https://numpy.org/doc/stable/reference/generated/numpy.ndarray.html\#numpy.ndarray}{ndarray}%
\begin{footnote}[7]\sphinxAtStartFootnote
\sphinxnolinkurl{https://numpy.org/doc/stable/reference/generated/numpy.ndarray.html\#numpy.ndarray}
%
\end{footnote}\DUrole{w,w}{  }\DUrole{p,p}{|}\DUrole{w,w}{  }None}\DUrole{w,w}{  }\DUrole{o,o}{=}\DUrole{w,w}{  }\DUrole{default_value}{None}}, \sphinxparam{\DUrole{n,n}{flags}\DUrole{p,p}{:}\DUrole{w,w}{  }\DUrole{n,n}{\sphinxhref{https://numpy.org/doc/stable/reference/generated/numpy.ndarray.html\#numpy.ndarray}{ndarray}%
\begin{footnote}[8]\sphinxAtStartFootnote
\sphinxnolinkurl{https://numpy.org/doc/stable/reference/generated/numpy.ndarray.html\#numpy.ndarray}
%
\end{footnote}\DUrole{w,w}{  }\DUrole{p,p}{|}\DUrole{w,w}{  }None}\DUrole{w,w}{  }\DUrole{o,o}{=}\DUrole{w,w}{  }\DUrole{default_value}{None}}, \sphinxparam{\DUrole{n,n}{header}\DUrole{p,p}{:}\DUrole{w,w}{  }\DUrole{n,n}{\sphinxhref{https://docs.astropy.org/en/stable/io/fits/api/headers.html\#astropy.io.fits.Header}{Header}%
\begin{footnote}[9]\sphinxAtStartFootnote
\sphinxnolinkurl{https://docs.astropy.org/en/stable/io/fits/api/headers.html\#astropy.io.fits.Header}
%
\end{footnote}\DUrole{w,w}{  }\DUrole{p,p}{|}\DUrole{w,w}{  }None}\DUrole{w,w}{  }\DUrole{o,o}{=}\DUrole{w,w}{  }\DUrole{default_value}{None}}}{}
\pysigstopsignatures
\sphinxAtStartPar
Bases: {\hyperref[\detokenize{code/lezargus.container.parent:lezargus.container.parent.LezargusContainerArithmetic}]{\sphinxcrossref{\sphinxcode{\sphinxupquote{LezargusContainerArithmetic}}}}}

\sphinxAtStartPar
Container to hold spectral cube data and perform operations on it.
\index{wavelength (lezargus.container.cube.LezargusCube attribute)@\spxentry{wavelength}\spxextra{lezargus.container.cube.LezargusCube attribute}}

\begin{savenotes}\begin{fulllineitems}
\phantomsection\label{\detokenize{code/lezargus.container.cube:lezargus.container.cube.LezargusCube.wavelength}}
\pysigstartsignatures
\pysigline{\sphinxbfcode{\sphinxupquote{wavelength}}}
\pysigstopsignatures
\sphinxAtStartPar
The wavelength of the spectra. The unit of wavelength is typically
in microns; but, check the \sphinxtitleref{wavelength\_unit} value.
\begin{quote}\begin{description}
\sphinxlineitem{Type}
\sphinxAtStartPar
ndarray

\end{description}\end{quote}

\end{fulllineitems}\end{savenotes}

\index{data (lezargus.container.cube.LezargusCube attribute)@\spxentry{data}\spxextra{lezargus.container.cube.LezargusCube attribute}}

\begin{savenotes}\begin{fulllineitems}
\phantomsection\label{\detokenize{code/lezargus.container.cube:lezargus.container.cube.LezargusCube.data}}
\pysigstartsignatures
\pysigline{\sphinxbfcode{\sphinxupquote{data}}}
\pysigstopsignatures
\sphinxAtStartPar
The flux of the spectra cube. The unit of the flux is typically
in flam; but, check the \sphinxtitleref{flux\_unit} value.
\begin{quote}\begin{description}
\sphinxlineitem{Type}
\sphinxAtStartPar
ndarray

\end{description}\end{quote}

\end{fulllineitems}\end{savenotes}

\index{uncertainty (lezargus.container.cube.LezargusCube attribute)@\spxentry{uncertainty}\spxextra{lezargus.container.cube.LezargusCube attribute}}

\begin{savenotes}\begin{fulllineitems}
\phantomsection\label{\detokenize{code/lezargus.container.cube:lezargus.container.cube.LezargusCube.uncertainty}}
\pysigstartsignatures
\pysigline{\sphinxbfcode{\sphinxupquote{uncertainty}}}
\pysigstopsignatures
\sphinxAtStartPar
The uncertainty in the flux of the spectra. The unit of the uncertainty
is the same as the flux value; per \sphinxtitleref{uncertainty\_unit}.
\begin{quote}\begin{description}
\sphinxlineitem{Type}
\sphinxAtStartPar
ndarray

\end{description}\end{quote}

\end{fulllineitems}\end{savenotes}

\index{wavelength\_unit (lezargus.container.cube.LezargusCube attribute)@\spxentry{wavelength\_unit}\spxextra{lezargus.container.cube.LezargusCube attribute}}

\begin{savenotes}\begin{fulllineitems}
\phantomsection\label{\detokenize{code/lezargus.container.cube:lezargus.container.cube.LezargusCube.wavelength_unit}}
\pysigstartsignatures
\pysigline{\sphinxbfcode{\sphinxupquote{wavelength\_unit}}}
\pysigstopsignatures
\sphinxAtStartPar
The unit of the wavelength array.
\begin{quote}\begin{description}
\sphinxlineitem{Type}
\sphinxAtStartPar
Astropy Unit

\end{description}\end{quote}

\end{fulllineitems}\end{savenotes}

\index{flux\_unit (lezargus.container.cube.LezargusCube attribute)@\spxentry{flux\_unit}\spxextra{lezargus.container.cube.LezargusCube attribute}}

\begin{savenotes}\begin{fulllineitems}
\phantomsection\label{\detokenize{code/lezargus.container.cube:lezargus.container.cube.LezargusCube.flux_unit}}
\pysigstartsignatures
\pysigline{\sphinxbfcode{\sphinxupquote{flux\_unit}}}
\pysigstopsignatures
\sphinxAtStartPar
The unit of the flux array.
\begin{quote}\begin{description}
\sphinxlineitem{Type}
\sphinxAtStartPar
Astropy Unit

\end{description}\end{quote}

\end{fulllineitems}\end{savenotes}

\index{uncertainty\_unit (lezargus.container.cube.LezargusCube attribute)@\spxentry{uncertainty\_unit}\spxextra{lezargus.container.cube.LezargusCube attribute}}

\begin{savenotes}\begin{fulllineitems}
\phantomsection\label{\detokenize{code/lezargus.container.cube:lezargus.container.cube.LezargusCube.uncertainty_unit}}
\pysigstartsignatures
\pysigline{\sphinxbfcode{\sphinxupquote{uncertainty\_unit}}}
\pysigstopsignatures
\sphinxAtStartPar
The unit of the uncertainty array. This unit is the same as the flux
unit.
\begin{quote}\begin{description}
\sphinxlineitem{Type}
\sphinxAtStartPar
Astropy Unit

\end{description}\end{quote}

\end{fulllineitems}\end{savenotes}

\index{mask (lezargus.container.cube.LezargusCube attribute)@\spxentry{mask}\spxextra{lezargus.container.cube.LezargusCube attribute}}

\begin{savenotes}\begin{fulllineitems}
\phantomsection\label{\detokenize{code/lezargus.container.cube:lezargus.container.cube.LezargusCube.mask}}
\pysigstartsignatures
\pysigline{\sphinxbfcode{\sphinxupquote{mask}}}
\pysigstopsignatures
\sphinxAtStartPar
A mask of the flux data, used to remove problematic areas. Where True,
the values of the flux is considered mask.
\begin{quote}\begin{description}
\sphinxlineitem{Type}
\sphinxAtStartPar
ndarray

\end{description}\end{quote}

\end{fulllineitems}\end{savenotes}

\index{flags (lezargus.container.cube.LezargusCube attribute)@\spxentry{flags}\spxextra{lezargus.container.cube.LezargusCube attribute}}

\begin{savenotes}\begin{fulllineitems}
\phantomsection\label{\detokenize{code/lezargus.container.cube:lezargus.container.cube.LezargusCube.flags}}
\pysigstartsignatures
\pysigline{\sphinxbfcode{\sphinxupquote{flags}}}
\pysigstopsignatures
\sphinxAtStartPar
Flags of the flux data. These flags store metadata about the flux.
\begin{quote}\begin{description}
\sphinxlineitem{Type}
\sphinxAtStartPar
ndarray

\end{description}\end{quote}

\end{fulllineitems}\end{savenotes}

\index{header (lezargus.container.cube.LezargusCube attribute)@\spxentry{header}\spxextra{lezargus.container.cube.LezargusCube attribute}}

\begin{savenotes}\begin{fulllineitems}
\phantomsection\label{\detokenize{code/lezargus.container.cube:lezargus.container.cube.LezargusCube.header}}
\pysigstartsignatures
\pysigline{\sphinxbfcode{\sphinxupquote{header}}}
\pysigstopsignatures
\sphinxAtStartPar
The header information, or metadata in general, about the data.
\begin{quote}\begin{description}
\sphinxlineitem{Type}
\sphinxAtStartPar
Header

\end{description}\end{quote}

\end{fulllineitems}\end{savenotes}

\index{\_\_init\_\_() (lezargus.container.cube.LezargusCube method)@\spxentry{\_\_init\_\_()}\spxextra{lezargus.container.cube.LezargusCube method}}

\begin{savenotes}\begin{fulllineitems}
\phantomsection\label{\detokenize{code/lezargus.container.cube:lezargus.container.cube.LezargusCube.__init__}}
\pysigstartsignatures
\pysiglinewithargsret{\sphinxbfcode{\sphinxupquote{\_\_init\_\_}}}{\sphinxparam{\DUrole{n,n}{wavelength}\DUrole{p,p}{:}\DUrole{w,w}{  }\DUrole{n,n}{\sphinxhref{https://numpy.org/doc/stable/reference/generated/numpy.ndarray.html\#numpy.ndarray}{ndarray}%
\begin{footnote}[10]\sphinxAtStartFootnote
\sphinxnolinkurl{https://numpy.org/doc/stable/reference/generated/numpy.ndarray.html\#numpy.ndarray}
%
\end{footnote}}}, \sphinxparam{\DUrole{n,n}{data}\DUrole{p,p}{:}\DUrole{w,w}{  }\DUrole{n,n}{\sphinxhref{https://numpy.org/doc/stable/reference/generated/numpy.ndarray.html\#numpy.ndarray}{ndarray}%
\begin{footnote}[11]\sphinxAtStartFootnote
\sphinxnolinkurl{https://numpy.org/doc/stable/reference/generated/numpy.ndarray.html\#numpy.ndarray}
%
\end{footnote}}}, \sphinxparam{\DUrole{n,n}{uncertainty}\DUrole{p,p}{:}\DUrole{w,w}{  }\DUrole{n,n}{\sphinxhref{https://numpy.org/doc/stable/reference/generated/numpy.ndarray.html\#numpy.ndarray}{ndarray}%
\begin{footnote}[12]\sphinxAtStartFootnote
\sphinxnolinkurl{https://numpy.org/doc/stable/reference/generated/numpy.ndarray.html\#numpy.ndarray}
%
\end{footnote}}\DUrole{w,w}{  }\DUrole{o,o}{=}\DUrole{w,w}{  }\DUrole{default_value}{None}}, \sphinxparam{\DUrole{n,n}{wavelength\_unit}\DUrole{p,p}{:}\DUrole{w,w}{  }\DUrole{n,n}{str\DUrole{w,w}{  }\DUrole{p,p}{|}\DUrole{w,w}{  }\sphinxhref{https://docs.astropy.org/en/stable/api/astropy.units.Unit.html\#astropy.units.Unit}{Unit}%
\begin{footnote}[13]\sphinxAtStartFootnote
\sphinxnolinkurl{https://docs.astropy.org/en/stable/api/astropy.units.Unit.html\#astropy.units.Unit}
%
\end{footnote}\DUrole{w,w}{  }\DUrole{p,p}{|}\DUrole{w,w}{  }None}\DUrole{w,w}{  }\DUrole{o,o}{=}\DUrole{w,w}{  }\DUrole{default_value}{None}}, \sphinxparam{\DUrole{n,n}{data\_unit}\DUrole{p,p}{:}\DUrole{w,w}{  }\DUrole{n,n}{str\DUrole{w,w}{  }\DUrole{p,p}{|}\DUrole{w,w}{  }\sphinxhref{https://docs.astropy.org/en/stable/api/astropy.units.Unit.html\#astropy.units.Unit}{Unit}%
\begin{footnote}[14]\sphinxAtStartFootnote
\sphinxnolinkurl{https://docs.astropy.org/en/stable/api/astropy.units.Unit.html\#astropy.units.Unit}
%
\end{footnote}\DUrole{w,w}{  }\DUrole{p,p}{|}\DUrole{w,w}{  }None}\DUrole{w,w}{  }\DUrole{o,o}{=}\DUrole{w,w}{  }\DUrole{default_value}{None}}, \sphinxparam{\DUrole{n,n}{mask}\DUrole{p,p}{:}\DUrole{w,w}{  }\DUrole{n,n}{\sphinxhref{https://numpy.org/doc/stable/reference/generated/numpy.ndarray.html\#numpy.ndarray}{ndarray}%
\begin{footnote}[15]\sphinxAtStartFootnote
\sphinxnolinkurl{https://numpy.org/doc/stable/reference/generated/numpy.ndarray.html\#numpy.ndarray}
%
\end{footnote}\DUrole{w,w}{  }\DUrole{p,p}{|}\DUrole{w,w}{  }None}\DUrole{w,w}{  }\DUrole{o,o}{=}\DUrole{w,w}{  }\DUrole{default_value}{None}}, \sphinxparam{\DUrole{n,n}{flags}\DUrole{p,p}{:}\DUrole{w,w}{  }\DUrole{n,n}{\sphinxhref{https://numpy.org/doc/stable/reference/generated/numpy.ndarray.html\#numpy.ndarray}{ndarray}%
\begin{footnote}[16]\sphinxAtStartFootnote
\sphinxnolinkurl{https://numpy.org/doc/stable/reference/generated/numpy.ndarray.html\#numpy.ndarray}
%
\end{footnote}\DUrole{w,w}{  }\DUrole{p,p}{|}\DUrole{w,w}{  }None}\DUrole{w,w}{  }\DUrole{o,o}{=}\DUrole{w,w}{  }\DUrole{default_value}{None}}, \sphinxparam{\DUrole{n,n}{header}\DUrole{p,p}{:}\DUrole{w,w}{  }\DUrole{n,n}{\sphinxhref{https://docs.astropy.org/en/stable/io/fits/api/headers.html\#astropy.io.fits.Header}{Header}%
\begin{footnote}[17]\sphinxAtStartFootnote
\sphinxnolinkurl{https://docs.astropy.org/en/stable/io/fits/api/headers.html\#astropy.io.fits.Header}
%
\end{footnote}\DUrole{w,w}{  }\DUrole{p,p}{|}\DUrole{w,w}{  }None}\DUrole{w,w}{  }\DUrole{o,o}{=}\DUrole{w,w}{  }\DUrole{default_value}{None}}}{{ $\rightarrow$ None}}
\pysigstopsignatures
\sphinxAtStartPar
Instantiate the spectra class.
\begin{quote}\begin{description}
\sphinxlineitem{Parameters}\begin{itemize}
\item {} 
\sphinxAtStartPar
\sphinxstyleliteralstrong{\sphinxupquote{wavelength}} (\sphinxstyleliteralemphasis{\sphinxupquote{ndarray}}) – The wavelength of the spectra.

\item {} 
\sphinxAtStartPar
\sphinxstyleliteralstrong{\sphinxupquote{data}} (\sphinxstyleliteralemphasis{\sphinxupquote{ndarray}}) – The flux of the spectra.

\item {} 
\sphinxAtStartPar
\sphinxstyleliteralstrong{\sphinxupquote{uncertainty}} (\sphinxstyleliteralemphasis{\sphinxupquote{ndarray}}\sphinxstyleliteralemphasis{\sphinxupquote{, }}\sphinxstyleliteralemphasis{\sphinxupquote{default = None}}) – The uncertainty of the spectra. By default, it is None and the
uncertainty value is 0.

\item {} 
\sphinxAtStartPar
\sphinxstyleliteralstrong{\sphinxupquote{wavelength\_unit}} (\sphinxstyleliteralemphasis{\sphinxupquote{Astropy\sphinxhyphen{}Unit like}}\sphinxstyleliteralemphasis{\sphinxupquote{, }}\sphinxstyleliteralemphasis{\sphinxupquote{default = None}}) – The wavelength unit of the spectra. It must be interpretable by
the Astropy Units package. If None, the the unit is dimensionless.

\item {} 
\sphinxAtStartPar
\sphinxstyleliteralstrong{\sphinxupquote{data\_unit}} (\sphinxstyleliteralemphasis{\sphinxupquote{Astropy\sphinxhyphen{}Unit like}}\sphinxstyleliteralemphasis{\sphinxupquote{, }}\sphinxstyleliteralemphasis{\sphinxupquote{default = None}}) – The data unit of the spectra. It must be interpretable by
the Astropy Units package. If None, the the unit is dimensionless.

\item {} 
\sphinxAtStartPar
\sphinxstyleliteralstrong{\sphinxupquote{mask}} (\sphinxstyleliteralemphasis{\sphinxupquote{ndarray}}\sphinxstyleliteralemphasis{\sphinxupquote{, }}\sphinxstyleliteralemphasis{\sphinxupquote{default = None}}) – A mask which should be applied to the spectra, if needed.

\item {} 
\sphinxAtStartPar
\sphinxstyleliteralstrong{\sphinxupquote{flags}} (\sphinxstyleliteralemphasis{\sphinxupquote{ndarray}}\sphinxstyleliteralemphasis{\sphinxupquote{, }}\sphinxstyleliteralemphasis{\sphinxupquote{default = None}}) – A set of flags which describe specific points of data in the
spectra.

\item {} 
\sphinxAtStartPar
\sphinxstyleliteralstrong{\sphinxupquote{header}} (\sphinxstyleliteralemphasis{\sphinxupquote{Header}}\sphinxstyleliteralemphasis{\sphinxupquote{, }}\sphinxstyleliteralemphasis{\sphinxupquote{default = None}}) – A set of header data describing the data. Note that when saving,
this header is written to disk with minimal processing. We highly
suggest writing of the metadata to conform to the FITS Header
specification as much as possible.

\end{itemize}

\end{description}\end{quote}

\end{fulllineitems}\end{savenotes}

\index{read\_fits\_file() (lezargus.container.cube.LezargusCube class method)@\spxentry{read\_fits\_file()}\spxextra{lezargus.container.cube.LezargusCube class method}}

\begin{savenotes}\begin{fulllineitems}
\phantomsection\label{\detokenize{code/lezargus.container.cube:lezargus.container.cube.LezargusCube.read_fits_file}}
\pysigstartsignatures
\pysiglinewithargsret{\sphinxbfcode{\sphinxupquote{classmethod\DUrole{w,w}{  }}}\sphinxbfcode{\sphinxupquote{read\_fits\_file}}}{\sphinxparam{\DUrole{n,n}{filename}\DUrole{p,p}{:}\DUrole{w,w}{  }\DUrole{n,n}{str}}}{{ $\rightarrow$ Self}}
\pysigstopsignatures
\sphinxAtStartPar
Read a Lezargus cube FITS file.

\sphinxAtStartPar
We load a Lezargus FITS file from disk. Note that this should only
be used for 3\sphinxhyphen{}D cube files.
\begin{quote}\begin{description}
\sphinxlineitem{Parameters}
\sphinxAtStartPar
\sphinxstyleliteralstrong{\sphinxupquote{filename}} (\sphinxstyleliteralemphasis{\sphinxupquote{str}}) – The filename to load.

\sphinxlineitem{Returns}
\sphinxAtStartPar
\sphinxstylestrong{cube} – The LezargusCube class instance.

\sphinxlineitem{Return type}
\sphinxAtStartPar
Self\sphinxhyphen{}like

\end{description}\end{quote}

\end{fulllineitems}\end{savenotes}

\index{write\_fits\_file() (lezargus.container.cube.LezargusCube method)@\spxentry{write\_fits\_file()}\spxextra{lezargus.container.cube.LezargusCube method}}

\begin{savenotes}\begin{fulllineitems}
\phantomsection\label{\detokenize{code/lezargus.container.cube:lezargus.container.cube.LezargusCube.write_fits_file}}
\pysigstartsignatures
\pysiglinewithargsret{\sphinxbfcode{\sphinxupquote{write\_fits\_file}}}{\sphinxparam{\DUrole{n,n}{filename}\DUrole{p,p}{:}\DUrole{w,w}{  }\DUrole{n,n}{str}}, \sphinxparam{\DUrole{n,n}{overwrite}\DUrole{p,p}{:}\DUrole{w,w}{  }\DUrole{n,n}{bool}\DUrole{w,w}{  }\DUrole{o,o}{=}\DUrole{w,w}{  }\DUrole{default_value}{False}}}{{ $\rightarrow$ Self}}
\pysigstopsignatures
\sphinxAtStartPar
Write a Lezargus cube FITS file.

\sphinxAtStartPar
We write a Lezargus FITS file to disk.
\begin{quote}\begin{description}
\sphinxlineitem{Parameters}\begin{itemize}
\item {} 
\sphinxAtStartPar
\sphinxstyleliteralstrong{\sphinxupquote{filename}} (\sphinxstyleliteralemphasis{\sphinxupquote{str}}) – The filename to write to.

\item {} 
\sphinxAtStartPar
\sphinxstyleliteralstrong{\sphinxupquote{overwrite}} (\sphinxstyleliteralemphasis{\sphinxupquote{bool}}\sphinxstyleliteralemphasis{\sphinxupquote{, }}\sphinxstyleliteralemphasis{\sphinxupquote{default = False}}) – If True, overwrite file conflicts.

\end{itemize}

\sphinxlineitem{Return type}
\sphinxAtStartPar
None

\end{description}\end{quote}

\end{fulllineitems}\end{savenotes}


\end{fulllineitems}\end{savenotes}


\sphinxstepscope


\subparagraph{lezargus.container.image module}
\label{\detokenize{code/lezargus.container.image:module-lezargus.container.image}}\label{\detokenize{code/lezargus.container.image:lezargus-container-image-module}}\label{\detokenize{code/lezargus.container.image::doc}}\index{module@\spxentry{module}!lezargus.container.image@\spxentry{lezargus.container.image}}\index{lezargus.container.image@\spxentry{lezargus.container.image}!module@\spxentry{module}}
\sphinxAtStartPar
Image data container.

\sphinxAtStartPar
This module and class primarily deals with images containing spatial
information.
\index{LezargusImage (class in lezargus.container.image)@\spxentry{LezargusImage}\spxextra{class in lezargus.container.image}}

\begin{savenotes}\begin{fulllineitems}
\phantomsection\label{\detokenize{code/lezargus.container.image:lezargus.container.image.LezargusImage}}
\pysigstartsignatures
\pysiglinewithargsret{\sphinxbfcode{\sphinxupquote{class\DUrole{w,w}{  }}}\sphinxcode{\sphinxupquote{lezargus.container.image.}}\sphinxbfcode{\sphinxupquote{LezargusImage}}}{\sphinxparam{\DUrole{n,n}{data}\DUrole{p,p}{:}\DUrole{w,w}{  }\DUrole{n,n}{\sphinxhref{https://numpy.org/doc/stable/reference/generated/numpy.ndarray.html\#numpy.ndarray}{ndarray}%
\begin{footnote}[18]\sphinxAtStartFootnote
\sphinxnolinkurl{https://numpy.org/doc/stable/reference/generated/numpy.ndarray.html\#numpy.ndarray}
%
\end{footnote}}}, \sphinxparam{\DUrole{n,n}{uncertainty}\DUrole{p,p}{:}\DUrole{w,w}{  }\DUrole{n,n}{\sphinxhref{https://numpy.org/doc/stable/reference/generated/numpy.ndarray.html\#numpy.ndarray}{ndarray}%
\begin{footnote}[19]\sphinxAtStartFootnote
\sphinxnolinkurl{https://numpy.org/doc/stable/reference/generated/numpy.ndarray.html\#numpy.ndarray}
%
\end{footnote}\DUrole{w,w}{  }\DUrole{p,p}{|}\DUrole{w,w}{  }None}\DUrole{w,w}{  }\DUrole{o,o}{=}\DUrole{w,w}{  }\DUrole{default_value}{None}}, \sphinxparam{\DUrole{n,n}{wavelength}\DUrole{p,p}{:}\DUrole{w,w}{  }\DUrole{n,n}{float\DUrole{w,w}{  }\DUrole{p,p}{|}\DUrole{w,w}{  }None}\DUrole{w,w}{  }\DUrole{o,o}{=}\DUrole{w,w}{  }\DUrole{default_value}{None}}, \sphinxparam{\DUrole{n,n}{wavelength\_unit}\DUrole{p,p}{:}\DUrole{w,w}{  }\DUrole{n,n}{str\DUrole{w,w}{  }\DUrole{p,p}{|}\DUrole{w,w}{  }\sphinxhref{https://docs.astropy.org/en/stable/api/astropy.units.Unit.html\#astropy.units.Unit}{Unit}%
\begin{footnote}[20]\sphinxAtStartFootnote
\sphinxnolinkurl{https://docs.astropy.org/en/stable/api/astropy.units.Unit.html\#astropy.units.Unit}
%
\end{footnote}}\DUrole{w,w}{  }\DUrole{o,o}{=}\DUrole{w,w}{  }\DUrole{default_value}{None}}, \sphinxparam{\DUrole{n,n}{data\_unit}\DUrole{p,p}{:}\DUrole{w,w}{  }\DUrole{n,n}{str\DUrole{w,w}{  }\DUrole{p,p}{|}\DUrole{w,w}{  }\sphinxhref{https://docs.astropy.org/en/stable/api/astropy.units.Unit.html\#astropy.units.Unit}{Unit}%
\begin{footnote}[21]\sphinxAtStartFootnote
\sphinxnolinkurl{https://docs.astropy.org/en/stable/api/astropy.units.Unit.html\#astropy.units.Unit}
%
\end{footnote}\DUrole{w,w}{  }\DUrole{p,p}{|}\DUrole{w,w}{  }None}\DUrole{w,w}{  }\DUrole{o,o}{=}\DUrole{w,w}{  }\DUrole{default_value}{None}}, \sphinxparam{\DUrole{n,n}{mask}\DUrole{p,p}{:}\DUrole{w,w}{  }\DUrole{n,n}{\sphinxhref{https://numpy.org/doc/stable/reference/generated/numpy.ndarray.html\#numpy.ndarray}{ndarray}%
\begin{footnote}[22]\sphinxAtStartFootnote
\sphinxnolinkurl{https://numpy.org/doc/stable/reference/generated/numpy.ndarray.html\#numpy.ndarray}
%
\end{footnote}\DUrole{w,w}{  }\DUrole{p,p}{|}\DUrole{w,w}{  }None}\DUrole{w,w}{  }\DUrole{o,o}{=}\DUrole{w,w}{  }\DUrole{default_value}{None}}, \sphinxparam{\DUrole{n,n}{flags}\DUrole{p,p}{:}\DUrole{w,w}{  }\DUrole{n,n}{\sphinxhref{https://numpy.org/doc/stable/reference/generated/numpy.ndarray.html\#numpy.ndarray}{ndarray}%
\begin{footnote}[23]\sphinxAtStartFootnote
\sphinxnolinkurl{https://numpy.org/doc/stable/reference/generated/numpy.ndarray.html\#numpy.ndarray}
%
\end{footnote}\DUrole{w,w}{  }\DUrole{p,p}{|}\DUrole{w,w}{  }None}\DUrole{w,w}{  }\DUrole{o,o}{=}\DUrole{w,w}{  }\DUrole{default_value}{None}}, \sphinxparam{\DUrole{n,n}{header}\DUrole{p,p}{:}\DUrole{w,w}{  }\DUrole{n,n}{\sphinxhref{https://docs.astropy.org/en/stable/io/fits/api/headers.html\#astropy.io.fits.Header}{Header}%
\begin{footnote}[24]\sphinxAtStartFootnote
\sphinxnolinkurl{https://docs.astropy.org/en/stable/io/fits/api/headers.html\#astropy.io.fits.Header}
%
\end{footnote}\DUrole{w,w}{  }\DUrole{p,p}{|}\DUrole{w,w}{  }None}\DUrole{w,w}{  }\DUrole{o,o}{=}\DUrole{w,w}{  }\DUrole{default_value}{None}}}{}
\pysigstopsignatures
\sphinxAtStartPar
Bases: {\hyperref[\detokenize{code/lezargus.container.parent:lezargus.container.parent.LezargusContainerArithmetic}]{\sphinxcrossref{\sphinxcode{\sphinxupquote{LezargusContainerArithmetic}}}}}

\sphinxAtStartPar
Container to hold image and perform operations on it.
\index{wavelength (lezargus.container.image.LezargusImage attribute)@\spxentry{wavelength}\spxextra{lezargus.container.image.LezargusImage attribute}}

\begin{savenotes}\begin{fulllineitems}
\phantomsection\label{\detokenize{code/lezargus.container.image:lezargus.container.image.LezargusImage.wavelength}}
\pysigstartsignatures
\pysigline{\sphinxbfcode{\sphinxupquote{wavelength}}}
\pysigstopsignatures
\sphinxAtStartPar
The wavelength of the image. The unit of wavelength is typically
in microns; but, check the \sphinxtitleref{wavelength\_unit} value. If none has
been provided, this value is an array of None.
\begin{quote}\begin{description}
\sphinxlineitem{Type}
\sphinxAtStartPar
float

\end{description}\end{quote}

\end{fulllineitems}\end{savenotes}

\index{data (lezargus.container.image.LezargusImage attribute)@\spxentry{data}\spxextra{lezargus.container.image.LezargusImage attribute}}

\begin{savenotes}\begin{fulllineitems}
\phantomsection\label{\detokenize{code/lezargus.container.image:lezargus.container.image.LezargusImage.data}}
\pysigstartsignatures
\pysigline{\sphinxbfcode{\sphinxupquote{data}}}
\pysigstopsignatures
\sphinxAtStartPar
The flux of the spectra cube. The unit of the flux is typically
in flam; but, check the \sphinxtitleref{flux\_unit} value.
\begin{quote}\begin{description}
\sphinxlineitem{Type}
\sphinxAtStartPar
ndarray

\end{description}\end{quote}

\end{fulllineitems}\end{savenotes}

\index{uncertainty (lezargus.container.image.LezargusImage attribute)@\spxentry{uncertainty}\spxextra{lezargus.container.image.LezargusImage attribute}}

\begin{savenotes}\begin{fulllineitems}
\phantomsection\label{\detokenize{code/lezargus.container.image:lezargus.container.image.LezargusImage.uncertainty}}
\pysigstartsignatures
\pysigline{\sphinxbfcode{\sphinxupquote{uncertainty}}}
\pysigstopsignatures
\sphinxAtStartPar
The uncertainty in the flux of the spectra. The unit of the uncertainty
is the same as the flux value; per \sphinxtitleref{uncertainty\_unit}.
\begin{quote}\begin{description}
\sphinxlineitem{Type}
\sphinxAtStartPar
ndarray

\end{description}\end{quote}

\end{fulllineitems}\end{savenotes}

\index{wavelength\_unit (lezargus.container.image.LezargusImage attribute)@\spxentry{wavelength\_unit}\spxextra{lezargus.container.image.LezargusImage attribute}}

\begin{savenotes}\begin{fulllineitems}
\phantomsection\label{\detokenize{code/lezargus.container.image:lezargus.container.image.LezargusImage.wavelength_unit}}
\pysigstartsignatures
\pysigline{\sphinxbfcode{\sphinxupquote{wavelength\_unit}}}
\pysigstopsignatures
\sphinxAtStartPar
The unit of the wavelength array.
\begin{quote}\begin{description}
\sphinxlineitem{Type}
\sphinxAtStartPar
Astropy Unit

\end{description}\end{quote}

\end{fulllineitems}\end{savenotes}

\index{flux\_unit (lezargus.container.image.LezargusImage attribute)@\spxentry{flux\_unit}\spxextra{lezargus.container.image.LezargusImage attribute}}

\begin{savenotes}\begin{fulllineitems}
\phantomsection\label{\detokenize{code/lezargus.container.image:lezargus.container.image.LezargusImage.flux_unit}}
\pysigstartsignatures
\pysigline{\sphinxbfcode{\sphinxupquote{flux\_unit}}}
\pysigstopsignatures
\sphinxAtStartPar
The unit of the flux array.
\begin{quote}\begin{description}
\sphinxlineitem{Type}
\sphinxAtStartPar
Astropy Unit

\end{description}\end{quote}

\end{fulllineitems}\end{savenotes}

\index{uncertainty\_unit (lezargus.container.image.LezargusImage attribute)@\spxentry{uncertainty\_unit}\spxextra{lezargus.container.image.LezargusImage attribute}}

\begin{savenotes}\begin{fulllineitems}
\phantomsection\label{\detokenize{code/lezargus.container.image:lezargus.container.image.LezargusImage.uncertainty_unit}}
\pysigstartsignatures
\pysigline{\sphinxbfcode{\sphinxupquote{uncertainty\_unit}}}
\pysigstopsignatures
\sphinxAtStartPar
The unit of the uncertainty array. This unit is the same as the flux
unit.
\begin{quote}\begin{description}
\sphinxlineitem{Type}
\sphinxAtStartPar
Astropy Unit

\end{description}\end{quote}

\end{fulllineitems}\end{savenotes}

\index{mask (lezargus.container.image.LezargusImage attribute)@\spxentry{mask}\spxextra{lezargus.container.image.LezargusImage attribute}}

\begin{savenotes}\begin{fulllineitems}
\phantomsection\label{\detokenize{code/lezargus.container.image:lezargus.container.image.LezargusImage.mask}}
\pysigstartsignatures
\pysigline{\sphinxbfcode{\sphinxupquote{mask}}}
\pysigstopsignatures
\sphinxAtStartPar
A mask of the flux data, used to remove problematic areas. Where True,
the values of the flux is considered mask.
\begin{quote}\begin{description}
\sphinxlineitem{Type}
\sphinxAtStartPar
ndarray

\end{description}\end{quote}

\end{fulllineitems}\end{savenotes}

\index{flags (lezargus.container.image.LezargusImage attribute)@\spxentry{flags}\spxextra{lezargus.container.image.LezargusImage attribute}}

\begin{savenotes}\begin{fulllineitems}
\phantomsection\label{\detokenize{code/lezargus.container.image:lezargus.container.image.LezargusImage.flags}}
\pysigstartsignatures
\pysigline{\sphinxbfcode{\sphinxupquote{flags}}}
\pysigstopsignatures
\sphinxAtStartPar
Flags of the flux data. These flags store metadata about the flux.
\begin{quote}\begin{description}
\sphinxlineitem{Type}
\sphinxAtStartPar
ndarray

\end{description}\end{quote}

\end{fulllineitems}\end{savenotes}

\index{header (lezargus.container.image.LezargusImage attribute)@\spxentry{header}\spxextra{lezargus.container.image.LezargusImage attribute}}

\begin{savenotes}\begin{fulllineitems}
\phantomsection\label{\detokenize{code/lezargus.container.image:lezargus.container.image.LezargusImage.header}}
\pysigstartsignatures
\pysigline{\sphinxbfcode{\sphinxupquote{header}}}
\pysigstopsignatures
\sphinxAtStartPar
The header information, or metadata in general, about the data.
\begin{quote}\begin{description}
\sphinxlineitem{Type}
\sphinxAtStartPar
Header

\end{description}\end{quote}

\end{fulllineitems}\end{savenotes}

\index{\_\_init\_\_() (lezargus.container.image.LezargusImage method)@\spxentry{\_\_init\_\_()}\spxextra{lezargus.container.image.LezargusImage method}}

\begin{savenotes}\begin{fulllineitems}
\phantomsection\label{\detokenize{code/lezargus.container.image:lezargus.container.image.LezargusImage.__init__}}
\pysigstartsignatures
\pysiglinewithargsret{\sphinxbfcode{\sphinxupquote{\_\_init\_\_}}}{\sphinxparam{\DUrole{n,n}{data}\DUrole{p,p}{:}\DUrole{w,w}{  }\DUrole{n,n}{\sphinxhref{https://numpy.org/doc/stable/reference/generated/numpy.ndarray.html\#numpy.ndarray}{ndarray}%
\begin{footnote}[25]\sphinxAtStartFootnote
\sphinxnolinkurl{https://numpy.org/doc/stable/reference/generated/numpy.ndarray.html\#numpy.ndarray}
%
\end{footnote}}}, \sphinxparam{\DUrole{n,n}{uncertainty}\DUrole{p,p}{:}\DUrole{w,w}{  }\DUrole{n,n}{\sphinxhref{https://numpy.org/doc/stable/reference/generated/numpy.ndarray.html\#numpy.ndarray}{ndarray}%
\begin{footnote}[26]\sphinxAtStartFootnote
\sphinxnolinkurl{https://numpy.org/doc/stable/reference/generated/numpy.ndarray.html\#numpy.ndarray}
%
\end{footnote}\DUrole{w,w}{  }\DUrole{p,p}{|}\DUrole{w,w}{  }None}\DUrole{w,w}{  }\DUrole{o,o}{=}\DUrole{w,w}{  }\DUrole{default_value}{None}}, \sphinxparam{\DUrole{n,n}{wavelength}\DUrole{p,p}{:}\DUrole{w,w}{  }\DUrole{n,n}{float\DUrole{w,w}{  }\DUrole{p,p}{|}\DUrole{w,w}{  }None}\DUrole{w,w}{  }\DUrole{o,o}{=}\DUrole{w,w}{  }\DUrole{default_value}{None}}, \sphinxparam{\DUrole{n,n}{wavelength\_unit}\DUrole{p,p}{:}\DUrole{w,w}{  }\DUrole{n,n}{str\DUrole{w,w}{  }\DUrole{p,p}{|}\DUrole{w,w}{  }\sphinxhref{https://docs.astropy.org/en/stable/api/astropy.units.Unit.html\#astropy.units.Unit}{Unit}%
\begin{footnote}[27]\sphinxAtStartFootnote
\sphinxnolinkurl{https://docs.astropy.org/en/stable/api/astropy.units.Unit.html\#astropy.units.Unit}
%
\end{footnote}}\DUrole{w,w}{  }\DUrole{o,o}{=}\DUrole{w,w}{  }\DUrole{default_value}{None}}, \sphinxparam{\DUrole{n,n}{data\_unit}\DUrole{p,p}{:}\DUrole{w,w}{  }\DUrole{n,n}{str\DUrole{w,w}{  }\DUrole{p,p}{|}\DUrole{w,w}{  }\sphinxhref{https://docs.astropy.org/en/stable/api/astropy.units.Unit.html\#astropy.units.Unit}{Unit}%
\begin{footnote}[28]\sphinxAtStartFootnote
\sphinxnolinkurl{https://docs.astropy.org/en/stable/api/astropy.units.Unit.html\#astropy.units.Unit}
%
\end{footnote}\DUrole{w,w}{  }\DUrole{p,p}{|}\DUrole{w,w}{  }None}\DUrole{w,w}{  }\DUrole{o,o}{=}\DUrole{w,w}{  }\DUrole{default_value}{None}}, \sphinxparam{\DUrole{n,n}{mask}\DUrole{p,p}{:}\DUrole{w,w}{  }\DUrole{n,n}{\sphinxhref{https://numpy.org/doc/stable/reference/generated/numpy.ndarray.html\#numpy.ndarray}{ndarray}%
\begin{footnote}[29]\sphinxAtStartFootnote
\sphinxnolinkurl{https://numpy.org/doc/stable/reference/generated/numpy.ndarray.html\#numpy.ndarray}
%
\end{footnote}\DUrole{w,w}{  }\DUrole{p,p}{|}\DUrole{w,w}{  }None}\DUrole{w,w}{  }\DUrole{o,o}{=}\DUrole{w,w}{  }\DUrole{default_value}{None}}, \sphinxparam{\DUrole{n,n}{flags}\DUrole{p,p}{:}\DUrole{w,w}{  }\DUrole{n,n}{\sphinxhref{https://numpy.org/doc/stable/reference/generated/numpy.ndarray.html\#numpy.ndarray}{ndarray}%
\begin{footnote}[30]\sphinxAtStartFootnote
\sphinxnolinkurl{https://numpy.org/doc/stable/reference/generated/numpy.ndarray.html\#numpy.ndarray}
%
\end{footnote}\DUrole{w,w}{  }\DUrole{p,p}{|}\DUrole{w,w}{  }None}\DUrole{w,w}{  }\DUrole{o,o}{=}\DUrole{w,w}{  }\DUrole{default_value}{None}}, \sphinxparam{\DUrole{n,n}{header}\DUrole{p,p}{:}\DUrole{w,w}{  }\DUrole{n,n}{\sphinxhref{https://docs.astropy.org/en/stable/io/fits/api/headers.html\#astropy.io.fits.Header}{Header}%
\begin{footnote}[31]\sphinxAtStartFootnote
\sphinxnolinkurl{https://docs.astropy.org/en/stable/io/fits/api/headers.html\#astropy.io.fits.Header}
%
\end{footnote}\DUrole{w,w}{  }\DUrole{p,p}{|}\DUrole{w,w}{  }None}\DUrole{w,w}{  }\DUrole{o,o}{=}\DUrole{w,w}{  }\DUrole{default_value}{None}}}{{ $\rightarrow$ None}}
\pysigstopsignatures
\sphinxAtStartPar
Instantiate the spectra class.
\begin{quote}\begin{description}
\sphinxlineitem{Parameters}\begin{itemize}
\item {} 
\sphinxAtStartPar
\sphinxstyleliteralstrong{\sphinxupquote{data}} (\sphinxstyleliteralemphasis{\sphinxupquote{ndarray}}) – The flux of the spectra.

\item {} 
\sphinxAtStartPar
\sphinxstyleliteralstrong{\sphinxupquote{uncertainty}} (\sphinxstyleliteralemphasis{\sphinxupquote{ndarray}}\sphinxstyleliteralemphasis{\sphinxupquote{, }}\sphinxstyleliteralemphasis{\sphinxupquote{default = None}}) – The uncertainty of the spectra. By default, it is None and the
uncertainty value is 0.

\item {} 
\sphinxAtStartPar
\sphinxstyleliteralstrong{\sphinxupquote{wavelength}} (\sphinxstyleliteralemphasis{\sphinxupquote{ndarray}}\sphinxstyleliteralemphasis{\sphinxupquote{, }}\sphinxstyleliteralemphasis{\sphinxupquote{default = None}}) – The wavelength of the image. If this is not provided, it defaults
to 0, otherwise, it is an array of a single value.

\item {} 
\sphinxAtStartPar
\sphinxstyleliteralstrong{\sphinxupquote{wavelength\_unit}} (\sphinxstyleliteralemphasis{\sphinxupquote{Astropy\sphinxhyphen{}Unit like}}\sphinxstyleliteralemphasis{\sphinxupquote{, }}\sphinxstyleliteralemphasis{\sphinxupquote{default = None}}) – The wavelength unit of the spectra. It must be interpretable by
the Astropy Units package. If None, the the unit is dimensionless.

\item {} 
\sphinxAtStartPar
\sphinxstyleliteralstrong{\sphinxupquote{data\_unit}} (\sphinxstyleliteralemphasis{\sphinxupquote{Astropy\sphinxhyphen{}Unit like}}\sphinxstyleliteralemphasis{\sphinxupquote{, }}\sphinxstyleliteralemphasis{\sphinxupquote{default = None}}) – The data unit of the spectra. It must be interpretable by
the Astropy Units package. If None, the the unit is dimensionless.

\item {} 
\sphinxAtStartPar
\sphinxstyleliteralstrong{\sphinxupquote{mask}} (\sphinxstyleliteralemphasis{\sphinxupquote{ndarray}}\sphinxstyleliteralemphasis{\sphinxupquote{, }}\sphinxstyleliteralemphasis{\sphinxupquote{default = None}}) – A mask which should be applied to the spectra, if needed.

\item {} 
\sphinxAtStartPar
\sphinxstyleliteralstrong{\sphinxupquote{flags}} (\sphinxstyleliteralemphasis{\sphinxupquote{ndarray}}\sphinxstyleliteralemphasis{\sphinxupquote{, }}\sphinxstyleliteralemphasis{\sphinxupquote{default = None}}) – A set of flags which describe specific points of data in the
spectra.

\item {} 
\sphinxAtStartPar
\sphinxstyleliteralstrong{\sphinxupquote{header}} (\sphinxstyleliteralemphasis{\sphinxupquote{Header}}\sphinxstyleliteralemphasis{\sphinxupquote{, }}\sphinxstyleliteralemphasis{\sphinxupquote{default = None}}) – A set of header data describing the data. Note that when saving,
this header is written to disk with minimal processing. We highly
suggest writing of the metadata to conform to the FITS Header
specification as much as possible.

\end{itemize}

\end{description}\end{quote}

\end{fulllineitems}\end{savenotes}

\index{read\_fits\_file() (lezargus.container.image.LezargusImage class method)@\spxentry{read\_fits\_file()}\spxextra{lezargus.container.image.LezargusImage class method}}

\begin{savenotes}\begin{fulllineitems}
\phantomsection\label{\detokenize{code/lezargus.container.image:lezargus.container.image.LezargusImage.read_fits_file}}
\pysigstartsignatures
\pysiglinewithargsret{\sphinxbfcode{\sphinxupquote{classmethod\DUrole{w,w}{  }}}\sphinxbfcode{\sphinxupquote{read\_fits\_file}}}{\sphinxparam{\DUrole{n,n}{filename}\DUrole{p,p}{:}\DUrole{w,w}{  }\DUrole{n,n}{str}}}{{ $\rightarrow$ Self}}
\pysigstopsignatures
\sphinxAtStartPar
Read a Lezargus image FITS file.

\sphinxAtStartPar
We load a Lezargus FITS file from disk. Note that this should only
be used for 2\sphinxhyphen{}D image files.
\begin{quote}\begin{description}
\sphinxlineitem{Parameters}
\sphinxAtStartPar
\sphinxstyleliteralstrong{\sphinxupquote{filename}} (\sphinxstyleliteralemphasis{\sphinxupquote{str}}) – The filename to load.

\sphinxlineitem{Returns}
\sphinxAtStartPar
\sphinxstylestrong{cube} – The LezargusImage class instance.

\sphinxlineitem{Return type}
\sphinxAtStartPar
Self\sphinxhyphen{}like

\end{description}\end{quote}

\end{fulllineitems}\end{savenotes}

\index{write\_fits\_file() (lezargus.container.image.LezargusImage method)@\spxentry{write\_fits\_file()}\spxextra{lezargus.container.image.LezargusImage method}}

\begin{savenotes}\begin{fulllineitems}
\phantomsection\label{\detokenize{code/lezargus.container.image:lezargus.container.image.LezargusImage.write_fits_file}}
\pysigstartsignatures
\pysiglinewithargsret{\sphinxbfcode{\sphinxupquote{write\_fits\_file}}}{\sphinxparam{\DUrole{n,n}{filename}\DUrole{p,p}{:}\DUrole{w,w}{  }\DUrole{n,n}{str}}, \sphinxparam{\DUrole{n,n}{overwrite}\DUrole{p,p}{:}\DUrole{w,w}{  }\DUrole{n,n}{bool}\DUrole{w,w}{  }\DUrole{o,o}{=}\DUrole{w,w}{  }\DUrole{default_value}{False}}}{{ $\rightarrow$ Self}}
\pysigstopsignatures
\sphinxAtStartPar
Write a Lezargus image FITS file.

\sphinxAtStartPar
We write a Lezargus FITS file to disk.
\begin{quote}\begin{description}
\sphinxlineitem{Parameters}\begin{itemize}
\item {} 
\sphinxAtStartPar
\sphinxstyleliteralstrong{\sphinxupquote{filename}} (\sphinxstyleliteralemphasis{\sphinxupquote{str}}) – The filename to write to.

\item {} 
\sphinxAtStartPar
\sphinxstyleliteralstrong{\sphinxupquote{overwrite}} (\sphinxstyleliteralemphasis{\sphinxupquote{bool}}\sphinxstyleliteralemphasis{\sphinxupquote{, }}\sphinxstyleliteralemphasis{\sphinxupquote{default = False}}) – If True, overwrite file conflicts.

\end{itemize}

\sphinxlineitem{Return type}
\sphinxAtStartPar
None

\end{description}\end{quote}

\end{fulllineitems}\end{savenotes}


\end{fulllineitems}\end{savenotes}


\sphinxstepscope


\subparagraph{lezargus.container.mosaic module}
\label{\detokenize{code/lezargus.container.mosaic:module-lezargus.container.mosaic}}\label{\detokenize{code/lezargus.container.mosaic:lezargus-container-mosaic-module}}\label{\detokenize{code/lezargus.container.mosaic::doc}}\index{module@\spxentry{module}!lezargus.container.mosaic@\spxentry{lezargus.container.mosaic}}\index{lezargus.container.mosaic@\spxentry{lezargus.container.mosaic}!module@\spxentry{module}}
\sphinxAtStartPar
Mosaic data container.

\sphinxAtStartPar
This module and class primarily deals with a collection of data cubes pieced
together into a single combined mosaic. Unlike the previous containers, this
does not directly subclass Astropy NDData and instead acts as a collection of
other reduced spectral cubes and performs operations on it.
\index{LezargusMosaic (class in lezargus.container.mosaic)@\spxentry{LezargusMosaic}\spxextra{class in lezargus.container.mosaic}}

\begin{savenotes}\begin{fulllineitems}
\phantomsection\label{\detokenize{code/lezargus.container.mosaic:lezargus.container.mosaic.LezargusMosaic}}
\pysigstartsignatures
\pysigline{\sphinxbfcode{\sphinxupquote{class\DUrole{w,w}{  }}}\sphinxcode{\sphinxupquote{lezargus.container.mosaic.}}\sphinxbfcode{\sphinxupquote{LezargusMosaic}}}
\pysigstopsignatures
\sphinxAtStartPar
Bases: \sphinxcode{\sphinxupquote{object}}

\sphinxAtStartPar
TODO.
\index{\_\_init\_\_() (lezargus.container.mosaic.LezargusMosaic method)@\spxentry{\_\_init\_\_()}\spxextra{lezargus.container.mosaic.LezargusMosaic method}}

\begin{savenotes}\begin{fulllineitems}
\phantomsection\label{\detokenize{code/lezargus.container.mosaic:lezargus.container.mosaic.LezargusMosaic.__init__}}
\pysigstartsignatures
\pysiglinewithargsret{\sphinxbfcode{\sphinxupquote{\_\_init\_\_}}}{}{{ $\rightarrow$ None}}
\pysigstopsignatures
\sphinxAtStartPar
Init doc.

\end{fulllineitems}\end{savenotes}


\end{fulllineitems}\end{savenotes}


\sphinxstepscope


\subparagraph{lezargus.container.parent module}
\label{\detokenize{code/lezargus.container.parent:module-lezargus.container.parent}}\label{\detokenize{code/lezargus.container.parent:lezargus-container-parent-module}}\label{\detokenize{code/lezargus.container.parent::doc}}\index{module@\spxentry{module}!lezargus.container.parent@\spxentry{lezargus.container.parent}}\index{lezargus.container.parent@\spxentry{lezargus.container.parent}!module@\spxentry{module}}
\sphinxAtStartPar
Parent class for the containers to implement arithmetic and other functions.

\sphinxAtStartPar
The Astropy NDArrayData arithmetic class is not wavelength aware. This class
overwrites and wraps around the NDArithmeticMixin class and allows it to be
wavelength aware. We also avoid the need to do a lot of the recreating of the
data object.
\index{LezargusContainerArithmetic (class in lezargus.container.parent)@\spxentry{LezargusContainerArithmetic}\spxextra{class in lezargus.container.parent}}

\begin{savenotes}\begin{fulllineitems}
\phantomsection\label{\detokenize{code/lezargus.container.parent:lezargus.container.parent.LezargusContainerArithmetic}}
\pysigstartsignatures
\pysiglinewithargsret{\sphinxbfcode{\sphinxupquote{class\DUrole{w,w}{  }}}\sphinxcode{\sphinxupquote{lezargus.container.parent.}}\sphinxbfcode{\sphinxupquote{LezargusContainerArithmetic}}}{\sphinxparam{\DUrole{n,n}{wavelength}\DUrole{p,p}{:}\DUrole{w,w}{  }\DUrole{n,n}{\sphinxhref{https://numpy.org/doc/stable/reference/generated/numpy.ndarray.html\#numpy.ndarray}{ndarray}%
\begin{footnote}[32]\sphinxAtStartFootnote
\sphinxnolinkurl{https://numpy.org/doc/stable/reference/generated/numpy.ndarray.html\#numpy.ndarray}
%
\end{footnote}}}, \sphinxparam{\DUrole{n,n}{data}\DUrole{p,p}{:}\DUrole{w,w}{  }\DUrole{n,n}{\sphinxhref{https://numpy.org/doc/stable/reference/generated/numpy.ndarray.html\#numpy.ndarray}{ndarray}%
\begin{footnote}[33]\sphinxAtStartFootnote
\sphinxnolinkurl{https://numpy.org/doc/stable/reference/generated/numpy.ndarray.html\#numpy.ndarray}
%
\end{footnote}}}, \sphinxparam{\DUrole{n,n}{uncertainty}\DUrole{p,p}{:}\DUrole{w,w}{  }\DUrole{n,n}{\sphinxhref{https://numpy.org/doc/stable/reference/generated/numpy.ndarray.html\#numpy.ndarray}{ndarray}%
\begin{footnote}[34]\sphinxAtStartFootnote
\sphinxnolinkurl{https://numpy.org/doc/stable/reference/generated/numpy.ndarray.html\#numpy.ndarray}
%
\end{footnote}}}, \sphinxparam{\DUrole{n,n}{wavelength\_unit}\DUrole{p,p}{:}\DUrole{w,w}{  }\DUrole{n,n}{\sphinxhref{https://docs.astropy.org/en/stable/api/astropy.units.Unit.html\#astropy.units.Unit}{Unit}%
\begin{footnote}[35]\sphinxAtStartFootnote
\sphinxnolinkurl{https://docs.astropy.org/en/stable/api/astropy.units.Unit.html\#astropy.units.Unit}
%
\end{footnote}}}, \sphinxparam{\DUrole{n,n}{data\_unit}\DUrole{p,p}{:}\DUrole{w,w}{  }\DUrole{n,n}{\sphinxhref{https://docs.astropy.org/en/stable/api/astropy.units.Unit.html\#astropy.units.Unit}{Unit}%
\begin{footnote}[36]\sphinxAtStartFootnote
\sphinxnolinkurl{https://docs.astropy.org/en/stable/api/astropy.units.Unit.html\#astropy.units.Unit}
%
\end{footnote}}}, \sphinxparam{\DUrole{n,n}{mask}\DUrole{p,p}{:}\DUrole{w,w}{  }\DUrole{n,n}{\sphinxhref{https://numpy.org/doc/stable/reference/generated/numpy.ndarray.html\#numpy.ndarray}{ndarray}%
\begin{footnote}[37]\sphinxAtStartFootnote
\sphinxnolinkurl{https://numpy.org/doc/stable/reference/generated/numpy.ndarray.html\#numpy.ndarray}
%
\end{footnote}}}, \sphinxparam{\DUrole{n,n}{flags}\DUrole{p,p}{:}\DUrole{w,w}{  }\DUrole{n,n}{\sphinxhref{https://numpy.org/doc/stable/reference/generated/numpy.ndarray.html\#numpy.ndarray}{ndarray}%
\begin{footnote}[38]\sphinxAtStartFootnote
\sphinxnolinkurl{https://numpy.org/doc/stable/reference/generated/numpy.ndarray.html\#numpy.ndarray}
%
\end{footnote}}}, \sphinxparam{\DUrole{n,n}{header}\DUrole{p,p}{:}\DUrole{w,w}{  }\DUrole{n,n}{dict}}}{}
\pysigstopsignatures
\sphinxAtStartPar
Bases: \sphinxcode{\sphinxupquote{object}}

\sphinxAtStartPar
Lezargus wavelength\sphinxhyphen{}aware arithmetic.

\sphinxAtStartPar
This is the class which allows for the arithmetic behind the scenes to
work with wavelength knowledge. All we do is overwrite the NDDataArray
arithmetic functions to perform wavelength checks and pass it through
without wavelength issues.
\index{wavelength (lezargus.container.parent.LezargusContainerArithmetic attribute)@\spxentry{wavelength}\spxextra{lezargus.container.parent.LezargusContainerArithmetic attribute}}

\begin{savenotes}\begin{fulllineitems}
\phantomsection\label{\detokenize{code/lezargus.container.parent:lezargus.container.parent.LezargusContainerArithmetic.wavelength}}
\pysigstartsignatures
\pysigline{\sphinxbfcode{\sphinxupquote{wavelength}}}
\pysigstopsignatures
\sphinxAtStartPar
The wavelength of the spectra. The unit of wavelength is typically
in microns; but, check the \sphinxtitleref{wavelength\_unit} value.
\begin{quote}\begin{description}
\sphinxlineitem{Type}
\sphinxAtStartPar
ndarray

\end{description}\end{quote}

\end{fulllineitems}\end{savenotes}

\index{data (lezargus.container.parent.LezargusContainerArithmetic attribute)@\spxentry{data}\spxextra{lezargus.container.parent.LezargusContainerArithmetic attribute}}

\begin{savenotes}\begin{fulllineitems}
\phantomsection\label{\detokenize{code/lezargus.container.parent:lezargus.container.parent.LezargusContainerArithmetic.data}}
\pysigstartsignatures
\pysigline{\sphinxbfcode{\sphinxupquote{data}}}
\pysigstopsignatures
\sphinxAtStartPar
The data or flux of the spectra cube. The unit of the flux is typically
in flam; but, check the \sphinxtitleref{data\_unit} value.
\begin{quote}\begin{description}
\sphinxlineitem{Type}
\sphinxAtStartPar
ndarray

\end{description}\end{quote}

\end{fulllineitems}\end{savenotes}

\index{uncertainty (lezargus.container.parent.LezargusContainerArithmetic attribute)@\spxentry{uncertainty}\spxextra{lezargus.container.parent.LezargusContainerArithmetic attribute}}

\begin{savenotes}\begin{fulllineitems}
\phantomsection\label{\detokenize{code/lezargus.container.parent:lezargus.container.parent.LezargusContainerArithmetic.uncertainty}}
\pysigstartsignatures
\pysigline{\sphinxbfcode{\sphinxupquote{uncertainty}}}
\pysigstopsignatures
\sphinxAtStartPar
The uncertainty in the data. The unit of the uncertainty
is the same as the flux value; per \sphinxtitleref{uncertainty\_unit}.
\begin{quote}\begin{description}
\sphinxlineitem{Type}
\sphinxAtStartPar
ndarray

\end{description}\end{quote}

\end{fulllineitems}\end{savenotes}

\index{wavelength\_unit (lezargus.container.parent.LezargusContainerArithmetic attribute)@\spxentry{wavelength\_unit}\spxextra{lezargus.container.parent.LezargusContainerArithmetic attribute}}

\begin{savenotes}\begin{fulllineitems}
\phantomsection\label{\detokenize{code/lezargus.container.parent:lezargus.container.parent.LezargusContainerArithmetic.wavelength_unit}}
\pysigstartsignatures
\pysigline{\sphinxbfcode{\sphinxupquote{wavelength\_unit}}}
\pysigstopsignatures
\sphinxAtStartPar
The unit of the wavelength array.
\begin{quote}\begin{description}
\sphinxlineitem{Type}
\sphinxAtStartPar
Astropy Unit

\end{description}\end{quote}

\end{fulllineitems}\end{savenotes}

\index{data\_unit (lezargus.container.parent.LezargusContainerArithmetic attribute)@\spxentry{data\_unit}\spxextra{lezargus.container.parent.LezargusContainerArithmetic attribute}}

\begin{savenotes}\begin{fulllineitems}
\phantomsection\label{\detokenize{code/lezargus.container.parent:lezargus.container.parent.LezargusContainerArithmetic.data_unit}}
\pysigstartsignatures
\pysigline{\sphinxbfcode{\sphinxupquote{data\_unit}}}
\pysigstopsignatures
\sphinxAtStartPar
The unit of the data array.
\begin{quote}\begin{description}
\sphinxlineitem{Type}
\sphinxAtStartPar
Astropy Unit

\end{description}\end{quote}

\end{fulllineitems}\end{savenotes}

\index{uncertainty\_unit (lezargus.container.parent.LezargusContainerArithmetic attribute)@\spxentry{uncertainty\_unit}\spxextra{lezargus.container.parent.LezargusContainerArithmetic attribute}}

\begin{savenotes}\begin{fulllineitems}
\phantomsection\label{\detokenize{code/lezargus.container.parent:lezargus.container.parent.LezargusContainerArithmetic.uncertainty_unit}}
\pysigstartsignatures
\pysigline{\sphinxbfcode{\sphinxupquote{uncertainty\_unit}}}
\pysigstopsignatures
\sphinxAtStartPar
The unit of the uncertainty array. This unit is the same as the data
unit.
\begin{quote}\begin{description}
\sphinxlineitem{Type}
\sphinxAtStartPar
Astropy Unit

\end{description}\end{quote}

\end{fulllineitems}\end{savenotes}

\index{mask (lezargus.container.parent.LezargusContainerArithmetic attribute)@\spxentry{mask}\spxextra{lezargus.container.parent.LezargusContainerArithmetic attribute}}

\begin{savenotes}\begin{fulllineitems}
\phantomsection\label{\detokenize{code/lezargus.container.parent:lezargus.container.parent.LezargusContainerArithmetic.mask}}
\pysigstartsignatures
\pysigline{\sphinxbfcode{\sphinxupquote{mask}}}
\pysigstopsignatures
\sphinxAtStartPar
A mask of the data, used to remove problematic areas. Where True,
the values of the data is considered masked.
\begin{quote}\begin{description}
\sphinxlineitem{Type}
\sphinxAtStartPar
ndarray

\end{description}\end{quote}

\end{fulllineitems}\end{savenotes}

\index{flags (lezargus.container.parent.LezargusContainerArithmetic attribute)@\spxentry{flags}\spxextra{lezargus.container.parent.LezargusContainerArithmetic attribute}}

\begin{savenotes}\begin{fulllineitems}
\phantomsection\label{\detokenize{code/lezargus.container.parent:lezargus.container.parent.LezargusContainerArithmetic.flags}}
\pysigstartsignatures
\pysigline{\sphinxbfcode{\sphinxupquote{flags}}}
\pysigstopsignatures
\sphinxAtStartPar
Flags of the data. These flags store metadata about the data.
\begin{quote}\begin{description}
\sphinxlineitem{Type}
\sphinxAtStartPar
ndarray

\end{description}\end{quote}

\end{fulllineitems}\end{savenotes}

\index{header (lezargus.container.parent.LezargusContainerArithmetic attribute)@\spxentry{header}\spxextra{lezargus.container.parent.LezargusContainerArithmetic attribute}}

\begin{savenotes}\begin{fulllineitems}
\phantomsection\label{\detokenize{code/lezargus.container.parent:lezargus.container.parent.LezargusContainerArithmetic.header}}
\pysigstartsignatures
\pysigline{\sphinxbfcode{\sphinxupquote{header}}}
\pysigstopsignatures
\sphinxAtStartPar
The header information, or metadata in general, about the data.
\begin{quote}\begin{description}
\sphinxlineitem{Type}
\sphinxAtStartPar
Header

\end{description}\end{quote}

\end{fulllineitems}\end{savenotes}

\index{\_\_add\_\_() (lezargus.container.parent.LezargusContainerArithmetic method)@\spxentry{\_\_add\_\_()}\spxextra{lezargus.container.parent.LezargusContainerArithmetic method}}

\begin{savenotes}\begin{fulllineitems}
\phantomsection\label{\detokenize{code/lezargus.container.parent:lezargus.container.parent.LezargusContainerArithmetic.__add__}}
\pysigstartsignatures
\pysiglinewithargsret{\sphinxbfcode{\sphinxupquote{\_\_add\_\_}}}{\sphinxparam{\DUrole{n,n}{operand}\DUrole{p,p}{:}\DUrole{w,w}{  }\DUrole{n,n}{Self}}}{{ $\rightarrow$ Self}}
\pysigstopsignatures
\sphinxAtStartPar
Perform an addition operation.
\begin{quote}\begin{description}
\sphinxlineitem{Parameters}
\sphinxAtStartPar
\sphinxstyleliteralstrong{\sphinxupquote{operand}} (\sphinxstyleliteralemphasis{\sphinxupquote{Self\sphinxhyphen{}like}}) – The container object to add to this.

\sphinxlineitem{Returns}
\sphinxAtStartPar
\sphinxstylestrong{result} – A copy of this object with the resultant calculations done.

\sphinxlineitem{Return type}
\sphinxAtStartPar
Self\sphinxhyphen{}like

\end{description}\end{quote}

\end{fulllineitems}\end{savenotes}

\index{\_\_init\_\_() (lezargus.container.parent.LezargusContainerArithmetic method)@\spxentry{\_\_init\_\_()}\spxextra{lezargus.container.parent.LezargusContainerArithmetic method}}

\begin{savenotes}\begin{fulllineitems}
\phantomsection\label{\detokenize{code/lezargus.container.parent:lezargus.container.parent.LezargusContainerArithmetic.__init__}}
\pysigstartsignatures
\pysiglinewithargsret{\sphinxbfcode{\sphinxupquote{\_\_init\_\_}}}{\sphinxparam{\DUrole{n,n}{wavelength}\DUrole{p,p}{:}\DUrole{w,w}{  }\DUrole{n,n}{\sphinxhref{https://numpy.org/doc/stable/reference/generated/numpy.ndarray.html\#numpy.ndarray}{ndarray}%
\begin{footnote}[39]\sphinxAtStartFootnote
\sphinxnolinkurl{https://numpy.org/doc/stable/reference/generated/numpy.ndarray.html\#numpy.ndarray}
%
\end{footnote}}}, \sphinxparam{\DUrole{n,n}{data}\DUrole{p,p}{:}\DUrole{w,w}{  }\DUrole{n,n}{\sphinxhref{https://numpy.org/doc/stable/reference/generated/numpy.ndarray.html\#numpy.ndarray}{ndarray}%
\begin{footnote}[40]\sphinxAtStartFootnote
\sphinxnolinkurl{https://numpy.org/doc/stable/reference/generated/numpy.ndarray.html\#numpy.ndarray}
%
\end{footnote}}}, \sphinxparam{\DUrole{n,n}{uncertainty}\DUrole{p,p}{:}\DUrole{w,w}{  }\DUrole{n,n}{\sphinxhref{https://numpy.org/doc/stable/reference/generated/numpy.ndarray.html\#numpy.ndarray}{ndarray}%
\begin{footnote}[41]\sphinxAtStartFootnote
\sphinxnolinkurl{https://numpy.org/doc/stable/reference/generated/numpy.ndarray.html\#numpy.ndarray}
%
\end{footnote}}}, \sphinxparam{\DUrole{n,n}{wavelength\_unit}\DUrole{p,p}{:}\DUrole{w,w}{  }\DUrole{n,n}{\sphinxhref{https://docs.astropy.org/en/stable/api/astropy.units.Unit.html\#astropy.units.Unit}{Unit}%
\begin{footnote}[42]\sphinxAtStartFootnote
\sphinxnolinkurl{https://docs.astropy.org/en/stable/api/astropy.units.Unit.html\#astropy.units.Unit}
%
\end{footnote}}}, \sphinxparam{\DUrole{n,n}{data\_unit}\DUrole{p,p}{:}\DUrole{w,w}{  }\DUrole{n,n}{\sphinxhref{https://docs.astropy.org/en/stable/api/astropy.units.Unit.html\#astropy.units.Unit}{Unit}%
\begin{footnote}[43]\sphinxAtStartFootnote
\sphinxnolinkurl{https://docs.astropy.org/en/stable/api/astropy.units.Unit.html\#astropy.units.Unit}
%
\end{footnote}}}, \sphinxparam{\DUrole{n,n}{mask}\DUrole{p,p}{:}\DUrole{w,w}{  }\DUrole{n,n}{\sphinxhref{https://numpy.org/doc/stable/reference/generated/numpy.ndarray.html\#numpy.ndarray}{ndarray}%
\begin{footnote}[44]\sphinxAtStartFootnote
\sphinxnolinkurl{https://numpy.org/doc/stable/reference/generated/numpy.ndarray.html\#numpy.ndarray}
%
\end{footnote}}}, \sphinxparam{\DUrole{n,n}{flags}\DUrole{p,p}{:}\DUrole{w,w}{  }\DUrole{n,n}{\sphinxhref{https://numpy.org/doc/stable/reference/generated/numpy.ndarray.html\#numpy.ndarray}{ndarray}%
\begin{footnote}[45]\sphinxAtStartFootnote
\sphinxnolinkurl{https://numpy.org/doc/stable/reference/generated/numpy.ndarray.html\#numpy.ndarray}
%
\end{footnote}}}, \sphinxparam{\DUrole{n,n}{header}\DUrole{p,p}{:}\DUrole{w,w}{  }\DUrole{n,n}{dict}}}{{ $\rightarrow$ None}}
\pysigstopsignatures
\sphinxAtStartPar
Construct a wavelength\sphinxhyphen{}aware NDDataArray for arithmetic.
\begin{quote}\begin{description}
\sphinxlineitem{Parameters}\begin{itemize}
\item {} 
\sphinxAtStartPar
\sphinxstyleliteralstrong{\sphinxupquote{wavelength}} (\sphinxstyleliteralemphasis{\sphinxupquote{ndarray}}) – The wavelength of the spectra. The unit of wavelength is typically
in microns; but, check the \sphinxtitleref{wavelength\_unit} value.

\item {} 
\sphinxAtStartPar
\sphinxstyleliteralstrong{\sphinxupquote{data}} (\sphinxstyleliteralemphasis{\sphinxupquote{ndarray}}) – The data of the spectra cube. The unit of the flux is typically
in flam; but, check the \sphinxtitleref{data\_unit} value.

\item {} 
\sphinxAtStartPar
\sphinxstyleliteralstrong{\sphinxupquote{uncertainty}} (\sphinxstyleliteralemphasis{\sphinxupquote{ndarray}}) – The uncertainty in the data of the spectra. The unit of the
uncertainty is the same as the data value; per \sphinxtitleref{uncertainty\_unit}.

\item {} 
\sphinxAtStartPar
\sphinxstyleliteralstrong{\sphinxupquote{wavelength\_unit}} (\sphinxstyleliteralemphasis{\sphinxupquote{Astropy Unit}}) – The unit of the wavelength array.

\item {} 
\sphinxAtStartPar
\sphinxstyleliteralstrong{\sphinxupquote{data\_unit}} (\sphinxstyleliteralemphasis{\sphinxupquote{Astropy Unit}}) – The unit of the data array.

\item {} 
\sphinxAtStartPar
\sphinxstyleliteralstrong{\sphinxupquote{mask}} (\sphinxstyleliteralemphasis{\sphinxupquote{ndarray}}) – A mask of the data, used to remove problematic areas. Where True,
the values of the data is considered masked.

\item {} 
\sphinxAtStartPar
\sphinxstyleliteralstrong{\sphinxupquote{flags}} (\sphinxstyleliteralemphasis{\sphinxupquote{ndarray}}) – Flags of the data. These flags store metadata about the data.

\item {} 
\sphinxAtStartPar
\sphinxstyleliteralstrong{\sphinxupquote{header}} (\sphinxstyleliteralemphasis{\sphinxupquote{Header}}\sphinxstyleliteralemphasis{\sphinxupquote{, }}\sphinxstyleliteralemphasis{\sphinxupquote{default = None}}) – A set of header data describing the data. Note that when saving,
this header is written to disk with minimal processing. We highly
suggest writing of the metadata to conform to the FITS Header
specification as much as possible.

\end{itemize}

\sphinxlineitem{Return type}
\sphinxAtStartPar
None

\end{description}\end{quote}

\end{fulllineitems}\end{savenotes}

\index{\_\_justify\_arithmetic\_operation() (lezargus.container.parent.LezargusContainerArithmetic method)@\spxentry{\_\_justify\_arithmetic\_operation()}\spxextra{lezargus.container.parent.LezargusContainerArithmetic method}}

\begin{savenotes}\begin{fulllineitems}
\phantomsection\label{\detokenize{code/lezargus.container.parent:lezargus.container.parent.LezargusContainerArithmetic.__justify_arithmetic_operation}}
\pysigstartsignatures
\pysiglinewithargsret{\sphinxbfcode{\sphinxupquote{\_\_justify\_arithmetic\_operation}}}{\sphinxparam{\DUrole{n,n}{operand}\DUrole{p,p}{:}\DUrole{w,w}{  }\DUrole{n,n}{Self\DUrole{w,w}{  }\DUrole{p,p}{|}\DUrole{w,w}{  }float}}}{{ $\rightarrow$ bool}}
\pysigstopsignatures
\sphinxAtStartPar
Justify operations between two objects is valid.

\sphinxAtStartPar
Operations done between different instances of the Lezargus data
structure need to keep in mind the wavelength dependance of the data.
We implement simple checks here to formalize if an operation between
this object, and some other operand, can be performed.
\begin{quote}\begin{description}
\sphinxlineitem{Parameters}
\sphinxAtStartPar
\sphinxstyleliteralstrong{\sphinxupquote{operand}} (\sphinxstyleliteralemphasis{\sphinxupquote{Self\sphinxhyphen{}like}}\sphinxstyleliteralemphasis{\sphinxupquote{ or }}\sphinxstyleliteralemphasis{\sphinxupquote{number}}) – The container object that we have an operation to apply with.

\sphinxlineitem{Returns}
\sphinxAtStartPar
\begin{itemize}
\item {} 
\sphinxAtStartPar
\sphinxstylestrong{justification} (\sphinxstyleemphasis{bool}) – The state of the justification test. If it is True, then the
operation can continue, otherwise, False.

\item {} 
\sphinxAtStartPar
\sphinxstyleemphasis{.. note::} – This function will also raise exceptions upon discovery of
incompatible objects. Therefore, the False return case is not
really that impactful.

\end{itemize}


\end{description}\end{quote}

\end{fulllineitems}\end{savenotes}

\index{\_\_mul\_\_() (lezargus.container.parent.LezargusContainerArithmetic method)@\spxentry{\_\_mul\_\_()}\spxextra{lezargus.container.parent.LezargusContainerArithmetic method}}

\begin{savenotes}\begin{fulllineitems}
\phantomsection\label{\detokenize{code/lezargus.container.parent:lezargus.container.parent.LezargusContainerArithmetic.__mul__}}
\pysigstartsignatures
\pysiglinewithargsret{\sphinxbfcode{\sphinxupquote{\_\_mul\_\_}}}{\sphinxparam{\DUrole{n,n}{operand}\DUrole{p,p}{:}\DUrole{w,w}{  }\DUrole{n,n}{Self}}}{{ $\rightarrow$ Self}}
\pysigstopsignatures
\sphinxAtStartPar
Perform a multiplication operation.
\begin{quote}\begin{description}
\sphinxlineitem{Parameters}
\sphinxAtStartPar
\sphinxstyleliteralstrong{\sphinxupquote{operand}} (\sphinxstyleliteralemphasis{\sphinxupquote{Self\sphinxhyphen{}like}}) – The container object to add to this.

\sphinxlineitem{Returns}
\sphinxAtStartPar
\sphinxstylestrong{result} – A copy of this object with the resultant calculations done.

\sphinxlineitem{Return type}
\sphinxAtStartPar
Self\sphinxhyphen{}like

\end{description}\end{quote}

\end{fulllineitems}\end{savenotes}

\index{\_\_pow\_\_() (lezargus.container.parent.LezargusContainerArithmetic method)@\spxentry{\_\_pow\_\_()}\spxextra{lezargus.container.parent.LezargusContainerArithmetic method}}

\begin{savenotes}\begin{fulllineitems}
\phantomsection\label{\detokenize{code/lezargus.container.parent:lezargus.container.parent.LezargusContainerArithmetic.__pow__}}
\pysigstartsignatures
\pysiglinewithargsret{\sphinxbfcode{\sphinxupquote{\_\_pow\_\_}}}{\sphinxparam{\DUrole{n,n}{operand}\DUrole{p,p}{:}\DUrole{w,w}{  }\DUrole{n,n}{Self}}}{{ $\rightarrow$ Self}}
\pysigstopsignatures
\sphinxAtStartPar
Perform a true division operation.
\begin{quote}\begin{description}
\sphinxlineitem{Parameters}
\sphinxAtStartPar
\sphinxstyleliteralstrong{\sphinxupquote{operand}} (\sphinxstyleliteralemphasis{\sphinxupquote{Self\sphinxhyphen{}like}}) – The container object to add to this.

\sphinxlineitem{Returns}
\sphinxAtStartPar
\sphinxstylestrong{result} – A copy of this object with the resultant calculations done.

\sphinxlineitem{Return type}
\sphinxAtStartPar
Self\sphinxhyphen{}like

\end{description}\end{quote}

\end{fulllineitems}\end{savenotes}

\index{\_\_sub\_\_() (lezargus.container.parent.LezargusContainerArithmetic method)@\spxentry{\_\_sub\_\_()}\spxextra{lezargus.container.parent.LezargusContainerArithmetic method}}

\begin{savenotes}\begin{fulllineitems}
\phantomsection\label{\detokenize{code/lezargus.container.parent:lezargus.container.parent.LezargusContainerArithmetic.__sub__}}
\pysigstartsignatures
\pysiglinewithargsret{\sphinxbfcode{\sphinxupquote{\_\_sub\_\_}}}{\sphinxparam{\DUrole{n,n}{operand}\DUrole{p,p}{:}\DUrole{w,w}{  }\DUrole{n,n}{Self}}}{{ $\rightarrow$ Self}}
\pysigstopsignatures
\sphinxAtStartPar
Perform a subtraction operation.
\begin{quote}\begin{description}
\sphinxlineitem{Parameters}
\sphinxAtStartPar
\sphinxstyleliteralstrong{\sphinxupquote{operand}} (\sphinxstyleliteralemphasis{\sphinxupquote{Self\sphinxhyphen{}like}}) – The container object to add to this.

\sphinxlineitem{Returns}
\sphinxAtStartPar
\sphinxstylestrong{result} – A copy of this object with the resultant calculations done.

\sphinxlineitem{Return type}
\sphinxAtStartPar
Self\sphinxhyphen{}like

\end{description}\end{quote}

\end{fulllineitems}\end{savenotes}

\index{\_\_truediv\_\_() (lezargus.container.parent.LezargusContainerArithmetic method)@\spxentry{\_\_truediv\_\_()}\spxextra{lezargus.container.parent.LezargusContainerArithmetic method}}

\begin{savenotes}\begin{fulllineitems}
\phantomsection\label{\detokenize{code/lezargus.container.parent:lezargus.container.parent.LezargusContainerArithmetic.__truediv__}}
\pysigstartsignatures
\pysiglinewithargsret{\sphinxbfcode{\sphinxupquote{\_\_truediv\_\_}}}{\sphinxparam{\DUrole{n,n}{operand}\DUrole{p,p}{:}\DUrole{w,w}{  }\DUrole{n,n}{Self}}}{{ $\rightarrow$ Self}}
\pysigstopsignatures
\sphinxAtStartPar
Perform a true division operation.
\begin{quote}\begin{description}
\sphinxlineitem{Parameters}
\sphinxAtStartPar
\sphinxstyleliteralstrong{\sphinxupquote{operand}} (\sphinxstyleliteralemphasis{\sphinxupquote{Self\sphinxhyphen{}like}}) – The container object to add to this.

\sphinxlineitem{Returns}
\sphinxAtStartPar
\sphinxstylestrong{result} – A copy of this object with the resultant calculations done.

\sphinxlineitem{Return type}
\sphinxAtStartPar
Self\sphinxhyphen{}like

\end{description}\end{quote}

\end{fulllineitems}\end{savenotes}

\index{\_read\_fits\_file() (lezargus.container.parent.LezargusContainerArithmetic class method)@\spxentry{\_read\_fits\_file()}\spxextra{lezargus.container.parent.LezargusContainerArithmetic class method}}

\begin{savenotes}\begin{fulllineitems}
\phantomsection\label{\detokenize{code/lezargus.container.parent:lezargus.container.parent.LezargusContainerArithmetic._read_fits_file}}
\pysigstartsignatures
\pysiglinewithargsret{\sphinxbfcode{\sphinxupquote{classmethod\DUrole{w,w}{  }}}\sphinxbfcode{\sphinxupquote{\_read\_fits\_file}}}{\sphinxparam{\DUrole{n,n}{filename}\DUrole{p,p}{:}\DUrole{w,w}{  }\DUrole{n,n}{str}}}{{ $\rightarrow$ Self}}
\pysigstopsignatures
\sphinxAtStartPar
Read in a FITS file into an object.

\sphinxAtStartPar
This is a wrapper around the main FITS class for uniform handling.
The respective containers should wrap around this for
container\sphinxhyphen{}specific handling and should not overwrite this function.
\begin{quote}\begin{description}
\sphinxlineitem{Parameters}
\sphinxAtStartPar
\sphinxstyleliteralstrong{\sphinxupquote{filename}} (\sphinxstyleliteralemphasis{\sphinxupquote{str}}) – The file to read in.

\sphinxlineitem{Returns}
\sphinxAtStartPar
\sphinxstylestrong{container} – The Lezargus container which was read into the file.

\sphinxlineitem{Return type}
\sphinxAtStartPar
Self\sphinxhyphen{}like

\end{description}\end{quote}

\end{fulllineitems}\end{savenotes}

\index{\_write\_fits\_file() (lezargus.container.parent.LezargusContainerArithmetic method)@\spxentry{\_write\_fits\_file()}\spxextra{lezargus.container.parent.LezargusContainerArithmetic method}}

\begin{savenotes}\begin{fulllineitems}
\phantomsection\label{\detokenize{code/lezargus.container.parent:lezargus.container.parent.LezargusContainerArithmetic._write_fits_file}}
\pysigstartsignatures
\pysiglinewithargsret{\sphinxbfcode{\sphinxupquote{\_write\_fits\_file}}}{\sphinxparam{\DUrole{n,n}{filename}\DUrole{p,p}{:}\DUrole{w,w}{  }\DUrole{n,n}{str}}, \sphinxparam{\DUrole{n,n}{overwrite}\DUrole{p,p}{:}\DUrole{w,w}{  }\DUrole{n,n}{bool}\DUrole{w,w}{  }\DUrole{o,o}{=}\DUrole{w,w}{  }\DUrole{default_value}{False}}}{{ $\rightarrow$ None}}
\pysigstopsignatures
\sphinxAtStartPar
Write a FITS object to disk..

\sphinxAtStartPar
This is a wrapper around the main FITS class for uniform handling.
The respective containers should wrap around this for
container\sphinxhyphen{}specific handling and should not overwrite this function.
\begin{quote}\begin{description}
\sphinxlineitem{Parameters}\begin{itemize}
\item {} 
\sphinxAtStartPar
\sphinxstyleliteralstrong{\sphinxupquote{filename}} (\sphinxstyleliteralemphasis{\sphinxupquote{str}}) – The file to be written out.

\item {} 
\sphinxAtStartPar
\sphinxstyleliteralstrong{\sphinxupquote{overwrite}} (\sphinxstyleliteralemphasis{\sphinxupquote{bool}}\sphinxstyleliteralemphasis{\sphinxupquote{, }}\sphinxstyleliteralemphasis{\sphinxupquote{default = False}}) – If True, overwrite any file conflicts.

\end{itemize}

\sphinxlineitem{Return type}
\sphinxAtStartPar
None

\end{description}\end{quote}

\end{fulllineitems}\end{savenotes}


\end{fulllineitems}\end{savenotes}


\sphinxstepscope


\subparagraph{lezargus.container.spectra module}
\label{\detokenize{code/lezargus.container.spectra:module-lezargus.container.spectra}}\label{\detokenize{code/lezargus.container.spectra:lezargus-container-spectra-module}}\label{\detokenize{code/lezargus.container.spectra::doc}}\index{module@\spxentry{module}!lezargus.container.spectra@\spxentry{lezargus.container.spectra}}\index{lezargus.container.spectra@\spxentry{lezargus.container.spectra}!module@\spxentry{module}}
\sphinxAtStartPar
Spectra data container.

\sphinxAtStartPar
This module and class primarily deals with spectral data.
\index{LezargusSpectra (class in lezargus.container.spectra)@\spxentry{LezargusSpectra}\spxextra{class in lezargus.container.spectra}}

\begin{savenotes}\begin{fulllineitems}
\phantomsection\label{\detokenize{code/lezargus.container.spectra:lezargus.container.spectra.LezargusSpectra}}
\pysigstartsignatures
\pysiglinewithargsret{\sphinxbfcode{\sphinxupquote{class\DUrole{w,w}{  }}}\sphinxcode{\sphinxupquote{lezargus.container.spectra.}}\sphinxbfcode{\sphinxupquote{LezargusSpectra}}}{\sphinxparam{\DUrole{n,n}{wavelength}\DUrole{p,p}{:}\DUrole{w,w}{  }\DUrole{n,n}{\sphinxhref{https://numpy.org/doc/stable/reference/generated/numpy.ndarray.html\#numpy.ndarray}{ndarray}%
\begin{footnote}[46]\sphinxAtStartFootnote
\sphinxnolinkurl{https://numpy.org/doc/stable/reference/generated/numpy.ndarray.html\#numpy.ndarray}
%
\end{footnote}}}, \sphinxparam{\DUrole{n,n}{data}\DUrole{p,p}{:}\DUrole{w,w}{  }\DUrole{n,n}{\sphinxhref{https://numpy.org/doc/stable/reference/generated/numpy.ndarray.html\#numpy.ndarray}{ndarray}%
\begin{footnote}[47]\sphinxAtStartFootnote
\sphinxnolinkurl{https://numpy.org/doc/stable/reference/generated/numpy.ndarray.html\#numpy.ndarray}
%
\end{footnote}}}, \sphinxparam{\DUrole{n,n}{uncertainty}\DUrole{p,p}{:}\DUrole{w,w}{  }\DUrole{n,n}{\sphinxhref{https://numpy.org/doc/stable/reference/generated/numpy.ndarray.html\#numpy.ndarray}{ndarray}%
\begin{footnote}[48]\sphinxAtStartFootnote
\sphinxnolinkurl{https://numpy.org/doc/stable/reference/generated/numpy.ndarray.html\#numpy.ndarray}
%
\end{footnote}\DUrole{w,w}{  }\DUrole{p,p}{|}\DUrole{w,w}{  }None}\DUrole{w,w}{  }\DUrole{o,o}{=}\DUrole{w,w}{  }\DUrole{default_value}{None}}, \sphinxparam{\DUrole{n,n}{wavelength\_unit}\DUrole{p,p}{:}\DUrole{w,w}{  }\DUrole{n,n}{str\DUrole{w,w}{  }\DUrole{p,p}{|}\DUrole{w,w}{  }\sphinxhref{https://docs.astropy.org/en/stable/api/astropy.units.Unit.html\#astropy.units.Unit}{Unit}%
\begin{footnote}[49]\sphinxAtStartFootnote
\sphinxnolinkurl{https://docs.astropy.org/en/stable/api/astropy.units.Unit.html\#astropy.units.Unit}
%
\end{footnote}\DUrole{w,w}{  }\DUrole{p,p}{|}\DUrole{w,w}{  }None}\DUrole{w,w}{  }\DUrole{o,o}{=}\DUrole{w,w}{  }\DUrole{default_value}{None}}, \sphinxparam{\DUrole{n,n}{data\_unit}\DUrole{p,p}{:}\DUrole{w,w}{  }\DUrole{n,n}{str\DUrole{w,w}{  }\DUrole{p,p}{|}\DUrole{w,w}{  }\sphinxhref{https://docs.astropy.org/en/stable/api/astropy.units.Unit.html\#astropy.units.Unit}{Unit}%
\begin{footnote}[50]\sphinxAtStartFootnote
\sphinxnolinkurl{https://docs.astropy.org/en/stable/api/astropy.units.Unit.html\#astropy.units.Unit}
%
\end{footnote}\DUrole{w,w}{  }\DUrole{p,p}{|}\DUrole{w,w}{  }None}\DUrole{w,w}{  }\DUrole{o,o}{=}\DUrole{w,w}{  }\DUrole{default_value}{None}}, \sphinxparam{\DUrole{n,n}{mask}\DUrole{p,p}{:}\DUrole{w,w}{  }\DUrole{n,n}{\sphinxhref{https://numpy.org/doc/stable/reference/generated/numpy.ndarray.html\#numpy.ndarray}{ndarray}%
\begin{footnote}[51]\sphinxAtStartFootnote
\sphinxnolinkurl{https://numpy.org/doc/stable/reference/generated/numpy.ndarray.html\#numpy.ndarray}
%
\end{footnote}\DUrole{w,w}{  }\DUrole{p,p}{|}\DUrole{w,w}{  }None}\DUrole{w,w}{  }\DUrole{o,o}{=}\DUrole{w,w}{  }\DUrole{default_value}{None}}, \sphinxparam{\DUrole{n,n}{flags}\DUrole{p,p}{:}\DUrole{w,w}{  }\DUrole{n,n}{\sphinxhref{https://numpy.org/doc/stable/reference/generated/numpy.ndarray.html\#numpy.ndarray}{ndarray}%
\begin{footnote}[52]\sphinxAtStartFootnote
\sphinxnolinkurl{https://numpy.org/doc/stable/reference/generated/numpy.ndarray.html\#numpy.ndarray}
%
\end{footnote}\DUrole{w,w}{  }\DUrole{p,p}{|}\DUrole{w,w}{  }None}\DUrole{w,w}{  }\DUrole{o,o}{=}\DUrole{w,w}{  }\DUrole{default_value}{None}}, \sphinxparam{\DUrole{n,n}{header}\DUrole{p,p}{:}\DUrole{w,w}{  }\DUrole{n,n}{\sphinxhref{https://docs.astropy.org/en/stable/io/fits/api/headers.html\#astropy.io.fits.Header}{Header}%
\begin{footnote}[53]\sphinxAtStartFootnote
\sphinxnolinkurl{https://docs.astropy.org/en/stable/io/fits/api/headers.html\#astropy.io.fits.Header}
%
\end{footnote}\DUrole{w,w}{  }\DUrole{p,p}{|}\DUrole{w,w}{  }None}\DUrole{w,w}{  }\DUrole{o,o}{=}\DUrole{w,w}{  }\DUrole{default_value}{None}}}{}
\pysigstopsignatures
\sphinxAtStartPar
Bases: {\hyperref[\detokenize{code/lezargus.container.parent:lezargus.container.parent.LezargusContainerArithmetic}]{\sphinxcrossref{\sphinxcode{\sphinxupquote{LezargusContainerArithmetic}}}}}

\sphinxAtStartPar
Container to hold spectral data and perform operations on it.
\index{wavelength (lezargus.container.spectra.LezargusSpectra attribute)@\spxentry{wavelength}\spxextra{lezargus.container.spectra.LezargusSpectra attribute}}

\begin{savenotes}\begin{fulllineitems}
\phantomsection\label{\detokenize{code/lezargus.container.spectra:lezargus.container.spectra.LezargusSpectra.wavelength}}
\pysigstartsignatures
\pysigline{\sphinxbfcode{\sphinxupquote{wavelength}}}
\pysigstopsignatures
\sphinxAtStartPar
The wavelength of the spectra. The unit of wavelength is typically
in microns; but, check the \sphinxtitleref{wavelength\_unit} value.
\begin{quote}\begin{description}
\sphinxlineitem{Type}
\sphinxAtStartPar
ndarray

\end{description}\end{quote}

\end{fulllineitems}\end{savenotes}

\index{data (lezargus.container.spectra.LezargusSpectra attribute)@\spxentry{data}\spxextra{lezargus.container.spectra.LezargusSpectra attribute}}

\begin{savenotes}\begin{fulllineitems}
\phantomsection\label{\detokenize{code/lezargus.container.spectra:lezargus.container.spectra.LezargusSpectra.data}}
\pysigstartsignatures
\pysigline{\sphinxbfcode{\sphinxupquote{data}}}
\pysigstopsignatures
\sphinxAtStartPar
The flux of the spectra. The unit of the flux is typically
in flam; but, check the \sphinxtitleref{flux\_unit} value.
\begin{quote}\begin{description}
\sphinxlineitem{Type}
\sphinxAtStartPar
ndarray

\end{description}\end{quote}

\end{fulllineitems}\end{savenotes}

\index{uncertainty (lezargus.container.spectra.LezargusSpectra attribute)@\spxentry{uncertainty}\spxextra{lezargus.container.spectra.LezargusSpectra attribute}}

\begin{savenotes}\begin{fulllineitems}
\phantomsection\label{\detokenize{code/lezargus.container.spectra:lezargus.container.spectra.LezargusSpectra.uncertainty}}
\pysigstartsignatures
\pysigline{\sphinxbfcode{\sphinxupquote{uncertainty}}}
\pysigstopsignatures
\sphinxAtStartPar
The uncertainty in the flux of the spectra. The unit of the uncertainty
is the same as the flux value; per \sphinxtitleref{uncertainty\_unit}.
\begin{quote}\begin{description}
\sphinxlineitem{Type}
\sphinxAtStartPar
ndarray

\end{description}\end{quote}

\end{fulllineitems}\end{savenotes}

\index{wavelength\_unit (lezargus.container.spectra.LezargusSpectra attribute)@\spxentry{wavelength\_unit}\spxextra{lezargus.container.spectra.LezargusSpectra attribute}}

\begin{savenotes}\begin{fulllineitems}
\phantomsection\label{\detokenize{code/lezargus.container.spectra:lezargus.container.spectra.LezargusSpectra.wavelength_unit}}
\pysigstartsignatures
\pysigline{\sphinxbfcode{\sphinxupquote{wavelength\_unit}}}
\pysigstopsignatures
\sphinxAtStartPar
The unit of the wavelength array.
\begin{quote}\begin{description}
\sphinxlineitem{Type}
\sphinxAtStartPar
Astropy Unit

\end{description}\end{quote}

\end{fulllineitems}\end{savenotes}

\index{flux\_unit (lezargus.container.spectra.LezargusSpectra attribute)@\spxentry{flux\_unit}\spxextra{lezargus.container.spectra.LezargusSpectra attribute}}

\begin{savenotes}\begin{fulllineitems}
\phantomsection\label{\detokenize{code/lezargus.container.spectra:lezargus.container.spectra.LezargusSpectra.flux_unit}}
\pysigstartsignatures
\pysigline{\sphinxbfcode{\sphinxupquote{flux\_unit}}}
\pysigstopsignatures
\sphinxAtStartPar
The unit of the flux array.
\begin{quote}\begin{description}
\sphinxlineitem{Type}
\sphinxAtStartPar
Astropy Unit

\end{description}\end{quote}

\end{fulllineitems}\end{savenotes}

\index{uncertainty\_unit (lezargus.container.spectra.LezargusSpectra attribute)@\spxentry{uncertainty\_unit}\spxextra{lezargus.container.spectra.LezargusSpectra attribute}}

\begin{savenotes}\begin{fulllineitems}
\phantomsection\label{\detokenize{code/lezargus.container.spectra:lezargus.container.spectra.LezargusSpectra.uncertainty_unit}}
\pysigstartsignatures
\pysigline{\sphinxbfcode{\sphinxupquote{uncertainty\_unit}}}
\pysigstopsignatures
\sphinxAtStartPar
The unit of the uncertainty array. This unit is the same as the flux
unit.
\begin{quote}\begin{description}
\sphinxlineitem{Type}
\sphinxAtStartPar
Astropy Unit

\end{description}\end{quote}

\end{fulllineitems}\end{savenotes}

\index{mask (lezargus.container.spectra.LezargusSpectra attribute)@\spxentry{mask}\spxextra{lezargus.container.spectra.LezargusSpectra attribute}}

\begin{savenotes}\begin{fulllineitems}
\phantomsection\label{\detokenize{code/lezargus.container.spectra:lezargus.container.spectra.LezargusSpectra.mask}}
\pysigstartsignatures
\pysigline{\sphinxbfcode{\sphinxupquote{mask}}}
\pysigstopsignatures
\sphinxAtStartPar
A mask of the flux data, used to remove problematic areas. Where True,
the values of the flux is considered mask.
\begin{quote}\begin{description}
\sphinxlineitem{Type}
\sphinxAtStartPar
ndarray

\end{description}\end{quote}

\end{fulllineitems}\end{savenotes}

\index{flags (lezargus.container.spectra.LezargusSpectra attribute)@\spxentry{flags}\spxextra{lezargus.container.spectra.LezargusSpectra attribute}}

\begin{savenotes}\begin{fulllineitems}
\phantomsection\label{\detokenize{code/lezargus.container.spectra:lezargus.container.spectra.LezargusSpectra.flags}}
\pysigstartsignatures
\pysigline{\sphinxbfcode{\sphinxupquote{flags}}}
\pysigstopsignatures
\sphinxAtStartPar
Flags of the flux data. These flags store metadata about the flux.
\begin{quote}\begin{description}
\sphinxlineitem{Type}
\sphinxAtStartPar
ndarray

\end{description}\end{quote}

\end{fulllineitems}\end{savenotes}

\index{header (lezargus.container.spectra.LezargusSpectra attribute)@\spxentry{header}\spxextra{lezargus.container.spectra.LezargusSpectra attribute}}

\begin{savenotes}\begin{fulllineitems}
\phantomsection\label{\detokenize{code/lezargus.container.spectra:lezargus.container.spectra.LezargusSpectra.header}}
\pysigstartsignatures
\pysigline{\sphinxbfcode{\sphinxupquote{header}}}
\pysigstopsignatures
\sphinxAtStartPar
The header information, or metadata in general, about the data.
\begin{quote}\begin{description}
\sphinxlineitem{Type}
\sphinxAtStartPar
Header

\end{description}\end{quote}

\end{fulllineitems}\end{savenotes}

\index{\_\_init\_\_() (lezargus.container.spectra.LezargusSpectra method)@\spxentry{\_\_init\_\_()}\spxextra{lezargus.container.spectra.LezargusSpectra method}}

\begin{savenotes}\begin{fulllineitems}
\phantomsection\label{\detokenize{code/lezargus.container.spectra:lezargus.container.spectra.LezargusSpectra.__init__}}
\pysigstartsignatures
\pysiglinewithargsret{\sphinxbfcode{\sphinxupquote{\_\_init\_\_}}}{\sphinxparam{\DUrole{n,n}{wavelength}\DUrole{p,p}{:}\DUrole{w,w}{  }\DUrole{n,n}{\sphinxhref{https://numpy.org/doc/stable/reference/generated/numpy.ndarray.html\#numpy.ndarray}{ndarray}%
\begin{footnote}[54]\sphinxAtStartFootnote
\sphinxnolinkurl{https://numpy.org/doc/stable/reference/generated/numpy.ndarray.html\#numpy.ndarray}
%
\end{footnote}}}, \sphinxparam{\DUrole{n,n}{data}\DUrole{p,p}{:}\DUrole{w,w}{  }\DUrole{n,n}{\sphinxhref{https://numpy.org/doc/stable/reference/generated/numpy.ndarray.html\#numpy.ndarray}{ndarray}%
\begin{footnote}[55]\sphinxAtStartFootnote
\sphinxnolinkurl{https://numpy.org/doc/stable/reference/generated/numpy.ndarray.html\#numpy.ndarray}
%
\end{footnote}}}, \sphinxparam{\DUrole{n,n}{uncertainty}\DUrole{p,p}{:}\DUrole{w,w}{  }\DUrole{n,n}{\sphinxhref{https://numpy.org/doc/stable/reference/generated/numpy.ndarray.html\#numpy.ndarray}{ndarray}%
\begin{footnote}[56]\sphinxAtStartFootnote
\sphinxnolinkurl{https://numpy.org/doc/stable/reference/generated/numpy.ndarray.html\#numpy.ndarray}
%
\end{footnote}\DUrole{w,w}{  }\DUrole{p,p}{|}\DUrole{w,w}{  }None}\DUrole{w,w}{  }\DUrole{o,o}{=}\DUrole{w,w}{  }\DUrole{default_value}{None}}, \sphinxparam{\DUrole{n,n}{wavelength\_unit}\DUrole{p,p}{:}\DUrole{w,w}{  }\DUrole{n,n}{str\DUrole{w,w}{  }\DUrole{p,p}{|}\DUrole{w,w}{  }\sphinxhref{https://docs.astropy.org/en/stable/api/astropy.units.Unit.html\#astropy.units.Unit}{Unit}%
\begin{footnote}[57]\sphinxAtStartFootnote
\sphinxnolinkurl{https://docs.astropy.org/en/stable/api/astropy.units.Unit.html\#astropy.units.Unit}
%
\end{footnote}\DUrole{w,w}{  }\DUrole{p,p}{|}\DUrole{w,w}{  }None}\DUrole{w,w}{  }\DUrole{o,o}{=}\DUrole{w,w}{  }\DUrole{default_value}{None}}, \sphinxparam{\DUrole{n,n}{data\_unit}\DUrole{p,p}{:}\DUrole{w,w}{  }\DUrole{n,n}{str\DUrole{w,w}{  }\DUrole{p,p}{|}\DUrole{w,w}{  }\sphinxhref{https://docs.astropy.org/en/stable/api/astropy.units.Unit.html\#astropy.units.Unit}{Unit}%
\begin{footnote}[58]\sphinxAtStartFootnote
\sphinxnolinkurl{https://docs.astropy.org/en/stable/api/astropy.units.Unit.html\#astropy.units.Unit}
%
\end{footnote}\DUrole{w,w}{  }\DUrole{p,p}{|}\DUrole{w,w}{  }None}\DUrole{w,w}{  }\DUrole{o,o}{=}\DUrole{w,w}{  }\DUrole{default_value}{None}}, \sphinxparam{\DUrole{n,n}{mask}\DUrole{p,p}{:}\DUrole{w,w}{  }\DUrole{n,n}{\sphinxhref{https://numpy.org/doc/stable/reference/generated/numpy.ndarray.html\#numpy.ndarray}{ndarray}%
\begin{footnote}[59]\sphinxAtStartFootnote
\sphinxnolinkurl{https://numpy.org/doc/stable/reference/generated/numpy.ndarray.html\#numpy.ndarray}
%
\end{footnote}\DUrole{w,w}{  }\DUrole{p,p}{|}\DUrole{w,w}{  }None}\DUrole{w,w}{  }\DUrole{o,o}{=}\DUrole{w,w}{  }\DUrole{default_value}{None}}, \sphinxparam{\DUrole{n,n}{flags}\DUrole{p,p}{:}\DUrole{w,w}{  }\DUrole{n,n}{\sphinxhref{https://numpy.org/doc/stable/reference/generated/numpy.ndarray.html\#numpy.ndarray}{ndarray}%
\begin{footnote}[60]\sphinxAtStartFootnote
\sphinxnolinkurl{https://numpy.org/doc/stable/reference/generated/numpy.ndarray.html\#numpy.ndarray}
%
\end{footnote}\DUrole{w,w}{  }\DUrole{p,p}{|}\DUrole{w,w}{  }None}\DUrole{w,w}{  }\DUrole{o,o}{=}\DUrole{w,w}{  }\DUrole{default_value}{None}}, \sphinxparam{\DUrole{n,n}{header}\DUrole{p,p}{:}\DUrole{w,w}{  }\DUrole{n,n}{\sphinxhref{https://docs.astropy.org/en/stable/io/fits/api/headers.html\#astropy.io.fits.Header}{Header}%
\begin{footnote}[61]\sphinxAtStartFootnote
\sphinxnolinkurl{https://docs.astropy.org/en/stable/io/fits/api/headers.html\#astropy.io.fits.Header}
%
\end{footnote}\DUrole{w,w}{  }\DUrole{p,p}{|}\DUrole{w,w}{  }None}\DUrole{w,w}{  }\DUrole{o,o}{=}\DUrole{w,w}{  }\DUrole{default_value}{None}}}{{ $\rightarrow$ None}}
\pysigstopsignatures
\sphinxAtStartPar
Instantiate the spectra class.
\begin{quote}\begin{description}
\sphinxlineitem{Parameters}\begin{itemize}
\item {} 
\sphinxAtStartPar
\sphinxstyleliteralstrong{\sphinxupquote{wavelength}} (\sphinxstyleliteralemphasis{\sphinxupquote{ndarray}}) – The wavelength of the spectra.

\item {} 
\sphinxAtStartPar
\sphinxstyleliteralstrong{\sphinxupquote{data}} (\sphinxstyleliteralemphasis{\sphinxupquote{ndarray}}) – The flux of the spectra.

\item {} 
\sphinxAtStartPar
\sphinxstyleliteralstrong{\sphinxupquote{uncertainty}} (\sphinxstyleliteralemphasis{\sphinxupquote{ndarray}}\sphinxstyleliteralemphasis{\sphinxupquote{, }}\sphinxstyleliteralemphasis{\sphinxupquote{default = None}}) – The uncertainty of the spectra. By default, it is None and the
uncertainty value is 0.

\item {} 
\sphinxAtStartPar
\sphinxstyleliteralstrong{\sphinxupquote{wavelength\_unit}} (\sphinxstyleliteralemphasis{\sphinxupquote{Astropy\sphinxhyphen{}Unit like}}\sphinxstyleliteralemphasis{\sphinxupquote{, }}\sphinxstyleliteralemphasis{\sphinxupquote{default = None}}) – The wavelength unit of the spectra. It must be interpretable by
the Astropy Units package. If None, the the unit is dimensionless.

\item {} 
\sphinxAtStartPar
\sphinxstyleliteralstrong{\sphinxupquote{data\_unit}} (\sphinxstyleliteralemphasis{\sphinxupquote{Astropy\sphinxhyphen{}Unit like}}\sphinxstyleliteralemphasis{\sphinxupquote{, }}\sphinxstyleliteralemphasis{\sphinxupquote{default = None}}) – The data unit of the spectra. It must be interpretable by
the Astropy Units package. If None, the the unit is dimensionless.

\item {} 
\sphinxAtStartPar
\sphinxstyleliteralstrong{\sphinxupquote{mask}} (\sphinxstyleliteralemphasis{\sphinxupquote{ndarray}}\sphinxstyleliteralemphasis{\sphinxupquote{, }}\sphinxstyleliteralemphasis{\sphinxupquote{default = None}}) – A mask which should be applied to the spectra, if needed.

\item {} 
\sphinxAtStartPar
\sphinxstyleliteralstrong{\sphinxupquote{flags}} (\sphinxstyleliteralemphasis{\sphinxupquote{ndarray}}\sphinxstyleliteralemphasis{\sphinxupquote{, }}\sphinxstyleliteralemphasis{\sphinxupquote{default = None}}) – A set of flags which describe specific points of data in the
spectra.

\item {} 
\sphinxAtStartPar
\sphinxstyleliteralstrong{\sphinxupquote{header}} (\sphinxstyleliteralemphasis{\sphinxupquote{Header}}\sphinxstyleliteralemphasis{\sphinxupquote{, }}\sphinxstyleliteralemphasis{\sphinxupquote{default = None}}) – A set of header data describing the data. Note that when saving,
this header is written to disk with minimal processing. We highly
suggest writing of the metadata to conform to the FITS Header
specification as much as possible.

\end{itemize}

\end{description}\end{quote}

\end{fulllineitems}\end{savenotes}

\index{read\_fits\_file() (lezargus.container.spectra.LezargusSpectra class method)@\spxentry{read\_fits\_file()}\spxextra{lezargus.container.spectra.LezargusSpectra class method}}

\begin{savenotes}\begin{fulllineitems}
\phantomsection\label{\detokenize{code/lezargus.container.spectra:lezargus.container.spectra.LezargusSpectra.read_fits_file}}
\pysigstartsignatures
\pysiglinewithargsret{\sphinxbfcode{\sphinxupquote{classmethod\DUrole{w,w}{  }}}\sphinxbfcode{\sphinxupquote{read\_fits\_file}}}{\sphinxparam{\DUrole{n,n}{filename}\DUrole{p,p}{:}\DUrole{w,w}{  }\DUrole{n,n}{str}}}{{ $\rightarrow$ Self}}
\pysigstopsignatures
\sphinxAtStartPar
Read a Lezargus spectra FITS file.

\sphinxAtStartPar
We load a Lezargus FITS file from disk. Note that this should only
be used for 1\sphinxhyphen{}D spectra files.
\begin{quote}\begin{description}
\sphinxlineitem{Parameters}
\sphinxAtStartPar
\sphinxstyleliteralstrong{\sphinxupquote{filename}} (\sphinxstyleliteralemphasis{\sphinxupquote{str}}) – The filename to load.

\sphinxlineitem{Returns}
\sphinxAtStartPar
\sphinxstylestrong{spectra} – The LezargusSpectra class instance.

\sphinxlineitem{Return type}
\sphinxAtStartPar
Self\sphinxhyphen{}like

\end{description}\end{quote}

\end{fulllineitems}\end{savenotes}

\index{write\_fits\_file() (lezargus.container.spectra.LezargusSpectra method)@\spxentry{write\_fits\_file()}\spxextra{lezargus.container.spectra.LezargusSpectra method}}

\begin{savenotes}\begin{fulllineitems}
\phantomsection\label{\detokenize{code/lezargus.container.spectra:lezargus.container.spectra.LezargusSpectra.write_fits_file}}
\pysigstartsignatures
\pysiglinewithargsret{\sphinxbfcode{\sphinxupquote{write\_fits\_file}}}{\sphinxparam{\DUrole{n,n}{filename}\DUrole{p,p}{:}\DUrole{w,w}{  }\DUrole{n,n}{str}}, \sphinxparam{\DUrole{n,n}{overwrite}\DUrole{p,p}{:}\DUrole{w,w}{  }\DUrole{n,n}{bool}\DUrole{w,w}{  }\DUrole{o,o}{=}\DUrole{w,w}{  }\DUrole{default_value}{False}}}{{ $\rightarrow$ Self}}
\pysigstopsignatures
\sphinxAtStartPar
Write a Lezargus spectra FITS file.

\sphinxAtStartPar
We write a Lezargus FITS file to disk.
\begin{quote}\begin{description}
\sphinxlineitem{Parameters}\begin{itemize}
\item {} 
\sphinxAtStartPar
\sphinxstyleliteralstrong{\sphinxupquote{filename}} (\sphinxstyleliteralemphasis{\sphinxupquote{str}}) – The filename to write to.

\item {} 
\sphinxAtStartPar
\sphinxstyleliteralstrong{\sphinxupquote{overwrite}} (\sphinxstyleliteralemphasis{\sphinxupquote{bool}}\sphinxstyleliteralemphasis{\sphinxupquote{, }}\sphinxstyleliteralemphasis{\sphinxupquote{default = False}}) – If True, overwrite file conflicts.

\end{itemize}

\sphinxlineitem{Return type}
\sphinxAtStartPar
None

\end{description}\end{quote}

\end{fulllineitems}\end{savenotes}


\end{fulllineitems}\end{savenotes}



\subparagraph{Module contents}
\label{\detokenize{code/lezargus.container:module-lezargus.container}}\label{\detokenize{code/lezargus.container:module-contents}}\index{module@\spxentry{module}!lezargus.container@\spxentry{lezargus.container}}\index{lezargus.container@\spxentry{lezargus.container}!module@\spxentry{module}}
\sphinxAtStartPar
Containers for data.

\sphinxAtStartPar
This module contains the containers for spectral data. We have 4 main classes,
broken into different files for ease. There is a parent class which we use
to define connivent arithmetic.

\sphinxstepscope


\paragraph{lezargus.data package}
\label{\detokenize{code/lezargus.data:lezargus-data-package}}\label{\detokenize{code/lezargus.data::doc}}

\subparagraph{Module contents}
\label{\detokenize{code/lezargus.data:module-lezargus.data}}\label{\detokenize{code/lezargus.data:module-contents}}\index{module@\spxentry{module}!lezargus.data@\spxentry{lezargus.data}}\index{lezargus.data@\spxentry{lezargus.data}!module@\spxentry{module}}
\sphinxAtStartPar
Supplemental data needed for Lezargus is stored here.

\sphinxAtStartPar
This directory comprises of all types of data which is required for Lezargus.
No code should be placed in directory. To read and write the data from this
directory, see lezargus/library/data.py. We have this information in a
\_\_init\_\_.py file just so it is conveniently traced by the documentation build
script.

\sphinxAtStartPar
The following conventions are used for the data files:
\begin{itemize}
\item {} \begin{description}
\sphinxlineitem{CSV files}{[}We use pipe\sphinxhyphen{}delimitated files. Pipes form natural column line for{]}
\sphinxAtStartPar
quick reading. Comments are delimitated by \# and serve to be a
header for documentation.

\end{description}

\item {} \begin{description}
\sphinxlineitem{FITS files}{[}FITS files listed are written by and loaded by the Lezargus{]}
\sphinxAtStartPar
data containers. The FITS files within this directory should
only be spectra, image, or cube data. Any other data is likely
better in a different file format.

\end{description}

\end{itemize}

\sphinxstepscope


\paragraph{lezargus.library package}
\label{\detokenize{code/lezargus.library:lezargus-library-package}}\label{\detokenize{code/lezargus.library::doc}}

\subparagraph{Submodules}
\label{\detokenize{code/lezargus.library:submodules}}
\sphinxstepscope


\subparagraph{lezargus.library.array module}
\label{\detokenize{code/lezargus.library.array:module-lezargus.library.array}}\label{\detokenize{code/lezargus.library.array:lezargus-library-array-module}}\label{\detokenize{code/lezargus.library.array::doc}}\index{module@\spxentry{module}!lezargus.library.array@\spxentry{lezargus.library.array}}\index{lezargus.library.array@\spxentry{lezargus.library.array}!module@\spxentry{module}}
\sphinxAtStartPar
Collection of array or image manipulation functions.

\sphinxAtStartPar
If there are any functions which are done on arrays (or anything that is
just an array under the hood), we usually group it here. Moreover, functions
which would otherwise operate on images are also placed here. As images are
just arrays under the hood (and to avoid conflict with
/lezargus/container/image.py), image manipulation functions are kept here too.

\sphinxAtStartPar
Note that all of these functions follow the axes convention of indexing being
(x, y, lambda). If a cube is not of this shape, then it will likely return
erroneous results, but, the functions themselves cannot detect this.
\index{bin\_cube\_array\_spatially() (in module lezargus.library.array)@\spxentry{bin\_cube\_array\_spatially()}\spxextra{in module lezargus.library.array}}

\begin{savenotes}\begin{fulllineitems}
\phantomsection\label{\detokenize{code/lezargus.library.array:lezargus.library.array.bin_cube_array_spatially}}
\pysigstartsignatures
\pysiglinewithargsret{\sphinxcode{\sphinxupquote{lezargus.library.array.}}\sphinxbfcode{\sphinxupquote{bin\_cube\_array\_spatially}}}{\sphinxparam{\DUrole{n,n}{cube}\DUrole{p,p}{:}\DUrole{w,w}{  }\DUrole{n,n}{\sphinxhref{https://numpy.org/doc/stable/reference/generated/numpy.ndarray.html\#numpy.ndarray}{ndarray}%
\begin{footnote}[62]\sphinxAtStartFootnote
\sphinxnolinkurl{https://numpy.org/doc/stable/reference/generated/numpy.ndarray.html\#numpy.ndarray}
%
\end{footnote}}}, \sphinxparam{\DUrole{n,n}{x\_bin}\DUrole{p,p}{:}\DUrole{w,w}{  }\DUrole{n,n}{int}}, \sphinxparam{\DUrole{n,n}{y\_bin}\DUrole{p,p}{:}\DUrole{w,w}{  }\DUrole{n,n}{int}}, \sphinxparam{\DUrole{n,n}{mode}\DUrole{p,p}{:}\DUrole{w,w}{  }\DUrole{n,n}{str}\DUrole{w,w}{  }\DUrole{o,o}{=}\DUrole{w,w}{  }\DUrole{default_value}{'add'}}}{{ $\rightarrow$ \sphinxhref{https://numpy.org/doc/stable/reference/generated/numpy.ndarray.html\#numpy.ndarray}{ndarray}%
\begin{footnote}[63]\sphinxAtStartFootnote
\sphinxnolinkurl{https://numpy.org/doc/stable/reference/generated/numpy.ndarray.html\#numpy.ndarray}
%
\end{footnote}}}
\pysigstopsignatures
\sphinxAtStartPar
Bin a cube spatially into super pixels.

\sphinxAtStartPar
We only bin the cube in the spatial directions, the spectral direction is
not touched.
\begin{quote}\begin{description}
\sphinxlineitem{Parameters}\begin{itemize}
\item {} 
\sphinxAtStartPar
\sphinxstyleliteralstrong{\sphinxupquote{cube}} (\sphinxstyleliteralemphasis{\sphinxupquote{ndarray}}) – The data cube to binned.

\item {} 
\sphinxAtStartPar
\sphinxstyleliteralstrong{\sphinxupquote{x\_bin}} (\sphinxstyleliteralemphasis{\sphinxupquote{int}}) – The number of pixels in the x\sphinxhyphen{}direction to bin over per super pixel.

\item {} 
\sphinxAtStartPar
\sphinxstyleliteralstrong{\sphinxupquote{y\_bin}} (\sphinxstyleliteralemphasis{\sphinxupquote{int}}) – The number of pixels in the y\sphinxhyphen{}direction to bin over per super pixel.

\item {} 
\sphinxAtStartPar
\sphinxstyleliteralstrong{\sphinxupquote{mode}} (\sphinxstyleliteralemphasis{\sphinxupquote{string}}\sphinxstyleliteralemphasis{\sphinxupquote{, }}\sphinxstyleliteralemphasis{\sphinxupquote{default = "add"}}) – 
\sphinxAtStartPar
The mode to combine the data.
\begin{itemize}
\item {} 
\sphinxAtStartPar
\sphinxtitleref{add} : Add the pixels together.

\item {} 
\sphinxAtStartPar
\sphinxtitleref{mean} : Use the mean of the pixels.

\end{itemize}


\end{itemize}

\sphinxlineitem{Returns}
\sphinxAtStartPar
\sphinxstylestrong{binned\_image} – The data cube after binning.

\sphinxlineitem{Return type}
\sphinxAtStartPar
ndarray

\end{description}\end{quote}

\end{fulllineitems}\end{savenotes}

\index{bin\_image\_array() (in module lezargus.library.array)@\spxentry{bin\_image\_array()}\spxextra{in module lezargus.library.array}}

\begin{savenotes}\begin{fulllineitems}
\phantomsection\label{\detokenize{code/lezargus.library.array:lezargus.library.array.bin_image_array}}
\pysigstartsignatures
\pysiglinewithargsret{\sphinxcode{\sphinxupquote{lezargus.library.array.}}\sphinxbfcode{\sphinxupquote{bin\_image\_array}}}{\sphinxparam{\DUrole{n,n}{image}\DUrole{p,p}{:}\DUrole{w,w}{  }\DUrole{n,n}{\sphinxhref{https://numpy.org/doc/stable/reference/generated/numpy.ndarray.html\#numpy.ndarray}{ndarray}%
\begin{footnote}[64]\sphinxAtStartFootnote
\sphinxnolinkurl{https://numpy.org/doc/stable/reference/generated/numpy.ndarray.html\#numpy.ndarray}
%
\end{footnote}}}, \sphinxparam{\DUrole{n,n}{x\_bin}\DUrole{p,p}{:}\DUrole{w,w}{  }\DUrole{n,n}{int}}, \sphinxparam{\DUrole{n,n}{y\_bin}\DUrole{p,p}{:}\DUrole{w,w}{  }\DUrole{n,n}{int}}, \sphinxparam{\DUrole{n,n}{mode}\DUrole{p,p}{:}\DUrole{w,w}{  }\DUrole{n,n}{str}\DUrole{w,w}{  }\DUrole{o,o}{=}\DUrole{w,w}{  }\DUrole{default_value}{'add'}}}{{ $\rightarrow$ \sphinxhref{https://numpy.org/doc/stable/reference/generated/numpy.ndarray.html\#numpy.ndarray}{ndarray}%
\begin{footnote}[65]\sphinxAtStartFootnote
\sphinxnolinkurl{https://numpy.org/doc/stable/reference/generated/numpy.ndarray.html\#numpy.ndarray}
%
\end{footnote}}}
\pysigstopsignatures
\sphinxAtStartPar
Bin an image by using integer super pixels.

\sphinxAtStartPar
A lot of inspiration for this function is from here:
\sphinxurl{https://scipython.com/blog/binning-a-2d-array-in-numpy/}
\begin{quote}\begin{description}
\sphinxlineitem{Parameters}\begin{itemize}
\item {} 
\sphinxAtStartPar
\sphinxstyleliteralstrong{\sphinxupquote{image}} (\sphinxstyleliteralemphasis{\sphinxupquote{ndarray}}) – The input image/array to binned.

\item {} 
\sphinxAtStartPar
\sphinxstyleliteralstrong{\sphinxupquote{x\_bin}} (\sphinxstyleliteralemphasis{\sphinxupquote{int}}) – The number of pixels in the x\sphinxhyphen{}direction to bin over per super pixel.

\item {} 
\sphinxAtStartPar
\sphinxstyleliteralstrong{\sphinxupquote{y\_bin}} (\sphinxstyleliteralemphasis{\sphinxupquote{int}}) – The number of pixels in the y\sphinxhyphen{}direction to bin over per super pixel.

\item {} 
\sphinxAtStartPar
\sphinxstyleliteralstrong{\sphinxupquote{mode}} (\sphinxstyleliteralemphasis{\sphinxupquote{string}}\sphinxstyleliteralemphasis{\sphinxupquote{, }}\sphinxstyleliteralemphasis{\sphinxupquote{default = "add"}}) – 
\sphinxAtStartPar
The mode to combine the data.
\begin{itemize}
\item {} 
\sphinxAtStartPar
\sphinxtitleref{add} : Add the pixels together.

\item {} 
\sphinxAtStartPar
\sphinxtitleref{mean} : Use the mean of the pixels.

\end{itemize}


\end{itemize}

\sphinxlineitem{Returns}
\sphinxAtStartPar
\sphinxstylestrong{binned\_image} – The image/array after binning.

\sphinxlineitem{Return type}
\sphinxAtStartPar
ndarray

\end{description}\end{quote}

\end{fulllineitems}\end{savenotes}

\index{convolve\_cube\_by\_image\_array() (in module lezargus.library.array)@\spxentry{convolve\_cube\_by\_image\_array()}\spxextra{in module lezargus.library.array}}

\begin{savenotes}\begin{fulllineitems}
\phantomsection\label{\detokenize{code/lezargus.library.array:lezargus.library.array.convolve_cube_by_image_array}}
\pysigstartsignatures
\pysiglinewithargsret{\sphinxcode{\sphinxupquote{lezargus.library.array.}}\sphinxbfcode{\sphinxupquote{convolve\_cube\_by\_image\_array}}}{\sphinxparam{\DUrole{n,n}{cube}\DUrole{p,p}{:}\DUrole{w,w}{  }\DUrole{n,n}{\sphinxhref{https://numpy.org/doc/stable/reference/generated/numpy.ndarray.html\#numpy.ndarray}{ndarray}%
\begin{footnote}[66]\sphinxAtStartFootnote
\sphinxnolinkurl{https://numpy.org/doc/stable/reference/generated/numpy.ndarray.html\#numpy.ndarray}
%
\end{footnote}}}, \sphinxparam{\DUrole{n,n}{kernel}\DUrole{p,p}{:}\DUrole{w,w}{  }\DUrole{n,n}{\sphinxhref{https://numpy.org/doc/stable/reference/generated/numpy.ndarray.html\#numpy.ndarray}{ndarray}%
\begin{footnote}[67]\sphinxAtStartFootnote
\sphinxnolinkurl{https://numpy.org/doc/stable/reference/generated/numpy.ndarray.html\#numpy.ndarray}
%
\end{footnote}}}}{{ $\rightarrow$ \sphinxhref{https://numpy.org/doc/stable/reference/generated/numpy.ndarray.html\#numpy.ndarray}{ndarray}%
\begin{footnote}[68]\sphinxAtStartFootnote
\sphinxnolinkurl{https://numpy.org/doc/stable/reference/generated/numpy.ndarray.html\#numpy.ndarray}
%
\end{footnote}}}
\pysigstopsignatures
\sphinxAtStartPar
Convolve the image slices of a 3D cube with a 2D image.

\sphinxAtStartPar
We loop over and convolve image slices of the cube with the provided
kernel; we do not try and do an entire 3D convolution of the cube due to
memory limitations.
\begin{quote}\begin{description}
\sphinxlineitem{Parameters}\begin{itemize}
\item {} 
\sphinxAtStartPar
\sphinxstyleliteralstrong{\sphinxupquote{cube}} (\sphinxstyleliteralemphasis{\sphinxupquote{ndarray}}) – The data cube from which we will convolve.

\item {} 
\sphinxAtStartPar
\sphinxstyleliteralstrong{\sphinxupquote{kernel}} (\sphinxstyleliteralemphasis{\sphinxupquote{ndarray}}) – The image kernel we are using to convolve.

\end{itemize}

\sphinxlineitem{Returns}
\sphinxAtStartPar
\sphinxstylestrong{convolved\_cube} – The cube, with the image slices convolved by the provided kernel.

\sphinxlineitem{Return type}
\sphinxAtStartPar
ndarray

\end{description}\end{quote}

\end{fulllineitems}\end{savenotes}

\index{rotate\_image\_array() (in module lezargus.library.array)@\spxentry{rotate\_image\_array()}\spxextra{in module lezargus.library.array}}

\begin{savenotes}\begin{fulllineitems}
\phantomsection\label{\detokenize{code/lezargus.library.array:lezargus.library.array.rotate_image_array}}
\pysigstartsignatures
\pysiglinewithargsret{\sphinxcode{\sphinxupquote{lezargus.library.array.}}\sphinxbfcode{\sphinxupquote{rotate\_image\_array}}}{\sphinxparam{\DUrole{n,n}{input\_array}\DUrole{p,p}{:}\DUrole{w,w}{  }\DUrole{n,n}{\sphinxhref{https://numpy.org/doc/stable/reference/generated/numpy.ndarray.html\#numpy.ndarray}{ndarray}%
\begin{footnote}[69]\sphinxAtStartFootnote
\sphinxnolinkurl{https://numpy.org/doc/stable/reference/generated/numpy.ndarray.html\#numpy.ndarray}
%
\end{footnote}}}, \sphinxparam{\DUrole{n,n}{rotation}\DUrole{p,p}{:}\DUrole{w,w}{  }\DUrole{n,n}{float}}}{{ $\rightarrow$ \sphinxhref{https://numpy.org/doc/stable/reference/generated/numpy.ndarray.html\#numpy.ndarray}{ndarray}%
\begin{footnote}[70]\sphinxAtStartFootnote
\sphinxnolinkurl{https://numpy.org/doc/stable/reference/generated/numpy.ndarray.html\#numpy.ndarray}
%
\end{footnote}}}
\pysigstopsignatures
\sphinxAtStartPar
Rotate a 2D image array array.

\sphinxAtStartPar
This function is a connivent wrapper around scipy’s function. The array is
padded with NaNs so any data outside the original array after rotation
is null.
\begin{quote}\begin{description}
\sphinxlineitem{Parameters}\begin{itemize}
\item {} 
\sphinxAtStartPar
\sphinxstyleliteralstrong{\sphinxupquote{input\_array}} (\sphinxstyleliteralemphasis{\sphinxupquote{ndarray}}) – The input array to be rotated.

\item {} 
\sphinxAtStartPar
\sphinxstyleliteralstrong{\sphinxupquote{rotation}} (\sphinxstyleliteralemphasis{\sphinxupquote{float}}) – The rotation angle, in radians.

\end{itemize}

\sphinxlineitem{Returns}
\sphinxAtStartPar
\sphinxstylestrong{rotated\_array} – The rotated array/image.

\sphinxlineitem{Return type}
\sphinxAtStartPar
ndarray

\end{description}\end{quote}

\end{fulllineitems}\end{savenotes}

\index{translate\_image\_array() (in module lezargus.library.array)@\spxentry{translate\_image\_array()}\spxextra{in module lezargus.library.array}}

\begin{savenotes}\begin{fulllineitems}
\phantomsection\label{\detokenize{code/lezargus.library.array:lezargus.library.array.translate_image_array}}
\pysigstartsignatures
\pysiglinewithargsret{\sphinxcode{\sphinxupquote{lezargus.library.array.}}\sphinxbfcode{\sphinxupquote{translate\_image\_array}}}{\sphinxparam{\DUrole{n,n}{input\_array}\DUrole{p,p}{:}\DUrole{w,w}{  }\DUrole{n,n}{\sphinxhref{https://numpy.org/doc/stable/reference/generated/numpy.ndarray.html\#numpy.ndarray}{ndarray}%
\begin{footnote}[71]\sphinxAtStartFootnote
\sphinxnolinkurl{https://numpy.org/doc/stable/reference/generated/numpy.ndarray.html\#numpy.ndarray}
%
\end{footnote}}}, \sphinxparam{\DUrole{n,n}{x\_shift}\DUrole{p,p}{:}\DUrole{w,w}{  }\DUrole{n,n}{float}}, \sphinxparam{\DUrole{n,n}{y\_shift}\DUrole{p,p}{:}\DUrole{w,w}{  }\DUrole{n,n}{float}}}{{ $\rightarrow$ \sphinxhref{https://numpy.org/doc/stable/reference/generated/numpy.ndarray.html\#numpy.ndarray}{ndarray}%
\begin{footnote}[72]\sphinxAtStartFootnote
\sphinxnolinkurl{https://numpy.org/doc/stable/reference/generated/numpy.ndarray.html\#numpy.ndarray}
%
\end{footnote}}}
\pysigstopsignatures
\sphinxAtStartPar
Translate a 2D image array array.

\sphinxAtStartPar
This function is a convient wrapper around scipy’s function. The array is
padded with NaNs so any data outside the original array after translation
is null.
\begin{quote}\begin{description}
\sphinxlineitem{Parameters}\begin{itemize}
\item {} 
\sphinxAtStartPar
\sphinxstyleliteralstrong{\sphinxupquote{input\_array}} (\sphinxstyleliteralemphasis{\sphinxupquote{ndarray}}) – The input array to be translated.

\item {} 
\sphinxAtStartPar
\sphinxstyleliteralstrong{\sphinxupquote{x\_shift}} (\sphinxstyleliteralemphasis{\sphinxupquote{float}}) – The number of pixels that the array is shifted in the x\sphinxhyphen{}axis.

\item {} 
\sphinxAtStartPar
\sphinxstyleliteralstrong{\sphinxupquote{y\_shift}} (\sphinxstyleliteralemphasis{\sphinxupquote{float}}) – The number of pixels that the array is shifted in the y\sphinxhyphen{}axis.

\end{itemize}

\sphinxlineitem{Returns}
\sphinxAtStartPar
\sphinxstylestrong{shifted\_array} – The shifted array/image.

\sphinxlineitem{Return type}
\sphinxAtStartPar
ndarray

\end{description}\end{quote}

\end{fulllineitems}\end{savenotes}


\sphinxstepscope


\subparagraph{lezargus.library.atmosphere module}
\label{\detokenize{code/lezargus.library.atmosphere:module-lezargus.library.atmosphere}}\label{\detokenize{code/lezargus.library.atmosphere:lezargus-library-atmosphere-module}}\label{\detokenize{code/lezargus.library.atmosphere::doc}}\index{module@\spxentry{module}!lezargus.library.atmosphere@\spxentry{lezargus.library.atmosphere}}\index{lezargus.library.atmosphere@\spxentry{lezargus.library.atmosphere}!module@\spxentry{module}}
\sphinxAtStartPar
Atmospheric functions and other operations.

\sphinxAtStartPar
This file keeps track of all of the functions and computations which deal
with the atmosphere.
\index{absolute\_atmospheric\_refraction\_function() (in module lezargus.library.atmosphere)@\spxentry{absolute\_atmospheric\_refraction\_function()}\spxextra{in module lezargus.library.atmosphere}}

\begin{savenotes}\begin{fulllineitems}
\phantomsection\label{\detokenize{code/lezargus.library.atmosphere:lezargus.library.atmosphere.absolute_atmospheric_refraction_function}}
\pysigstartsignatures
\pysiglinewithargsret{\sphinxcode{\sphinxupquote{lezargus.library.atmosphere.}}\sphinxbfcode{\sphinxupquote{absolute\_atmospheric\_refraction\_function}}}{\sphinxparam{\DUrole{n,n}{wavelength}\DUrole{p,p}{:}\DUrole{w,w}{  }\DUrole{n,n}{\sphinxhref{https://numpy.org/doc/stable/reference/generated/numpy.ndarray.html\#numpy.ndarray}{ndarray}%
\begin{footnote}[73]\sphinxAtStartFootnote
\sphinxnolinkurl{https://numpy.org/doc/stable/reference/generated/numpy.ndarray.html\#numpy.ndarray}
%
\end{footnote}}}, \sphinxparam{\DUrole{n,n}{zenith\_angle}\DUrole{p,p}{:}\DUrole{w,w}{  }\DUrole{n,n}{float}}, \sphinxparam{\DUrole{n,n}{temperature}\DUrole{p,p}{:}\DUrole{w,w}{  }\DUrole{n,n}{float}}, \sphinxparam{\DUrole{n,n}{pressure}\DUrole{p,p}{:}\DUrole{w,w}{  }\DUrole{n,n}{float}}, \sphinxparam{\DUrole{n,n}{water\_pressure}\DUrole{p,p}{:}\DUrole{w,w}{  }\DUrole{n,n}{float}}}{{ $\rightarrow$ Callable\DUrole{p,p}{{[}}\DUrole{p,p}{{[}}\sphinxhref{https://numpy.org/doc/stable/reference/generated/numpy.ndarray.html\#numpy.ndarray}{ndarray}%
\begin{footnote}[74]\sphinxAtStartFootnote
\sphinxnolinkurl{https://numpy.org/doc/stable/reference/generated/numpy.ndarray.html\#numpy.ndarray}
%
\end{footnote}\DUrole{p,p}{{]}}\DUrole{p,p}{,}\DUrole{w,w}{  }\sphinxhref{https://numpy.org/doc/stable/reference/generated/numpy.ndarray.html\#numpy.ndarray}{ndarray}%
\begin{footnote}[75]\sphinxAtStartFootnote
\sphinxnolinkurl{https://numpy.org/doc/stable/reference/generated/numpy.ndarray.html\#numpy.ndarray}
%
\end{footnote}\DUrole{p,p}{{]}}}}
\pysigstopsignatures
\sphinxAtStartPar
Compute the absolute atmospheric refraction function.

\sphinxAtStartPar
The absolute atmospheric refraction is not as useful as the relative
atmospheric refraction function. To calculate how the atmosphere refracts
one’s object, use that function instead.
\begin{quote}\begin{description}
\sphinxlineitem{Parameters}\begin{itemize}
\item {} 
\sphinxAtStartPar
\sphinxstyleliteralstrong{\sphinxupquote{wavelength}} (\sphinxstyleliteralemphasis{\sphinxupquote{ndarray}}) – The wavelength over which the absolute atmospheric refraction is
being computed over, in microns.

\item {} 
\sphinxAtStartPar
\sphinxstyleliteralstrong{\sphinxupquote{zenith\_angle}} (\sphinxstyleliteralemphasis{\sphinxupquote{float}}) – The zenith angle of the sight line, in radians.

\item {} 
\sphinxAtStartPar
\sphinxstyleliteralstrong{\sphinxupquote{temperature}} (\sphinxstyleliteralemphasis{\sphinxupquote{float}}) – The temperature of the atmosphere, in Kelvin.

\item {} 
\sphinxAtStartPar
\sphinxstyleliteralstrong{\sphinxupquote{pressure}} (\sphinxstyleliteralemphasis{\sphinxupquote{float}}) – The pressure of the atmosphere, in Pascals.

\item {} 
\sphinxAtStartPar
\sphinxstyleliteralstrong{\sphinxupquote{water\_pressure}} (\sphinxstyleliteralemphasis{\sphinxupquote{float}}) – The partial pressure of water in the atmosphere, Pascals.

\end{itemize}

\sphinxlineitem{Returns}
\sphinxAtStartPar
\sphinxstylestrong{abs\_atm\_refr\_func} – The absolute atmospheric refraction function, as an actual callable
function.

\sphinxlineitem{Return type}
\sphinxAtStartPar
Callable

\end{description}\end{quote}

\end{fulllineitems}\end{savenotes}

\index{airmass() (in module lezargus.library.atmosphere)@\spxentry{airmass()}\spxextra{in module lezargus.library.atmosphere}}

\begin{savenotes}\begin{fulllineitems}
\phantomsection\label{\detokenize{code/lezargus.library.atmosphere:lezargus.library.atmosphere.airmass}}
\pysigstartsignatures
\pysiglinewithargsret{\sphinxcode{\sphinxupquote{lezargus.library.atmosphere.}}\sphinxbfcode{\sphinxupquote{airmass}}}{\sphinxparam{\DUrole{n,n}{zenith\_angle}\DUrole{p,p}{:}\DUrole{w,w}{  }\DUrole{n,n}{float\DUrole{w,w}{  }\DUrole{p,p}{|}\DUrole{w,w}{  }\sphinxhref{https://numpy.org/doc/stable/reference/generated/numpy.ndarray.html\#numpy.ndarray}{ndarray}%
\begin{footnote}[76]\sphinxAtStartFootnote
\sphinxnolinkurl{https://numpy.org/doc/stable/reference/generated/numpy.ndarray.html\#numpy.ndarray}
%
\end{footnote}}}}{{ $\rightarrow$ float\DUrole{w,w}{  }\DUrole{p,p}{|}\DUrole{w,w}{  }\sphinxhref{https://numpy.org/doc/stable/reference/generated/numpy.ndarray.html\#numpy.ndarray}{ndarray}%
\begin{footnote}[77]\sphinxAtStartFootnote
\sphinxnolinkurl{https://numpy.org/doc/stable/reference/generated/numpy.ndarray.html\#numpy.ndarray}
%
\end{footnote}}}
\pysigstopsignatures
\sphinxAtStartPar
Calculate the airmass from the zenith angle.

\sphinxAtStartPar
This function calculates the airmass provided a zenith angle. For most
cases the plane\sphinxhyphen{}parallel atmosphere method works, and it is what this
function uses. However, we also use a more accurate formula for airmass at
higher zenith angles (>80 degree), namely from DOI:10.1364/AO.28.004735.
We use a weighted average between 75 < z < 80 degrees to allow for a
smooth transition.
\begin{quote}\begin{description}
\sphinxlineitem{Parameters}
\sphinxAtStartPar
\sphinxstyleliteralstrong{\sphinxupquote{zenith\_angle}} (\sphinxstyleliteralemphasis{\sphinxupquote{float}}\sphinxstyleliteralemphasis{\sphinxupquote{ or }}\sphinxstyleliteralemphasis{\sphinxupquote{ndarray}}) – The zenith angle, in radians.

\sphinxlineitem{Returns}
\sphinxAtStartPar
\sphinxstylestrong{airmass\_} – The airmass. The variable name is to avoid name conflicts.

\sphinxlineitem{Return type}
\sphinxAtStartPar
float or ndarray

\end{description}\end{quote}

\end{fulllineitems}\end{savenotes}

\index{gaussian\_psf\_kernel() (in module lezargus.library.atmosphere)@\spxentry{gaussian\_psf\_kernel()}\spxextra{in module lezargus.library.atmosphere}}

\begin{savenotes}\begin{fulllineitems}
\phantomsection\label{\detokenize{code/lezargus.library.atmosphere:lezargus.library.atmosphere.gaussian_psf_kernel}}
\pysigstartsignatures
\pysiglinewithargsret{\sphinxcode{\sphinxupquote{lezargus.library.atmosphere.}}\sphinxbfcode{\sphinxupquote{gaussian\_psf\_kernel}}}{\sphinxparam{\DUrole{n,n}{shape}\DUrole{p,p}{:}\DUrole{w,w}{  }\DUrole{n,n}{tuple}}, \sphinxparam{\DUrole{n,n}{x\_stddev}\DUrole{p,p}{:}\DUrole{w,w}{  }\DUrole{n,n}{float}}, \sphinxparam{\DUrole{n,n}{y\_stddev}\DUrole{p,p}{:}\DUrole{w,w}{  }\DUrole{n,n}{float}}, \sphinxparam{\DUrole{n,n}{rotation}\DUrole{p,p}{:}\DUrole{w,w}{  }\DUrole{n,n}{float}}}{{ $\rightarrow$ \sphinxhref{https://numpy.org/doc/stable/reference/generated/numpy.ndarray.html\#numpy.ndarray}{ndarray}%
\begin{footnote}[78]\sphinxAtStartFootnote
\sphinxnolinkurl{https://numpy.org/doc/stable/reference/generated/numpy.ndarray.html\#numpy.ndarray}
%
\end{footnote}}}
\pysigstopsignatures
\sphinxAtStartPar
Return a 2D Gaussian point spread function convolution kernel.

\sphinxAtStartPar
We normalize the point spread function via the amplitude of the Gaussian
function as a whole for maximal precision: volume = 1. We require the
input of the shape of the kernel to allow for \sphinxtitleref{x\_stddev} and \sphinxtitleref{y\_stddev}
to be expressed in pixels to keep it general. By definition, the center
of the Gaussian kernel is in the center of the array.
\begin{quote}\begin{description}
\sphinxlineitem{Parameters}\begin{itemize}
\item {} 
\sphinxAtStartPar
\sphinxstyleliteralstrong{\sphinxupquote{shape}} (\sphinxstyleliteralemphasis{\sphinxupquote{tuple}}) – The shape of the 2D kernel, in pixels.

\item {} 
\sphinxAtStartPar
\sphinxstyleliteralstrong{\sphinxupquote{x\_stddev}} (\sphinxstyleliteralemphasis{\sphinxupquote{float}}) – The standard deviation of the Gaussian in the x direction, in pixels.

\item {} 
\sphinxAtStartPar
\sphinxstyleliteralstrong{\sphinxupquote{y\_stddev}} (\sphinxstyleliteralemphasis{\sphinxupquote{float}}) – The standard deviation of the Gaussian in the y direction, in pixels.

\item {} 
\sphinxAtStartPar
\sphinxstyleliteralstrong{\sphinxupquote{rotation}} (\sphinxstyleliteralemphasis{\sphinxupquote{float}}) – The rotation angle, increasing counterclockwise, in radians.

\end{itemize}

\sphinxlineitem{Returns}
\sphinxAtStartPar
\sphinxstylestrong{gaussian\_kernel} – The discrete kernel array.

\sphinxlineitem{Return type}
\sphinxAtStartPar
ndarray

\end{description}\end{quote}

\end{fulllineitems}\end{savenotes}

\index{index\_of\_refraction\_dry\_air() (in module lezargus.library.atmosphere)@\spxentry{index\_of\_refraction\_dry\_air()}\spxextra{in module lezargus.library.atmosphere}}

\begin{savenotes}\begin{fulllineitems}
\phantomsection\label{\detokenize{code/lezargus.library.atmosphere:lezargus.library.atmosphere.index_of_refraction_dry_air}}
\pysigstartsignatures
\pysiglinewithargsret{\sphinxcode{\sphinxupquote{lezargus.library.atmosphere.}}\sphinxbfcode{\sphinxupquote{index\_of\_refraction\_dry\_air}}}{\sphinxparam{\DUrole{n,n}{wavelength}\DUrole{p,p}{:}\DUrole{w,w}{  }\DUrole{n,n}{\sphinxhref{https://numpy.org/doc/stable/reference/generated/numpy.ndarray.html\#numpy.ndarray}{ndarray}%
\begin{footnote}[79]\sphinxAtStartFootnote
\sphinxnolinkurl{https://numpy.org/doc/stable/reference/generated/numpy.ndarray.html\#numpy.ndarray}
%
\end{footnote}}}, \sphinxparam{\DUrole{n,n}{pressure}\DUrole{p,p}{:}\DUrole{w,w}{  }\DUrole{n,n}{float}}, \sphinxparam{\DUrole{n,n}{temperature}\DUrole{p,p}{:}\DUrole{w,w}{  }\DUrole{n,n}{float}}}{{ $\rightarrow$ \sphinxhref{https://numpy.org/doc/stable/reference/generated/numpy.ndarray.html\#numpy.ndarray}{ndarray}%
\begin{footnote}[80]\sphinxAtStartFootnote
\sphinxnolinkurl{https://numpy.org/doc/stable/reference/generated/numpy.ndarray.html\#numpy.ndarray}
%
\end{footnote}}}
\pysigstopsignatures
\sphinxAtStartPar
Calculate the refraction of air of pressured warm dry air.

\sphinxAtStartPar
The index of refraction depends on wavelength, pressure and temperature, we
use the updated Edlén equations found in DOI: 10.1088/0026\sphinxhyphen{}1394/30/3/004.
\begin{quote}\begin{description}
\sphinxlineitem{Parameters}\begin{itemize}
\item {} 
\sphinxAtStartPar
\sphinxstyleliteralstrong{\sphinxupquote{wavelength}} (\sphinxstyleliteralemphasis{\sphinxupquote{ndarray}}) – The wavelength that we are calculating the index of refraction over.
This must in microns.

\item {} 
\sphinxAtStartPar
\sphinxstyleliteralstrong{\sphinxupquote{pressure}} (\sphinxstyleliteralemphasis{\sphinxupquote{float}}) – The pressure of the atmosphere, in Pascals.

\item {} 
\sphinxAtStartPar
\sphinxstyleliteralstrong{\sphinxupquote{temperature}} (\sphinxstyleliteralemphasis{\sphinxupquote{float}}) – The temperature of the atmosphere, in Kelvin.

\end{itemize}

\sphinxlineitem{Returns}
\sphinxAtStartPar
\sphinxstylestrong{ior\_dry\_air} – The dry air index of refraction.

\sphinxlineitem{Return type}
\sphinxAtStartPar
ndarray

\end{description}\end{quote}

\end{fulllineitems}\end{savenotes}

\index{index\_of\_refraction\_ideal\_air() (in module lezargus.library.atmosphere)@\spxentry{index\_of\_refraction\_ideal\_air()}\spxextra{in module lezargus.library.atmosphere}}

\begin{savenotes}\begin{fulllineitems}
\phantomsection\label{\detokenize{code/lezargus.library.atmosphere:lezargus.library.atmosphere.index_of_refraction_ideal_air}}
\pysigstartsignatures
\pysiglinewithargsret{\sphinxcode{\sphinxupquote{lezargus.library.atmosphere.}}\sphinxbfcode{\sphinxupquote{index\_of\_refraction\_ideal\_air}}}{\sphinxparam{\DUrole{n,n}{wavelength}\DUrole{p,p}{:}\DUrole{w,w}{  }\DUrole{n,n}{\sphinxhref{https://numpy.org/doc/stable/reference/generated/numpy.ndarray.html\#numpy.ndarray}{ndarray}%
\begin{footnote}[81]\sphinxAtStartFootnote
\sphinxnolinkurl{https://numpy.org/doc/stable/reference/generated/numpy.ndarray.html\#numpy.ndarray}
%
\end{footnote}}}}{{ $\rightarrow$ \sphinxhref{https://numpy.org/doc/stable/reference/generated/numpy.ndarray.html\#numpy.ndarray}{ndarray}%
\begin{footnote}[82]\sphinxAtStartFootnote
\sphinxnolinkurl{https://numpy.org/doc/stable/reference/generated/numpy.ndarray.html\#numpy.ndarray}
%
\end{footnote}}}
\pysigstopsignatures
\sphinxAtStartPar
Calculate the ideal refraction of air over wavelength.

\sphinxAtStartPar
The index of refraction of air depends slightly on wavelength, we use
the updated Edlen equations found in DOI: 10.1088/0026\sphinxhyphen{}1394/30/3/004.
\begin{quote}\begin{description}
\sphinxlineitem{Parameters}
\sphinxAtStartPar
\sphinxstyleliteralstrong{\sphinxupquote{wavelength}} (\sphinxstyleliteralemphasis{\sphinxupquote{ndarray}}) – The wavelength that we are calculating the index of refraction over.
This must in microns.

\sphinxlineitem{Returns}
\sphinxAtStartPar
\sphinxstylestrong{ior\_ideal\_air} – The ideal air index of refraction.

\sphinxlineitem{Return type}
\sphinxAtStartPar
ndarray

\end{description}\end{quote}

\end{fulllineitems}\end{savenotes}

\index{index\_of\_refraction\_moist\_air() (in module lezargus.library.atmosphere)@\spxentry{index\_of\_refraction\_moist\_air()}\spxextra{in module lezargus.library.atmosphere}}

\begin{savenotes}\begin{fulllineitems}
\phantomsection\label{\detokenize{code/lezargus.library.atmosphere:lezargus.library.atmosphere.index_of_refraction_moist_air}}
\pysigstartsignatures
\pysiglinewithargsret{\sphinxcode{\sphinxupquote{lezargus.library.atmosphere.}}\sphinxbfcode{\sphinxupquote{index\_of\_refraction\_moist\_air}}}{\sphinxparam{\DUrole{n,n}{wavelength}\DUrole{p,p}{:}\DUrole{w,w}{  }\DUrole{n,n}{\sphinxhref{https://numpy.org/doc/stable/reference/generated/numpy.ndarray.html\#numpy.ndarray}{ndarray}%
\begin{footnote}[83]\sphinxAtStartFootnote
\sphinxnolinkurl{https://numpy.org/doc/stable/reference/generated/numpy.ndarray.html\#numpy.ndarray}
%
\end{footnote}}}, \sphinxparam{\DUrole{n,n}{temperature}\DUrole{p,p}{:}\DUrole{w,w}{  }\DUrole{n,n}{float}}, \sphinxparam{\DUrole{n,n}{pressure}\DUrole{p,p}{:}\DUrole{w,w}{  }\DUrole{n,n}{float}}, \sphinxparam{\DUrole{n,n}{water\_pressure}\DUrole{p,p}{:}\DUrole{w,w}{  }\DUrole{n,n}{float}}}{{ $\rightarrow$ \sphinxhref{https://numpy.org/doc/stable/reference/generated/numpy.ndarray.html\#numpy.ndarray}{ndarray}%
\begin{footnote}[84]\sphinxAtStartFootnote
\sphinxnolinkurl{https://numpy.org/doc/stable/reference/generated/numpy.ndarray.html\#numpy.ndarray}
%
\end{footnote}}}
\pysigstopsignatures
\sphinxAtStartPar
Calculate the refraction of air of pressured warm moist air.

\sphinxAtStartPar
The index of refraction depends on wavelength, pressure, temperature, and
humidity, we use the updated Edlen equations found in
DOI: 10.1088/0026\sphinxhyphen{}1394/30/3/004. We use the partial pressure of water in
the atmosphere as opposed to actual humidity.
\begin{quote}\begin{description}
\sphinxlineitem{Parameters}\begin{itemize}
\item {} 
\sphinxAtStartPar
\sphinxstyleliteralstrong{\sphinxupquote{wavelength}} (\sphinxstyleliteralemphasis{\sphinxupquote{ndarray}}) – The wavelength that we are calculating the index of refraction over.
This must in microns.

\item {} 
\sphinxAtStartPar
\sphinxstyleliteralstrong{\sphinxupquote{temperature}} (\sphinxstyleliteralemphasis{\sphinxupquote{float}}) – The temperature of the atmosphere, in Kelvin.

\item {} 
\sphinxAtStartPar
\sphinxstyleliteralstrong{\sphinxupquote{pressure}} (\sphinxstyleliteralemphasis{\sphinxupquote{float}}) – The pressure of the atmosphere, in Pascals.

\item {} 
\sphinxAtStartPar
\sphinxstyleliteralstrong{\sphinxupquote{water\_pressure}} (\sphinxstyleliteralemphasis{\sphinxupquote{float}}) – The partial pressure of water in the atmosphere, Pascals.

\end{itemize}

\sphinxlineitem{Returns}
\sphinxAtStartPar
\sphinxstylestrong{ior\_moist\_air} – The moist air index of refraction.

\sphinxlineitem{Return type}
\sphinxAtStartPar
ndarray

\end{description}\end{quote}

\end{fulllineitems}\end{savenotes}

\index{relative\_atmospheric\_refraction\_function() (in module lezargus.library.atmosphere)@\spxentry{relative\_atmospheric\_refraction\_function()}\spxextra{in module lezargus.library.atmosphere}}

\begin{savenotes}\begin{fulllineitems}
\phantomsection\label{\detokenize{code/lezargus.library.atmosphere:lezargus.library.atmosphere.relative_atmospheric_refraction_function}}
\pysigstartsignatures
\pysiglinewithargsret{\sphinxcode{\sphinxupquote{lezargus.library.atmosphere.}}\sphinxbfcode{\sphinxupquote{relative\_atmospheric\_refraction\_function}}}{\sphinxparam{\DUrole{n,n}{wavelength}\DUrole{p,p}{:}\DUrole{w,w}{  }\DUrole{n,n}{\sphinxhref{https://numpy.org/doc/stable/reference/generated/numpy.ndarray.html\#numpy.ndarray}{ndarray}%
\begin{footnote}[85]\sphinxAtStartFootnote
\sphinxnolinkurl{https://numpy.org/doc/stable/reference/generated/numpy.ndarray.html\#numpy.ndarray}
%
\end{footnote}}}, \sphinxparam{\DUrole{n,n}{reference\_wavelength}\DUrole{p,p}{:}\DUrole{w,w}{  }\DUrole{n,n}{float}}, \sphinxparam{\DUrole{n,n}{zenith\_angle}\DUrole{p,p}{:}\DUrole{w,w}{  }\DUrole{n,n}{float}}, \sphinxparam{\DUrole{n,n}{temperature}\DUrole{p,p}{:}\DUrole{w,w}{  }\DUrole{n,n}{float}}, \sphinxparam{\DUrole{n,n}{pressure}\DUrole{p,p}{:}\DUrole{w,w}{  }\DUrole{n,n}{float}}, \sphinxparam{\DUrole{n,n}{water\_pressure}\DUrole{p,p}{:}\DUrole{w,w}{  }\DUrole{n,n}{float}}}{{ $\rightarrow$ Callable\DUrole{p,p}{{[}}\DUrole{p,p}{{[}}\sphinxhref{https://numpy.org/doc/stable/reference/generated/numpy.ndarray.html\#numpy.ndarray}{ndarray}%
\begin{footnote}[86]\sphinxAtStartFootnote
\sphinxnolinkurl{https://numpy.org/doc/stable/reference/generated/numpy.ndarray.html\#numpy.ndarray}
%
\end{footnote}\DUrole{p,p}{{]}}\DUrole{p,p}{,}\DUrole{w,w}{  }\sphinxhref{https://numpy.org/doc/stable/reference/generated/numpy.ndarray.html\#numpy.ndarray}{ndarray}%
\begin{footnote}[87]\sphinxAtStartFootnote
\sphinxnolinkurl{https://numpy.org/doc/stable/reference/generated/numpy.ndarray.html\#numpy.ndarray}
%
\end{footnote}\DUrole{p,p}{{]}}}}
\pysigstopsignatures
\sphinxAtStartPar
Compute the relative atmospheric refraction function.

\sphinxAtStartPar
The relative refraction function is the same as the absolute refraction
function, however, it is all relative to some specific wavelength.
\begin{quote}\begin{description}
\sphinxlineitem{Parameters}\begin{itemize}
\item {} 
\sphinxAtStartPar
\sphinxstyleliteralstrong{\sphinxupquote{wavelength}} (\sphinxstyleliteralemphasis{\sphinxupquote{ndarray}}) – The wavelength over which the absolute atmospheric refraction is
being computed over, in microns.

\item {} 
\sphinxAtStartPar
\sphinxstyleliteralstrong{\sphinxupquote{reference\_wavelength}} (\sphinxstyleliteralemphasis{\sphinxupquote{float}}) – The reference wavelength which the relative refraction is computed
against, in microns.

\item {} 
\sphinxAtStartPar
\sphinxstyleliteralstrong{\sphinxupquote{zenith\_angle}} (\sphinxstyleliteralemphasis{\sphinxupquote{float}}) – The zenith angle of the sight line, in radians.

\item {} 
\sphinxAtStartPar
\sphinxstyleliteralstrong{\sphinxupquote{temperature}} (\sphinxstyleliteralemphasis{\sphinxupquote{float}}) – The temperature of the atmosphere, in Kelvin.

\item {} 
\sphinxAtStartPar
\sphinxstyleliteralstrong{\sphinxupquote{pressure}} (\sphinxstyleliteralemphasis{\sphinxupquote{float}}) – The pressure of the atmosphere, in Pascals.

\item {} 
\sphinxAtStartPar
\sphinxstyleliteralstrong{\sphinxupquote{water\_pressure}} (\sphinxstyleliteralemphasis{\sphinxupquote{float}}) – The partial pressure of water in the atmosphere, Pascals.

\end{itemize}

\sphinxlineitem{Returns}
\sphinxAtStartPar
\sphinxstylestrong{rel\_atm\_refr\_func} – The absolute atmospheric refraction function, as an actual callable
function.

\sphinxlineitem{Return type}
\sphinxAtStartPar
Callable

\end{description}\end{quote}

\end{fulllineitems}\end{savenotes}


\sphinxstepscope


\subparagraph{lezargus.library.config module}
\label{\detokenize{code/lezargus.library.config:module-lezargus.library.config}}\label{\detokenize{code/lezargus.library.config:lezargus-library-config-module}}\label{\detokenize{code/lezargus.library.config::doc}}\index{module@\spxentry{module}!lezargus.library.config@\spxentry{lezargus.library.config}}\index{lezargus.library.config@\spxentry{lezargus.library.config}!module@\spxentry{module}}
\sphinxAtStartPar
Controls the inputting of configuration files.

\sphinxAtStartPar
This also serves to bring all of the configuration parameters into a more
accessible space which other parts of Lezargus can use.

\sphinxAtStartPar
Note these configuration constant parameters are all accessed using capital
letters regardless of the configuration file’s labels. Moreover, there are
constant parameters which are stored here which are not otherwise changeable
by the configuration file.
\index{generate\_configuration\_file\_copy() (in module lezargus.library.config)@\spxentry{generate\_configuration\_file\_copy()}\spxextra{in module lezargus.library.config}}

\begin{savenotes}\begin{fulllineitems}
\phantomsection\label{\detokenize{code/lezargus.library.config:lezargus.library.config.generate_configuration_file_copy}}
\pysigstartsignatures
\pysiglinewithargsret{\sphinxcode{\sphinxupquote{lezargus.library.config.}}\sphinxbfcode{\sphinxupquote{generate\_configuration\_file\_copy}}}{\sphinxparam{\DUrole{n,n}{filename}\DUrole{p,p}{:}\DUrole{w,w}{  }\DUrole{n,n}{str}}, \sphinxparam{\DUrole{n,n}{overwrite}\DUrole{p,p}{:}\DUrole{w,w}{  }\DUrole{n,n}{bool}\DUrole{w,w}{  }\DUrole{o,o}{=}\DUrole{w,w}{  }\DUrole{default_value}{False}}}{{ $\rightarrow$ None}}
\pysigstopsignatures
\sphinxAtStartPar
Generate a copy of the default configuration file to the given location.
\begin{quote}\begin{description}
\sphinxlineitem{Parameters}\begin{itemize}
\item {} 
\sphinxAtStartPar
\sphinxstyleliteralstrong{\sphinxupquote{filename}} (\sphinxstyleliteralemphasis{\sphinxupquote{str}}) – The pathname or filename where the configuration file should be put
to. If it does not have the proper yaml extension, it will be added.

\item {} 
\sphinxAtStartPar
\sphinxstyleliteralstrong{\sphinxupquote{overwrite}} (\sphinxstyleliteralemphasis{\sphinxupquote{bool}}\sphinxstyleliteralemphasis{\sphinxupquote{, }}\sphinxstyleliteralemphasis{\sphinxupquote{default = False}}) – If the file already exists, overwrite it. If False, it would raise
an error instead.

\end{itemize}

\sphinxlineitem{Return type}
\sphinxAtStartPar
None

\end{description}\end{quote}

\end{fulllineitems}\end{savenotes}

\index{load\_configuration\_file() (in module lezargus.library.config)@\spxentry{load\_configuration\_file()}\spxextra{in module lezargus.library.config}}

\begin{savenotes}\begin{fulllineitems}
\phantomsection\label{\detokenize{code/lezargus.library.config:lezargus.library.config.load_configuration_file}}
\pysigstartsignatures
\pysiglinewithargsret{\sphinxcode{\sphinxupquote{lezargus.library.config.}}\sphinxbfcode{\sphinxupquote{load\_configuration\_file}}}{\sphinxparam{\DUrole{n,n}{filename}\DUrole{p,p}{:}\DUrole{w,w}{  }\DUrole{n,n}{str}}}{{ $\rightarrow$ dict}}
\pysigstopsignatures
\sphinxAtStartPar
Load the configuration file and output a dictionary of parameters.

\sphinxAtStartPar
Note configuration files should be flat, there should be no nested
configuration parameters.
\begin{quote}\begin{description}
\sphinxlineitem{Parameters}
\sphinxAtStartPar
\sphinxstyleliteralstrong{\sphinxupquote{filename}} (\sphinxstyleliteralemphasis{\sphinxupquote{str}}) – The filename of the configuration file, with the extension. Will raise
if the filename is not the correct extension, just as a quick check.

\sphinxlineitem{Returns}
\sphinxAtStartPar
\sphinxstylestrong{configuration\_dict} – The dictionary which contains all of the configuration parameters
within it.

\sphinxlineitem{Return type}
\sphinxAtStartPar
dictionary

\end{description}\end{quote}

\end{fulllineitems}\end{savenotes}

\index{load\_then\_apply\_configuration() (in module lezargus.library.config)@\spxentry{load\_then\_apply\_configuration()}\spxextra{in module lezargus.library.config}}

\begin{savenotes}\begin{fulllineitems}
\phantomsection\label{\detokenize{code/lezargus.library.config:lezargus.library.config.load_then_apply_configuration}}
\pysigstartsignatures
\pysiglinewithargsret{\sphinxcode{\sphinxupquote{lezargus.library.config.}}\sphinxbfcode{\sphinxupquote{load\_then\_apply\_configuration}}}{\sphinxparam{\DUrole{n,n}{filename}\DUrole{p,p}{:}\DUrole{w,w}{  }\DUrole{n,n}{str}}}{{ $\rightarrow$ None}}
\pysigstopsignatures
\sphinxAtStartPar
Load a configuration file, then applies it to the entire Lezargus system.

\sphinxAtStartPar
Loads a configuration file and overwrites any overlapping
configurations. It writes the configuration to the configuration module
for usage throughout the entire program.

\sphinxAtStartPar
Note configuration files should be flat, there should be no nested
configuration parameters.
\begin{quote}\begin{description}
\sphinxlineitem{Parameters}
\sphinxAtStartPar
\sphinxstyleliteralstrong{\sphinxupquote{filename}} (\sphinxstyleliteralemphasis{\sphinxupquote{str}}) – The filename of the configuration file, with the extension. Will raise
if the filename is not the correct extension, just as a quick check.

\sphinxlineitem{Return type}
\sphinxAtStartPar
None

\end{description}\end{quote}

\end{fulllineitems}\end{savenotes}


\sphinxstepscope


\subparagraph{lezargus.library.conversion module}
\label{\detokenize{code/lezargus.library.conversion:module-lezargus.library.conversion}}\label{\detokenize{code/lezargus.library.conversion:lezargus-library-conversion-module}}\label{\detokenize{code/lezargus.library.conversion::doc}}\index{module@\spxentry{module}!lezargus.library.conversion@\spxentry{lezargus.library.conversion}}\index{lezargus.library.conversion@\spxentry{lezargus.library.conversion}!module@\spxentry{module}}
\sphinxAtStartPar
Functions to convert things into something else.

\sphinxAtStartPar
Any and all generic conversions (string, units, or otherwise) can be found in
here. Extremely standard conversion functions are welcome in here, but,
sometimes, a simple multiplication factor is more effective.
\index{convert\_to\_allowable\_fits\_header\_data\_types() (in module lezargus.library.conversion)@\spxentry{convert\_to\_allowable\_fits\_header\_data\_types()}\spxextra{in module lezargus.library.conversion}}

\begin{savenotes}\begin{fulllineitems}
\phantomsection\label{\detokenize{code/lezargus.library.conversion:lezargus.library.conversion.convert_to_allowable_fits_header_data_types}}
\pysigstartsignatures
\pysiglinewithargsret{\sphinxcode{\sphinxupquote{lezargus.library.conversion.}}\sphinxbfcode{\sphinxupquote{convert\_to\_allowable\_fits\_header\_data\_types}}}{\sphinxparam{\DUrole{n,n}{input\_data}\DUrole{p,p}{:}\DUrole{w,w}{  }\DUrole{n,n}{object}}}{{ $\rightarrow$ str\DUrole{w,w}{  }\DUrole{p,p}{|}\DUrole{w,w}{  }int\DUrole{w,w}{  }\DUrole{p,p}{|}\DUrole{w,w}{  }float\DUrole{w,w}{  }\DUrole{p,p}{|}\DUrole{w,w}{  }bool\DUrole{w,w}{  }\DUrole{p,p}{|}\DUrole{w,w}{  }Undefined}}
\pysigstopsignatures
\sphinxAtStartPar
Convert any input into something FITS headers allow.

\sphinxAtStartPar
Per the FITS standard, the allowable data types which values entered in
FITS headers is a subset of what Python can do. As such, this function
converts any type of reasonable input into something the FITS headers
would allow. Note, we mostly do basic checking and conversions. If the
object is too exotic, it may cause issues down the line.

\sphinxAtStartPar
In general, only strings, integers, floating point, boolean, and no values
are allowed. Astropy usually will handle further conversion from the basic
Python types so we only convert up to there.
\begin{quote}\begin{description}
\sphinxlineitem{Parameters}
\sphinxAtStartPar
\sphinxstyleliteralstrong{\sphinxupquote{input\_data}} (\sphinxstyleliteralemphasis{\sphinxupquote{object}}) – The input to convert into an allowable FITS header keyword.

\sphinxlineitem{Returns}
\sphinxAtStartPar
\sphinxstylestrong{header\_output} – The output after conversion. Note the None is not actually a None
type itself, but Astropy’s header None/Undefined type.

\sphinxlineitem{Return type}
\sphinxAtStartPar
str, int, float, bool, or None

\end{description}\end{quote}

\end{fulllineitems}\end{savenotes}

\index{parse\_unit\_to\_astropy\_unit() (in module lezargus.library.conversion)@\spxentry{parse\_unit\_to\_astropy\_unit()}\spxextra{in module lezargus.library.conversion}}

\begin{savenotes}\begin{fulllineitems}
\phantomsection\label{\detokenize{code/lezargus.library.conversion:lezargus.library.conversion.parse_unit_to_astropy_unit}}
\pysigstartsignatures
\pysiglinewithargsret{\sphinxcode{\sphinxupquote{lezargus.library.conversion.}}\sphinxbfcode{\sphinxupquote{parse\_unit\_to\_astropy\_unit}}}{\sphinxparam{\DUrole{n,n}{unit\_string}\DUrole{p,p}{:}\DUrole{w,w}{  }\DUrole{n,n}{str}}}{{ $\rightarrow$ \sphinxhref{https://docs.astropy.org/en/stable/api/astropy.units.Unit.html\#astropy.units.Unit}{Unit}%
\begin{footnote}[88]\sphinxAtStartFootnote
\sphinxnolinkurl{https://docs.astropy.org/en/stable/api/astropy.units.Unit.html\#astropy.units.Unit}
%
\end{footnote}}}
\pysigstopsignatures
\sphinxAtStartPar
Parse a unit string to an Astropy Unit class.

\sphinxAtStartPar
Although for most cases, it is easier to use the Unit instantiation class
directly, Astropy does not properly understand some unit conventions so
we need to parse them in manually. Because of this, we just build a unified
interface for all unit strings in general.
\begin{quote}\begin{description}
\sphinxlineitem{Parameters}
\sphinxAtStartPar
\sphinxstyleliteralstrong{\sphinxupquote{unit\_string}} (\sphinxstyleliteralemphasis{\sphinxupquote{str}}) – The unit string to parse into an Astropy unit. If it is None, then we
return a dimensionless quantity unit.

\sphinxlineitem{Returns}
\sphinxAtStartPar
\sphinxstylestrong{unit\_instance} – The unit instance after parsing.

\sphinxlineitem{Return type}
\sphinxAtStartPar
Unit

\end{description}\end{quote}

\end{fulllineitems}\end{savenotes}


\sphinxstepscope


\subparagraph{lezargus.library.data module}
\label{\detokenize{code/lezargus.library.data:module-lezargus.library.data}}\label{\detokenize{code/lezargus.library.data:lezargus-library-data-module}}\label{\detokenize{code/lezargus.library.data::doc}}\index{module@\spxentry{module}!lezargus.library.data@\spxentry{lezargus.library.data}}\index{lezargus.library.data@\spxentry{lezargus.library.data}!module@\spxentry{module}}
\sphinxAtStartPar
Data file functions.

\sphinxAtStartPar
This file deals with the loading in and saving of data files which are in
the /data/ directory of Lezargus. Moreover, the contents of the data
are accessed using attributes of this module.
\index{initialize\_data\_files() (in module lezargus.library.data)@\spxentry{initialize\_data\_files()}\spxextra{in module lezargus.library.data}}

\begin{savenotes}\begin{fulllineitems}
\phantomsection\label{\detokenize{code/lezargus.library.data:lezargus.library.data.initialize_data_files}}
\pysigstartsignatures
\pysiglinewithargsret{\sphinxcode{\sphinxupquote{lezargus.library.data.}}\sphinxbfcode{\sphinxupquote{initialize\_data\_files}}}{}{{ $\rightarrow$ None}}
\pysigstopsignatures
\sphinxAtStartPar
Create all of the data files and instances of classes.

\sphinxAtStartPar
This function creates all of the data objects which represent all of the
data and saves it to this module. This must be done in a function, and
called by the initialization of the module, to avoid import errors and
dependency issues.
\begin{quote}\begin{description}
\sphinxlineitem{Parameters}
\sphinxAtStartPar
\sphinxstyleliteralstrong{\sphinxupquote{None}} – 

\sphinxlineitem{Return type}
\sphinxAtStartPar
None

\end{description}\end{quote}

\end{fulllineitems}\end{savenotes}


\sphinxstepscope


\subparagraph{lezargus.library.fits module}
\label{\detokenize{code/lezargus.library.fits:module-lezargus.library.fits}}\label{\detokenize{code/lezargus.library.fits:lezargus-library-fits-module}}\label{\detokenize{code/lezargus.library.fits::doc}}\index{module@\spxentry{module}!lezargus.library.fits@\spxentry{lezargus.library.fits}}\index{lezargus.library.fits@\spxentry{lezargus.library.fits}!module@\spxentry{module}}
\sphinxAtStartPar
FITS file reading, writing, and other manipulations.
\index{create\_lezargus\_fits\_header() (in module lezargus.library.fits)@\spxentry{create\_lezargus\_fits\_header()}\spxextra{in module lezargus.library.fits}}

\begin{savenotes}\begin{fulllineitems}
\phantomsection\label{\detokenize{code/lezargus.library.fits:lezargus.library.fits.create_lezargus_fits_header}}
\pysigstartsignatures
\pysiglinewithargsret{\sphinxcode{\sphinxupquote{lezargus.library.fits.}}\sphinxbfcode{\sphinxupquote{create\_lezargus\_fits\_header}}}{\sphinxparam{\DUrole{n,n}{header}\DUrole{p,p}{:}\DUrole{w,w}{  }\DUrole{n,n}{\sphinxhref{https://docs.astropy.org/en/stable/io/fits/api/headers.html\#astropy.io.fits.Header}{Header}%
\begin{footnote}[89]\sphinxAtStartFootnote
\sphinxnolinkurl{https://docs.astropy.org/en/stable/io/fits/api/headers.html\#astropy.io.fits.Header}
%
\end{footnote}}}, \sphinxparam{\DUrole{n,n}{entries}\DUrole{p,p}{:}\DUrole{w,w}{  }\DUrole{n,n}{dict\DUrole{w,w}{  }\DUrole{p,p}{|}\DUrole{w,w}{  }None}\DUrole{w,w}{  }\DUrole{o,o}{=}\DUrole{w,w}{  }\DUrole{default_value}{None}}}{{ $\rightarrow$ \sphinxhref{https://docs.astropy.org/en/stable/io/fits/api/headers.html\#astropy.io.fits.Header}{Header}%
\begin{footnote}[90]\sphinxAtStartFootnote
\sphinxnolinkurl{https://docs.astropy.org/en/stable/io/fits/api/headers.html\#astropy.io.fits.Header}
%
\end{footnote}}}
\pysigstopsignatures
\sphinxAtStartPar
Create a Lezargus header.

\sphinxAtStartPar
This function creates an ordered Lezargus header from a header containing
both Lezargus keywords and non\sphinxhyphen{}Lezargus keywords. We only include the
relevant headers. WCS header information is also extracted and added as
we consider it within our domain even though it does not follow the
keyword naming convention (as WCS keywords must follow WCS convention).

\sphinxAtStartPar
Additional header entries may be provided as a last\sphinxhyphen{}minute overwrite. We
also operate on a copy of the header to prevent conflicts.
\begin{quote}\begin{description}
\sphinxlineitem{Parameters}\begin{itemize}
\item {} 
\sphinxAtStartPar
\sphinxstyleliteralstrong{\sphinxupquote{header}} (\sphinxstyleliteralemphasis{\sphinxupquote{Astropy Header}}) – The header which the entries will be added to.

\item {} 
\sphinxAtStartPar
\sphinxstyleliteralstrong{\sphinxupquote{entries}} (\sphinxstyleliteralemphasis{\sphinxupquote{dictionary}}\sphinxstyleliteralemphasis{\sphinxupquote{, }}\sphinxstyleliteralemphasis{\sphinxupquote{default = None}}) – The new entries to the header. By default, None means nothing is
to be overwritten at the last minute.

\end{itemize}

\sphinxlineitem{Returns}
\sphinxAtStartPar
\sphinxstylestrong{lezargus\_header} – The header which Lezargus entries have been be added to. The order
of the entries are specified.

\sphinxlineitem{Return type}
\sphinxAtStartPar
Astropy Header

\end{description}\end{quote}

\end{fulllineitems}\end{savenotes}

\index{create\_wcs\_header\_from\_lezargus\_header() (in module lezargus.library.fits)@\spxentry{create\_wcs\_header\_from\_lezargus\_header()}\spxextra{in module lezargus.library.fits}}

\begin{savenotes}\begin{fulllineitems}
\phantomsection\label{\detokenize{code/lezargus.library.fits:lezargus.library.fits.create_wcs_header_from_lezargus_header}}
\pysigstartsignatures
\pysiglinewithargsret{\sphinxcode{\sphinxupquote{lezargus.library.fits.}}\sphinxbfcode{\sphinxupquote{create\_wcs\_header\_from\_lezargus\_header}}}{\sphinxparam{\DUrole{n,n}{header}\DUrole{p,p}{:}\DUrole{w,w}{  }\DUrole{n,n}{\sphinxhref{https://docs.astropy.org/en/stable/io/fits/api/headers.html\#astropy.io.fits.Header}{Header}%
\begin{footnote}[91]\sphinxAtStartFootnote
\sphinxnolinkurl{https://docs.astropy.org/en/stable/io/fits/api/headers.html\#astropy.io.fits.Header}
%
\end{footnote}}}}{{ $\rightarrow$ \sphinxhref{https://docs.astropy.org/en/stable/io/fits/api/headers.html\#astropy.io.fits.Header}{Header}%
\begin{footnote}[92]\sphinxAtStartFootnote
\sphinxnolinkurl{https://docs.astropy.org/en/stable/io/fits/api/headers.html\#astropy.io.fits.Header}
%
\end{footnote}}}
\pysigstopsignatures
\sphinxAtStartPar
Create WCS header keywords from Lezargus header.

\sphinxAtStartPar
See the FITS standard for more information.
\begin{quote}\begin{description}
\sphinxlineitem{Parameters}
\sphinxAtStartPar
\sphinxstyleliteralstrong{\sphinxupquote{header}} (\sphinxstyleliteralemphasis{\sphinxupquote{Header}}) – The Lezargus header from which we will derive a WCS header from.

\sphinxlineitem{Returns}
\sphinxAtStartPar
\sphinxstylestrong{wcs\_header} – The WCS header.

\sphinxlineitem{Return type}
\sphinxAtStartPar
Header

\end{description}\end{quote}

\end{fulllineitems}\end{savenotes}

\index{read\_fits\_header() (in module lezargus.library.fits)@\spxentry{read\_fits\_header()}\spxextra{in module lezargus.library.fits}}

\begin{savenotes}\begin{fulllineitems}
\phantomsection\label{\detokenize{code/lezargus.library.fits:lezargus.library.fits.read_fits_header}}
\pysigstartsignatures
\pysiglinewithargsret{\sphinxcode{\sphinxupquote{lezargus.library.fits.}}\sphinxbfcode{\sphinxupquote{read\_fits\_header}}}{\sphinxparam{\DUrole{n,n}{filename}\DUrole{p,p}{:}\DUrole{w,w}{  }\DUrole{n,n}{str}}, \sphinxparam{\DUrole{n,n}{extension}\DUrole{p,p}{:}\DUrole{w,w}{  }\DUrole{n,n}{int\DUrole{w,w}{  }\DUrole{p,p}{|}\DUrole{w,w}{  }str}\DUrole{w,w}{  }\DUrole{o,o}{=}\DUrole{w,w}{  }\DUrole{default_value}{0}}}{{ $\rightarrow$ \sphinxhref{https://docs.astropy.org/en/stable/io/fits/api/headers.html\#astropy.io.fits.Header}{Header}%
\begin{footnote}[93]\sphinxAtStartFootnote
\sphinxnolinkurl{https://docs.astropy.org/en/stable/io/fits/api/headers.html\#astropy.io.fits.Header}
%
\end{footnote}}}
\pysigstopsignatures
\sphinxAtStartPar
Read a FITS file header.

\sphinxAtStartPar
This reads the header of fits files only. This should be used only if
there is no data. Really, this is just a wrapper around Astropy, but it
is made for consistency and to avoid the usage of the convince functions.
\begin{quote}\begin{description}
\sphinxlineitem{Parameters}\begin{itemize}
\item {} 
\sphinxAtStartPar
\sphinxstyleliteralstrong{\sphinxupquote{filename}} (\sphinxstyleliteralemphasis{\sphinxupquote{str}}) – The filename that the fits image file is at.

\item {} 
\sphinxAtStartPar
\sphinxstyleliteralstrong{\sphinxupquote{extension}} (\sphinxstyleliteralemphasis{\sphinxupquote{int}}\sphinxstyleliteralemphasis{\sphinxupquote{ or }}\sphinxstyleliteralemphasis{\sphinxupquote{str}}\sphinxstyleliteralemphasis{\sphinxupquote{, }}\sphinxstyleliteralemphasis{\sphinxupquote{default = 0}}) – The fits extension that is desired to be opened.

\end{itemize}

\sphinxlineitem{Returns}
\sphinxAtStartPar
\sphinxstylestrong{header} – The header of the fits file.

\sphinxlineitem{Return type}
\sphinxAtStartPar
Astropy Header

\end{description}\end{quote}

\end{fulllineitems}\end{savenotes}

\index{read\_lezargus\_fits\_file() (in module lezargus.library.fits)@\spxentry{read\_lezargus\_fits\_file()}\spxextra{in module lezargus.library.fits}}

\begin{savenotes}\begin{fulllineitems}
\phantomsection\label{\detokenize{code/lezargus.library.fits:lezargus.library.fits.read_lezargus_fits_file}}
\pysigstartsignatures
\pysiglinewithargsret{\sphinxcode{\sphinxupquote{lezargus.library.fits.}}\sphinxbfcode{\sphinxupquote{read\_lezargus\_fits\_file}}}{\sphinxparam{\DUrole{n,n}{filename}\DUrole{p,p}{:}\DUrole{w,w}{  }\DUrole{n,n}{str}}}{{ $\rightarrow$ tuple\DUrole{p,p}{{[}}\sphinxhref{https://docs.astropy.org/en/stable/io/fits/api/headers.html\#astropy.io.fits.Header}{astropy.io.fits.header.Header}%
\begin{footnote}[94]\sphinxAtStartFootnote
\sphinxnolinkurl{https://docs.astropy.org/en/stable/io/fits/api/headers.html\#astropy.io.fits.Header}
%
\end{footnote}\DUrole{p,p}{,}\DUrole{w,w}{  }\sphinxhref{https://numpy.org/doc/stable/reference/generated/numpy.ndarray.html\#numpy.ndarray}{numpy.ndarray}%
\begin{footnote}[95]\sphinxAtStartFootnote
\sphinxnolinkurl{https://numpy.org/doc/stable/reference/generated/numpy.ndarray.html\#numpy.ndarray}
%
\end{footnote}\DUrole{p,p}{,}\DUrole{w,w}{  }\sphinxhref{https://numpy.org/doc/stable/reference/generated/numpy.ndarray.html\#numpy.ndarray}{numpy.ndarray}%
\begin{footnote}[96]\sphinxAtStartFootnote
\sphinxnolinkurl{https://numpy.org/doc/stable/reference/generated/numpy.ndarray.html\#numpy.ndarray}
%
\end{footnote}\DUrole{p,p}{,}\DUrole{w,w}{  }\sphinxhref{https://numpy.org/doc/stable/reference/generated/numpy.ndarray.html\#numpy.ndarray}{numpy.ndarray}%
\begin{footnote}[97]\sphinxAtStartFootnote
\sphinxnolinkurl{https://numpy.org/doc/stable/reference/generated/numpy.ndarray.html\#numpy.ndarray}
%
\end{footnote}\DUrole{p,p}{,}\DUrole{w,w}{  }\sphinxhref{https://docs.astropy.org/en/stable/api/astropy.units.Unit.html\#astropy.units.Unit}{astropy.units.core.Unit}%
\begin{footnote}[98]\sphinxAtStartFootnote
\sphinxnolinkurl{https://docs.astropy.org/en/stable/api/astropy.units.Unit.html\#astropy.units.Unit}
%
\end{footnote}\DUrole{p,p}{,}\DUrole{w,w}{  }\sphinxhref{https://docs.astropy.org/en/stable/api/astropy.units.Unit.html\#astropy.units.Unit}{astropy.units.core.Unit}%
\begin{footnote}[99]\sphinxAtStartFootnote
\sphinxnolinkurl{https://docs.astropy.org/en/stable/api/astropy.units.Unit.html\#astropy.units.Unit}
%
\end{footnote}\DUrole{p,p}{,}\DUrole{w,w}{  }\sphinxhref{https://numpy.org/doc/stable/reference/generated/numpy.ndarray.html\#numpy.ndarray}{numpy.ndarray}%
\begin{footnote}[100]\sphinxAtStartFootnote
\sphinxnolinkurl{https://numpy.org/doc/stable/reference/generated/numpy.ndarray.html\#numpy.ndarray}
%
\end{footnote}\DUrole{p,p}{,}\DUrole{w,w}{  }\sphinxhref{https://numpy.org/doc/stable/reference/generated/numpy.ndarray.html\#numpy.ndarray}{numpy.ndarray}%
\begin{footnote}[101]\sphinxAtStartFootnote
\sphinxnolinkurl{https://numpy.org/doc/stable/reference/generated/numpy.ndarray.html\#numpy.ndarray}
%
\end{footnote}\DUrole{p,p}{{]}}}}
\pysigstopsignatures
\sphinxAtStartPar
Read in a Lezargus fits file.

\sphinxAtStartPar
This function reads in a Lezargus FITS file and parses it based on the
convention of Lezargus. See TODO for the specification. However, we do
not construct the actual classes here and instead leave that to the class
reader and writers of the container themselves so we can reuse error
reporting code there.

\sphinxAtStartPar
In general, it is advisable to use the reading and writing class
functions of the container instance you want.
\begin{quote}\begin{description}
\sphinxlineitem{Parameters}
\sphinxAtStartPar
\sphinxstyleliteralstrong{\sphinxupquote{filename}} (\sphinxstyleliteralemphasis{\sphinxupquote{str}}) – The filename of the FITS file to read.

\sphinxlineitem{Returns}
\sphinxAtStartPar
\begin{itemize}
\item {} 
\sphinxAtStartPar
\sphinxstylestrong{header} (\sphinxstyleemphasis{Header}) – The header of the Lezargus FITS file.

\item {} 
\sphinxAtStartPar
\sphinxstylestrong{wavelength} (\sphinxstyleemphasis{ndarray}) – The wavelength information of the file.

\item {} 
\sphinxAtStartPar
\sphinxstylestrong{data} (\sphinxstyleemphasis{ndarray}) – The data array of the Lezargus FITS file.

\item {} 
\sphinxAtStartPar
\sphinxstylestrong{uncertainty} (\sphinxstyleemphasis{ndarray}) – The uncertainty in the data.

\item {} 
\sphinxAtStartPar
\sphinxstylestrong{wavelength\_unit} (\sphinxstyleemphasis{Unit}) – The unit of the wavelength array.

\item {} 
\sphinxAtStartPar
\sphinxstylestrong{data\_unit} (\sphinxstyleemphasis{Unit}) – The unit of the data.

\item {} 
\sphinxAtStartPar
\sphinxstylestrong{mask} (\sphinxstyleemphasis{ndarray}) – The mask of the data.

\item {} 
\sphinxAtStartPar
\sphinxstylestrong{flags} (\sphinxstyleemphasis{ndarray}) – The noted flags for each of the data points.

\end{itemize}


\end{description}\end{quote}

\end{fulllineitems}\end{savenotes}

\index{write\_lezargus\_fits\_file() (in module lezargus.library.fits)@\spxentry{write\_lezargus\_fits\_file()}\spxextra{in module lezargus.library.fits}}

\begin{savenotes}\begin{fulllineitems}
\phantomsection\label{\detokenize{code/lezargus.library.fits:lezargus.library.fits.write_lezargus_fits_file}}
\pysigstartsignatures
\pysiglinewithargsret{\sphinxcode{\sphinxupquote{lezargus.library.fits.}}\sphinxbfcode{\sphinxupquote{write\_lezargus\_fits\_file}}}{\sphinxparam{\DUrole{n,n}{filename}\DUrole{p,p}{:}\DUrole{w,w}{  }\DUrole{n,n}{str}}, \sphinxparam{\DUrole{n,n}{header}\DUrole{p,p}{:}\DUrole{w,w}{  }\DUrole{n,n}{\sphinxhref{https://docs.astropy.org/en/stable/io/fits/api/headers.html\#astropy.io.fits.Header}{Header}%
\begin{footnote}[102]\sphinxAtStartFootnote
\sphinxnolinkurl{https://docs.astropy.org/en/stable/io/fits/api/headers.html\#astropy.io.fits.Header}
%
\end{footnote}}}, \sphinxparam{\DUrole{n,n}{wavelength}\DUrole{p,p}{:}\DUrole{w,w}{  }\DUrole{n,n}{\sphinxhref{https://numpy.org/doc/stable/reference/generated/numpy.ndarray.html\#numpy.ndarray}{ndarray}%
\begin{footnote}[103]\sphinxAtStartFootnote
\sphinxnolinkurl{https://numpy.org/doc/stable/reference/generated/numpy.ndarray.html\#numpy.ndarray}
%
\end{footnote}}}, \sphinxparam{\DUrole{n,n}{data}\DUrole{p,p}{:}\DUrole{w,w}{  }\DUrole{n,n}{\sphinxhref{https://numpy.org/doc/stable/reference/generated/numpy.ndarray.html\#numpy.ndarray}{ndarray}%
\begin{footnote}[104]\sphinxAtStartFootnote
\sphinxnolinkurl{https://numpy.org/doc/stable/reference/generated/numpy.ndarray.html\#numpy.ndarray}
%
\end{footnote}}}, \sphinxparam{\DUrole{n,n}{uncertainty}\DUrole{p,p}{:}\DUrole{w,w}{  }\DUrole{n,n}{\sphinxhref{https://numpy.org/doc/stable/reference/generated/numpy.ndarray.html\#numpy.ndarray}{ndarray}%
\begin{footnote}[105]\sphinxAtStartFootnote
\sphinxnolinkurl{https://numpy.org/doc/stable/reference/generated/numpy.ndarray.html\#numpy.ndarray}
%
\end{footnote}}}, \sphinxparam{\DUrole{n,n}{wavelength\_unit}\DUrole{p,p}{:}\DUrole{w,w}{  }\DUrole{n,n}{\sphinxhref{https://docs.astropy.org/en/stable/api/astropy.units.Unit.html\#astropy.units.Unit}{Unit}%
\begin{footnote}[106]\sphinxAtStartFootnote
\sphinxnolinkurl{https://docs.astropy.org/en/stable/api/astropy.units.Unit.html\#astropy.units.Unit}
%
\end{footnote}}}, \sphinxparam{\DUrole{n,n}{data\_unit}\DUrole{p,p}{:}\DUrole{w,w}{  }\DUrole{n,n}{\sphinxhref{https://docs.astropy.org/en/stable/api/astropy.units.Unit.html\#astropy.units.Unit}{Unit}%
\begin{footnote}[107]\sphinxAtStartFootnote
\sphinxnolinkurl{https://docs.astropy.org/en/stable/api/astropy.units.Unit.html\#astropy.units.Unit}
%
\end{footnote}}}, \sphinxparam{\DUrole{n,n}{uncertainty\_unit}\DUrole{p,p}{:}\DUrole{w,w}{  }\DUrole{n,n}{\sphinxhref{https://docs.astropy.org/en/stable/api/astropy.units.Unit.html\#astropy.units.Unit}{Unit}%
\begin{footnote}[108]\sphinxAtStartFootnote
\sphinxnolinkurl{https://docs.astropy.org/en/stable/api/astropy.units.Unit.html\#astropy.units.Unit}
%
\end{footnote}}}, \sphinxparam{\DUrole{n,n}{mask}\DUrole{p,p}{:}\DUrole{w,w}{  }\DUrole{n,n}{\sphinxhref{https://numpy.org/doc/stable/reference/generated/numpy.ndarray.html\#numpy.ndarray}{ndarray}%
\begin{footnote}[109]\sphinxAtStartFootnote
\sphinxnolinkurl{https://numpy.org/doc/stable/reference/generated/numpy.ndarray.html\#numpy.ndarray}
%
\end{footnote}}}, \sphinxparam{\DUrole{n,n}{flags}\DUrole{p,p}{:}\DUrole{w,w}{  }\DUrole{n,n}{\sphinxhref{https://numpy.org/doc/stable/reference/generated/numpy.ndarray.html\#numpy.ndarray}{ndarray}%
\begin{footnote}[110]\sphinxAtStartFootnote
\sphinxnolinkurl{https://numpy.org/doc/stable/reference/generated/numpy.ndarray.html\#numpy.ndarray}
%
\end{footnote}}}, \sphinxparam{\DUrole{n,n}{overwrite}\DUrole{p,p}{:}\DUrole{w,w}{  }\DUrole{n,n}{bool}\DUrole{w,w}{  }\DUrole{o,o}{=}\DUrole{w,w}{  }\DUrole{default_value}{False}}}{{ $\rightarrow$ None}}
\pysigstopsignatures
\sphinxAtStartPar
Write to a Lezargus fits file.

\sphinxAtStartPar
This function reads in a Lezargus FITS file and parses it based on the
convention of Lezargus. See TODO for the specification. However, we do
not construct the actual classes here and instead leave that to the class
reader and writers of the container themselves so we can reuse error
reporting code there.

\sphinxAtStartPar
In general, it is advisable to use the reading and writing class
functions of the container instance you want.
\begin{quote}\begin{description}
\sphinxlineitem{Parameters}\begin{itemize}
\item {} 
\sphinxAtStartPar
\sphinxstyleliteralstrong{\sphinxupquote{filename}} (\sphinxstyleliteralemphasis{\sphinxupquote{str}}) – The filename of the FITS file to write to.

\item {} 
\sphinxAtStartPar
\sphinxstyleliteralstrong{\sphinxupquote{header}} (\sphinxstyleliteralemphasis{\sphinxupquote{Header}}) – The header of the Lezargus FITS file.

\item {} 
\sphinxAtStartPar
\sphinxstyleliteralstrong{\sphinxupquote{wavelength}} (\sphinxstyleliteralemphasis{\sphinxupquote{ndarray}}) – The wavelength information of the file.

\item {} 
\sphinxAtStartPar
\sphinxstyleliteralstrong{\sphinxupquote{data}} (\sphinxstyleliteralemphasis{\sphinxupquote{ndarray}}) – The data array of the Lezargus FITS file.

\item {} 
\sphinxAtStartPar
\sphinxstyleliteralstrong{\sphinxupquote{uncertainty}} (\sphinxstyleliteralemphasis{\sphinxupquote{ndarray}}) – The uncertainty in the data.

\item {} 
\sphinxAtStartPar
\sphinxstyleliteralstrong{\sphinxupquote{wavelength\_unit}} (\sphinxstyleliteralemphasis{\sphinxupquote{Unit}}) – The unit of the wavelength array.

\item {} 
\sphinxAtStartPar
\sphinxstyleliteralstrong{\sphinxupquote{data\_unit}} (\sphinxstyleliteralemphasis{\sphinxupquote{Unit}}) – The unit of the data.

\item {} 
\sphinxAtStartPar
\sphinxstyleliteralstrong{\sphinxupquote{uncertainty\_unit}} (\sphinxstyleliteralemphasis{\sphinxupquote{Unit}}) – The unit of the uncertainty of the data.

\item {} 
\sphinxAtStartPar
\sphinxstyleliteralstrong{\sphinxupquote{mask}} (\sphinxstyleliteralemphasis{\sphinxupquote{ndarray}}) – The mask of the data.

\item {} 
\sphinxAtStartPar
\sphinxstyleliteralstrong{\sphinxupquote{flags}} (\sphinxstyleliteralemphasis{\sphinxupquote{ndarray}}) – The noted flags for each of the data points.

\item {} 
\sphinxAtStartPar
\sphinxstyleliteralstrong{\sphinxupquote{overwrite}} (\sphinxstyleliteralemphasis{\sphinxupquote{bool}}\sphinxstyleliteralemphasis{\sphinxupquote{, }}\sphinxstyleliteralemphasis{\sphinxupquote{default = False}}) – If True, overwrite the file upon conflicts.

\end{itemize}

\sphinxlineitem{Return type}
\sphinxAtStartPar
None

\end{description}\end{quote}

\end{fulllineitems}\end{savenotes}


\sphinxstepscope


\subparagraph{lezargus.library.flags module}
\label{\detokenize{code/lezargus.library.flags:module-lezargus.library.flags}}\label{\detokenize{code/lezargus.library.flags:lezargus-library-flags-module}}\label{\detokenize{code/lezargus.library.flags::doc}}\index{module@\spxentry{module}!lezargus.library.flags@\spxentry{lezargus.library.flags}}\index{lezargus.library.flags@\spxentry{lezargus.library.flags}!module@\spxentry{module}}
\sphinxAtStartPar
Functions to deal with quality flag tracking, masks, and other information.

\sphinxAtStartPar
Flags and masks is the primary backbone of how the quality of data can be
communicated. Here, we package all of the different functions regarding
flags and masks.
\index{combine\_flags() (in module lezargus.library.flags)@\spxentry{combine\_flags()}\spxextra{in module lezargus.library.flags}}

\begin{savenotes}\begin{fulllineitems}
\phantomsection\label{\detokenize{code/lezargus.library.flags:lezargus.library.flags.combine_flags}}
\pysigstartsignatures
\pysiglinewithargsret{\sphinxcode{\sphinxupquote{lezargus.library.flags.}}\sphinxbfcode{\sphinxupquote{combine\_flags}}}{\sphinxparam{\DUrole{o,o}{*}\DUrole{n,n}{flags}\DUrole{p,p}{:}\DUrole{w,w}{  }\DUrole{n,n}{\sphinxhref{https://numpy.org/doc/stable/reference/generated/numpy.ndarray.html\#numpy.ndarray}{ndarray}%
\begin{footnote}[111]\sphinxAtStartFootnote
\sphinxnolinkurl{https://numpy.org/doc/stable/reference/generated/numpy.ndarray.html\#numpy.ndarray}
%
\end{footnote}}}}{{ $\rightarrow$ \sphinxhref{https://numpy.org/doc/stable/reference/generated/numpy.ndarray.html\#numpy.ndarray}{ndarray}%
\begin{footnote}[112]\sphinxAtStartFootnote
\sphinxnolinkurl{https://numpy.org/doc/stable/reference/generated/numpy.ndarray.html\#numpy.ndarray}
%
\end{footnote}}}
\pysigstopsignatures
\sphinxAtStartPar
Combine two or more flag arrays.

\sphinxAtStartPar
The flag values here follow the Lezargus convention, see {[}{[}TODO{]}{]}.
\begin{quote}\begin{description}
\sphinxlineitem{Parameters}
\sphinxAtStartPar
\sphinxstyleliteralstrong{\sphinxupquote{*flags}} (\sphinxstyleliteralemphasis{\sphinxupquote{ndarray}}) – The set of flags to combine.

\sphinxlineitem{Returns}
\sphinxAtStartPar
\sphinxstylestrong{combined\_flags} – The combined mask.

\sphinxlineitem{Return type}
\sphinxAtStartPar
ndarray

\end{description}\end{quote}

\end{fulllineitems}\end{savenotes}

\index{combine\_masks() (in module lezargus.library.flags)@\spxentry{combine\_masks()}\spxextra{in module lezargus.library.flags}}

\begin{savenotes}\begin{fulllineitems}
\phantomsection\label{\detokenize{code/lezargus.library.flags:lezargus.library.flags.combine_masks}}
\pysigstartsignatures
\pysiglinewithargsret{\sphinxcode{\sphinxupquote{lezargus.library.flags.}}\sphinxbfcode{\sphinxupquote{combine\_masks}}}{\sphinxparam{\DUrole{o,o}{*}\DUrole{n,n}{masks}\DUrole{p,p}{:}\DUrole{w,w}{  }\DUrole{n,n}{\sphinxhref{https://numpy.org/doc/stable/reference/generated/numpy.ndarray.html\#numpy.ndarray}{ndarray}%
\begin{footnote}[113]\sphinxAtStartFootnote
\sphinxnolinkurl{https://numpy.org/doc/stable/reference/generated/numpy.ndarray.html\#numpy.ndarray}
%
\end{footnote}}}}{{ $\rightarrow$ \sphinxhref{https://numpy.org/doc/stable/reference/generated/numpy.ndarray.html\#numpy.ndarray}{ndarray}%
\begin{footnote}[114]\sphinxAtStartFootnote
\sphinxnolinkurl{https://numpy.org/doc/stable/reference/generated/numpy.ndarray.html\#numpy.ndarray}
%
\end{footnote}}}
\pysigstopsignatures
\sphinxAtStartPar
Combine two or more masks.

\sphinxAtStartPar
The masks follow the Numpy convention; a True value means that the data
is considered masked.
\begin{quote}\begin{description}
\sphinxlineitem{Parameters}
\sphinxAtStartPar
\sphinxstyleliteralstrong{\sphinxupquote{*masks}} (\sphinxstyleliteralemphasis{\sphinxupquote{ndarray}}) – The set of masks to combine.

\sphinxlineitem{Returns}
\sphinxAtStartPar
\sphinxstylestrong{combined\_mask} – The combined mask.

\sphinxlineitem{Return type}
\sphinxAtStartPar
ndarray

\end{description}\end{quote}

\end{fulllineitems}\end{savenotes}

\index{reduce\_flags() (in module lezargus.library.flags)@\spxentry{reduce\_flags()}\spxextra{in module lezargus.library.flags}}

\begin{savenotes}\begin{fulllineitems}
\phantomsection\label{\detokenize{code/lezargus.library.flags:lezargus.library.flags.reduce_flags}}
\pysigstartsignatures
\pysiglinewithargsret{\sphinxcode{\sphinxupquote{lezargus.library.flags.}}\sphinxbfcode{\sphinxupquote{reduce\_flags}}}{\sphinxparam{\DUrole{n,n}{flag\_array}\DUrole{p,p}{:}\DUrole{w,w}{  }\DUrole{n,n}{\sphinxhref{https://numpy.org/doc/stable/reference/generated/numpy.ndarray.html\#numpy.ndarray}{ndarray}%
\begin{footnote}[115]\sphinxAtStartFootnote
\sphinxnolinkurl{https://numpy.org/doc/stable/reference/generated/numpy.ndarray.html\#numpy.ndarray}
%
\end{footnote}}}}{{ $\rightarrow$ \sphinxhref{https://numpy.org/doc/stable/reference/generated/numpy.ndarray.html\#numpy.ndarray}{ndarray}%
\begin{footnote}[116]\sphinxAtStartFootnote
\sphinxnolinkurl{https://numpy.org/doc/stable/reference/generated/numpy.ndarray.html\#numpy.ndarray}
%
\end{footnote}}}
\pysigstopsignatures
\sphinxAtStartPar
Reduce the flag value to the minimum it can be.

\sphinxAtStartPar
Flags, based on the Lezargus convention (see {[}{[}TODO{]}{]}),
rely on the prime factors to determine the total flags present. As
multiplication is how flags propagate, the value can get big quickly.
We reduce the values within a flag array to the lowest it can be.
\begin{quote}\begin{description}
\sphinxlineitem{Parameters}
\sphinxAtStartPar
\sphinxstyleliteralstrong{\sphinxupquote{flag\_array}} (\sphinxstyleliteralemphasis{\sphinxupquote{ndarray}}) – The flag array to be reduced into its lowest form.

\sphinxlineitem{Returns}
\sphinxAtStartPar
\sphinxstylestrong{lowest\_flag\_array} – The flags, reduced to the lowest value.

\sphinxlineitem{Return type}
\sphinxAtStartPar
ndarray

\end{description}\end{quote}

\end{fulllineitems}\end{savenotes}


\sphinxstepscope


\subparagraph{lezargus.library.hint module}
\label{\detokenize{code/lezargus.library.hint:module-lezargus.library.hint}}\label{\detokenize{code/lezargus.library.hint:lezargus-library-hint-module}}\label{\detokenize{code/lezargus.library.hint::doc}}\index{module@\spxentry{module}!lezargus.library.hint@\spxentry{lezargus.library.hint}}\index{lezargus.library.hint@\spxentry{lezargus.library.hint}!module@\spxentry{module}}
\sphinxAtStartPar
A collection of types for linting.

\sphinxAtStartPar
These are redefinitions and wrapping variables for type hints. Its purpose
is for just uniform hinting types.

\sphinxAtStartPar
This should only be used for types which are otherwise not native and would
require an import, including the typing module. The whole point of this is to
be a central collection of types for the purpose of type hinting.

\sphinxAtStartPar
This module should never be used for anything other than hinting. Use proper
imports to access these classes. Otherwise, you will likely get circular
imports and other nasty things.

\sphinxstepscope


\subparagraph{lezargus.library.logging module}
\label{\detokenize{code/lezargus.library.logging:module-lezargus.library.logging}}\label{\detokenize{code/lezargus.library.logging:lezargus-library-logging-module}}\label{\detokenize{code/lezargus.library.logging::doc}}\index{module@\spxentry{module}!lezargus.library.logging@\spxentry{lezargus.library.logging}}\index{lezargus.library.logging@\spxentry{lezargus.library.logging}!module@\spxentry{module}}
\sphinxAtStartPar
Error, warning, and logging functionality pertinent to Lezargus.

\sphinxAtStartPar
Use the functions here when logging or issuing errors or other information.
\index{AccuracyError@\spxentry{AccuracyError}}

\begin{savenotes}\begin{fulllineitems}
\phantomsection\label{\detokenize{code/lezargus.library.logging:lezargus.library.logging.AccuracyError}}
\pysigstartsignatures
\pysigline{\sphinxbfcode{\sphinxupquote{exception\DUrole{w,w}{  }}}\sphinxcode{\sphinxupquote{lezargus.library.logging.}}\sphinxbfcode{\sphinxupquote{AccuracyError}}}
\pysigstopsignatures
\sphinxAtStartPar
Bases: {\hyperref[\detokenize{code/lezargus.library.logging:lezargus.library.logging.LezargusError}]{\sphinxcrossref{\sphinxcode{\sphinxupquote{LezargusError}}}}}

\sphinxAtStartPar
An error for inaccurate results.

\sphinxAtStartPar
This error is used when some elements of the simulation or data
reduction would yield very undesirable results. This class should only
be used for cases where the results would be quite wrong for most
cases; say, much greater +/\sphinxhyphen{} 10\%. This is just a rule of thumb. For
accuracy issues much tamer and closer to the rule of thumb, use
AccuracyWarning instead.

\end{fulllineitems}\end{savenotes}

\index{AccuracyWarning@\spxentry{AccuracyWarning}}

\begin{savenotes}\begin{fulllineitems}
\phantomsection\label{\detokenize{code/lezargus.library.logging:lezargus.library.logging.AccuracyWarning}}
\pysigstartsignatures
\pysigline{\sphinxbfcode{\sphinxupquote{exception\DUrole{w,w}{  }}}\sphinxcode{\sphinxupquote{lezargus.library.logging.}}\sphinxbfcode{\sphinxupquote{AccuracyWarning}}}
\pysigstopsignatures
\sphinxAtStartPar
Bases: {\hyperref[\detokenize{code/lezargus.library.logging:lezargus.library.logging.LezargusWarning}]{\sphinxcrossref{\sphinxcode{\sphinxupquote{LezargusWarning}}}}}

\sphinxAtStartPar
A warning for inaccurate results.

\sphinxAtStartPar
This warning is used when some elements of the simulation or data
reduction would yield less than desireable results. This class should only
be used for cases where the results would be slightly wrong for most
cases; say, on the order of +/\sphinxhyphen{} 10\%. This is just a rule of thumb. For
accuracy issues much larger, use AccuracyError instead.

\end{fulllineitems}\end{savenotes}

\index{AlgorithmWarning@\spxentry{AlgorithmWarning}}

\begin{savenotes}\begin{fulllineitems}
\phantomsection\label{\detokenize{code/lezargus.library.logging:lezargus.library.logging.AlgorithmWarning}}
\pysigstartsignatures
\pysigline{\sphinxbfcode{\sphinxupquote{exception\DUrole{w,w}{  }}}\sphinxcode{\sphinxupquote{lezargus.library.logging.}}\sphinxbfcode{\sphinxupquote{AlgorithmWarning}}}
\pysigstopsignatures
\sphinxAtStartPar
Bases: {\hyperref[\detokenize{code/lezargus.library.logging:lezargus.library.logging.LezargusWarning}]{\sphinxcrossref{\sphinxcode{\sphinxupquote{LezargusWarning}}}}}

\sphinxAtStartPar
A warning for issues with algorithms or methods.

\sphinxAtStartPar
This warning should be used when something went wrong with an algorithm.
Examples include when continuing would lead to inaccurate results, or
when alternative methods must be used, or when predicted execution time
would be slow. This warning should be used in conjunction with other
warnings to give a full picture of the issue.

\end{fulllineitems}\end{savenotes}

\index{ArithmeticalError@\spxentry{ArithmeticalError}}

\begin{savenotes}\begin{fulllineitems}
\phantomsection\label{\detokenize{code/lezargus.library.logging:lezargus.library.logging.ArithmeticalError}}
\pysigstartsignatures
\pysigline{\sphinxbfcode{\sphinxupquote{exception\DUrole{w,w}{  }}}\sphinxcode{\sphinxupquote{lezargus.library.logging.}}\sphinxbfcode{\sphinxupquote{ArithmeticalError}}}
\pysigstopsignatures
\sphinxAtStartPar
Bases: {\hyperref[\detokenize{code/lezargus.library.logging:lezargus.library.logging.LezargusError}]{\sphinxcrossref{\sphinxcode{\sphinxupquote{LezargusError}}}}}

\sphinxAtStartPar
An error to be used when undefined arithmetic operations are attempted.

\sphinxAtStartPar
This error is to be used when any arithmetic functions are being attempted
which do not have a definition. The most common use case for this error is
doing operations on incompatible Lezargus containers. Note, it is named
ArithmeticalError to avoid a name conflict with the built\sphinxhyphen{}in
ArithmeticError.

\end{fulllineitems}\end{savenotes}

\index{ColoredLogFormatter (class in lezargus.library.logging)@\spxentry{ColoredLogFormatter}\spxextra{class in lezargus.library.logging}}

\begin{savenotes}\begin{fulllineitems}
\phantomsection\label{\detokenize{code/lezargus.library.logging:lezargus.library.logging.ColoredLogFormatter}}
\pysigstartsignatures
\pysiglinewithargsret{\sphinxbfcode{\sphinxupquote{class\DUrole{w,w}{  }}}\sphinxcode{\sphinxupquote{lezargus.library.logging.}}\sphinxbfcode{\sphinxupquote{ColoredLogFormatter}}}{\sphinxparam{\DUrole{n,n}{message\_format: str}}, \sphinxparam{\DUrole{n,n}{date\_format: str}}, \sphinxparam{\DUrole{n,n}{color\_hex\_dict: dict{[}slice(<class 'int'>}}, \sphinxparam{\DUrole{n,n}{<class 'str'>}}, \sphinxparam{\DUrole{n,n}{None){]} | None = None}}}{}
\pysigstopsignatures
\sphinxAtStartPar
Bases: \sphinxcode{\sphinxupquote{Formatter}}

\sphinxAtStartPar
Use this formatter to have colors.
\index{message\_format (lezargus.library.logging.ColoredLogFormatter attribute)@\spxentry{message\_format}\spxextra{lezargus.library.logging.ColoredLogFormatter attribute}}

\begin{savenotes}\begin{fulllineitems}
\phantomsection\label{\detokenize{code/lezargus.library.logging:lezargus.library.logging.ColoredLogFormatter.message_format}}
\pysigstartsignatures
\pysigline{\sphinxbfcode{\sphinxupquote{message\_format}}}
\pysigstopsignatures
\sphinxAtStartPar
The message format, passed directly to the logger formatter after
the color keys are added.
\begin{quote}\begin{description}
\sphinxlineitem{Type}
\sphinxAtStartPar
str

\end{description}\end{quote}

\end{fulllineitems}\end{savenotes}

\index{date\_format (lezargus.library.logging.ColoredLogFormatter attribute)@\spxentry{date\_format}\spxextra{lezargus.library.logging.ColoredLogFormatter attribute}}

\begin{savenotes}\begin{fulllineitems}
\phantomsection\label{\detokenize{code/lezargus.library.logging:lezargus.library.logging.ColoredLogFormatter.date_format}}
\pysigstartsignatures
\pysigline{\sphinxbfcode{\sphinxupquote{date\_format}}}
\pysigstopsignatures
\sphinxAtStartPar
The date format, passed directly to the logger formatter.
\begin{quote}\begin{description}
\sphinxlineitem{Type}
\sphinxAtStartPar
str

\end{description}\end{quote}

\end{fulllineitems}\end{savenotes}

\index{color\_formatting (lezargus.library.logging.ColoredLogFormatter attribute)@\spxentry{color\_formatting}\spxextra{lezargus.library.logging.ColoredLogFormatter attribute}}

\begin{savenotes}\begin{fulllineitems}
\phantomsection\label{\detokenize{code/lezargus.library.logging:lezargus.library.logging.ColoredLogFormatter.color_formatting}}
\pysigstartsignatures
\pysigline{\sphinxbfcode{\sphinxupquote{color\_formatting}}}
\pysigstopsignatures
\sphinxAtStartPar
The formatting for the color.
\begin{quote}\begin{description}
\sphinxlineitem{Type}
\sphinxAtStartPar
dict

\end{description}\end{quote}

\end{fulllineitems}\end{savenotes}

\index{\_\_convert\_color\_hex\_to\_ansi\_escape() (lezargus.library.logging.ColoredLogFormatter method)@\spxentry{\_\_convert\_color\_hex\_to\_ansi\_escape()}\spxextra{lezargus.library.logging.ColoredLogFormatter method}}

\begin{savenotes}\begin{fulllineitems}
\phantomsection\label{\detokenize{code/lezargus.library.logging:lezargus.library.logging.ColoredLogFormatter.__convert_color_hex_to_ansi_escape}}
\pysigstartsignatures
\pysiglinewithargsret{\sphinxbfcode{\sphinxupquote{\_\_convert\_color\_hex\_to\_ansi\_escape}}}{}{{ $\rightarrow$ str}}
\pysigstopsignatures
\sphinxAtStartPar
Convert a hex code to a ANSI escape code.
\begin{quote}\begin{description}
\sphinxlineitem{Parameters}
\sphinxAtStartPar
\sphinxstyleliteralstrong{\sphinxupquote{color\_hex}} (\sphinxstyleliteralemphasis{\sphinxupquote{str}}) – The HEX code of the color, including the \# symbol.

\sphinxlineitem{Returns}
\sphinxAtStartPar
\sphinxstylestrong{color\_ansi\_escape} – The ANSI escape code for the color.

\sphinxlineitem{Return type}
\sphinxAtStartPar
str

\end{description}\end{quote}

\end{fulllineitems}\end{savenotes}

\index{\_\_init\_\_() (lezargus.library.logging.ColoredLogFormatter method)@\spxentry{\_\_init\_\_()}\spxextra{lezargus.library.logging.ColoredLogFormatter method}}

\begin{savenotes}\begin{fulllineitems}
\phantomsection\label{\detokenize{code/lezargus.library.logging:lezargus.library.logging.ColoredLogFormatter.__init__}}
\pysigstartsignatures
\pysiglinewithargsret{\sphinxbfcode{\sphinxupquote{\_\_init\_\_}}}{\sphinxparam{\DUrole{n,n}{message\_format: str}}, \sphinxparam{\DUrole{n,n}{date\_format: str}}, \sphinxparam{\DUrole{n,n}{color\_hex\_dict: dict{[}slice(<class 'int'>}}, \sphinxparam{\DUrole{n,n}{<class 'str'>}}, \sphinxparam{\DUrole{n,n}{None){]} | None = None}}}{{ $\rightarrow$ None}}
\pysigstopsignatures
\sphinxAtStartPar
Initialize the color formatter.
\begin{quote}\begin{description}
\sphinxlineitem{Parameters}\begin{itemize}
\item {} 
\sphinxAtStartPar
\sphinxstyleliteralstrong{\sphinxupquote{message\_format}} (\sphinxstyleliteralemphasis{\sphinxupquote{str}}) – The message format, passed directly to the logger formatter after
the color keys are added.

\item {} 
\sphinxAtStartPar
\sphinxstyleliteralstrong{\sphinxupquote{date\_format}} (\sphinxstyleliteralemphasis{\sphinxupquote{str}}) – The date format, passed directly to the logger formatter.

\item {} 
\sphinxAtStartPar
\sphinxstyleliteralstrong{\sphinxupquote{color\_hex\_dict}} (\sphinxstyleliteralemphasis{\sphinxupquote{dict}}\sphinxstyleliteralemphasis{\sphinxupquote{, }}\sphinxstyleliteralemphasis{\sphinxupquote{default = None}}) – The dictionary containing the color pairings between logging
levels and its actual color. It should be a
\{level\_number:hex\_color\} dictionary.

\end{itemize}

\sphinxlineitem{Return type}
\sphinxAtStartPar
None

\end{description}\end{quote}

\end{fulllineitems}\end{savenotes}

\index{format() (lezargus.library.logging.ColoredLogFormatter method)@\spxentry{format()}\spxextra{lezargus.library.logging.ColoredLogFormatter method}}

\begin{savenotes}\begin{fulllineitems}
\phantomsection\label{\detokenize{code/lezargus.library.logging:lezargus.library.logging.ColoredLogFormatter.format}}
\pysigstartsignatures
\pysiglinewithargsret{\sphinxbfcode{\sphinxupquote{format}}}{\sphinxparam{\DUrole{n,n}{record}\DUrole{p,p}{:}\DUrole{w,w}{  }\DUrole{n,n}{LogRecord}}}{{ $\rightarrow$ str}}
\pysigstopsignatures
\sphinxAtStartPar
Format a log record.

\sphinxAtStartPar
The name of this function cannot be helped as it is required for the
Python logging module.
\begin{quote}\begin{description}
\sphinxlineitem{Parameters}
\sphinxAtStartPar
\sphinxstyleliteralstrong{\sphinxupquote{record}} (\sphinxstyleliteralemphasis{\sphinxupquote{LogRecord}}) – The record to format.

\sphinxlineitem{Returns}
\sphinxAtStartPar
\sphinxstylestrong{formatted\_record} – The formatted string.

\sphinxlineitem{Return type}
\sphinxAtStartPar
str

\end{description}\end{quote}

\end{fulllineitems}\end{savenotes}


\end{fulllineitems}\end{savenotes}

\index{CommandLineError@\spxentry{CommandLineError}}

\begin{savenotes}\begin{fulllineitems}
\phantomsection\label{\detokenize{code/lezargus.library.logging:lezargus.library.logging.CommandLineError}}
\pysigstartsignatures
\pysigline{\sphinxbfcode{\sphinxupquote{exception\DUrole{w,w}{  }}}\sphinxcode{\sphinxupquote{lezargus.library.logging.}}\sphinxbfcode{\sphinxupquote{CommandLineError}}}
\pysigstopsignatures
\sphinxAtStartPar
Bases: {\hyperref[\detokenize{code/lezargus.library.logging:lezargus.library.logging.LezargusError}]{\sphinxcrossref{\sphinxcode{\sphinxupquote{LezargusError}}}}}

\sphinxAtStartPar
An error used for an error with the command\sphinxhyphen{}line.

\sphinxAtStartPar
This error is used when the entered command\sphinxhyphen{}line command or its options
are not correct.

\end{fulllineitems}\end{savenotes}

\index{ConfigurationError@\spxentry{ConfigurationError}}

\begin{savenotes}\begin{fulllineitems}
\phantomsection\label{\detokenize{code/lezargus.library.logging:lezargus.library.logging.ConfigurationError}}
\pysigstartsignatures
\pysigline{\sphinxbfcode{\sphinxupquote{exception\DUrole{w,w}{  }}}\sphinxcode{\sphinxupquote{lezargus.library.logging.}}\sphinxbfcode{\sphinxupquote{ConfigurationError}}}
\pysigstopsignatures
\sphinxAtStartPar
Bases: {\hyperref[\detokenize{code/lezargus.library.logging:lezargus.library.logging.LezargusError}]{\sphinxcrossref{\sphinxcode{\sphinxupquote{LezargusError}}}}}

\sphinxAtStartPar
An error used for an error with the configuration file.

\sphinxAtStartPar
This error is to be used when the configuration file is wrong. There is a
specific expectation for how configuration files and configuration
parameters are structures are defined.

\end{fulllineitems}\end{savenotes}

\index{ConfigurationWarning@\spxentry{ConfigurationWarning}}

\begin{savenotes}\begin{fulllineitems}
\phantomsection\label{\detokenize{code/lezargus.library.logging:lezargus.library.logging.ConfigurationWarning}}
\pysigstartsignatures
\pysigline{\sphinxbfcode{\sphinxupquote{exception\DUrole{w,w}{  }}}\sphinxcode{\sphinxupquote{lezargus.library.logging.}}\sphinxbfcode{\sphinxupquote{ConfigurationWarning}}}
\pysigstopsignatures
\sphinxAtStartPar
Bases: {\hyperref[\detokenize{code/lezargus.library.logging:lezargus.library.logging.LezargusWarning}]{\sphinxcrossref{\sphinxcode{\sphinxupquote{LezargusWarning}}}}}

\sphinxAtStartPar
A warning for inappropriate configurations.

\sphinxAtStartPar
This warning is to be used when the configuration file is wrong. There is a
specific expectation for how configuration files and configuration
parameters are structures are defined.

\end{fulllineitems}\end{savenotes}

\index{DataLossWarning@\spxentry{DataLossWarning}}

\begin{savenotes}\begin{fulllineitems}
\phantomsection\label{\detokenize{code/lezargus.library.logging:lezargus.library.logging.DataLossWarning}}
\pysigstartsignatures
\pysigline{\sphinxbfcode{\sphinxupquote{exception\DUrole{w,w}{  }}}\sphinxcode{\sphinxupquote{lezargus.library.logging.}}\sphinxbfcode{\sphinxupquote{DataLossWarning}}}
\pysigstopsignatures
\sphinxAtStartPar
Bases: {\hyperref[\detokenize{code/lezargus.library.logging:lezargus.library.logging.LezargusWarning}]{\sphinxcrossref{\sphinxcode{\sphinxupquote{LezargusWarning}}}}}

\sphinxAtStartPar
A warning to caution on data loss.

\sphinxAtStartPar
This warning is used when something is being done which might result in
a loss of important data, for example, because a file is not saved or
only part of a data file is read.

\end{fulllineitems}\end{savenotes}

\index{DevelopmentError@\spxentry{DevelopmentError}}

\begin{savenotes}\begin{fulllineitems}
\phantomsection\label{\detokenize{code/lezargus.library.logging:lezargus.library.logging.DevelopmentError}}
\pysigstartsignatures
\pysigline{\sphinxbfcode{\sphinxupquote{exception\DUrole{w,w}{  }}}\sphinxcode{\sphinxupquote{lezargus.library.logging.}}\sphinxbfcode{\sphinxupquote{DevelopmentError}}}
\pysigstopsignatures
\sphinxAtStartPar
Bases: {\hyperref[\detokenize{code/lezargus.library.logging:lezargus.library.logging.LezargusBaseError}]{\sphinxcrossref{\sphinxcode{\sphinxupquote{LezargusBaseError}}}}}

\sphinxAtStartPar
An error used for a development error.

\sphinxAtStartPar
This is an error where the development of Lezargus is not correct and
something is not coded based on the expectations of the software itself.
This is not the fault of the user.

\end{fulllineitems}\end{savenotes}

\index{DevelopmentWarning@\spxentry{DevelopmentWarning}}

\begin{savenotes}\begin{fulllineitems}
\phantomsection\label{\detokenize{code/lezargus.library.logging:lezargus.library.logging.DevelopmentWarning}}
\pysigstartsignatures
\pysigline{\sphinxbfcode{\sphinxupquote{exception\DUrole{w,w}{  }}}\sphinxcode{\sphinxupquote{lezargus.library.logging.}}\sphinxbfcode{\sphinxupquote{DevelopmentWarning}}}
\pysigstopsignatures
\sphinxAtStartPar
Bases: {\hyperref[\detokenize{code/lezargus.library.logging:lezargus.library.logging.LezargusWarning}]{\sphinxcrossref{\sphinxcode{\sphinxupquote{LezargusWarning}}}}}

\sphinxAtStartPar
A warning used for a development issue.

\sphinxAtStartPar
This is a warning where the development of Lezargus is not correct and
something is not coded based on the expectations of the software itself.
This is not the fault of the user.

\end{fulllineitems}\end{savenotes}

\index{DirectoryError@\spxentry{DirectoryError}}

\begin{savenotes}\begin{fulllineitems}
\phantomsection\label{\detokenize{code/lezargus.library.logging:lezargus.library.logging.DirectoryError}}
\pysigstartsignatures
\pysigline{\sphinxbfcode{\sphinxupquote{exception\DUrole{w,w}{  }}}\sphinxcode{\sphinxupquote{lezargus.library.logging.}}\sphinxbfcode{\sphinxupquote{DirectoryError}}}
\pysigstopsignatures
\sphinxAtStartPar
Bases: {\hyperref[\detokenize{code/lezargus.library.logging:lezargus.library.logging.LezargusError}]{\sphinxcrossref{\sphinxcode{\sphinxupquote{LezargusError}}}}}

\sphinxAtStartPar
An error used for directory issues.

\sphinxAtStartPar
If there are issues with directories, use this error.

\end{fulllineitems}\end{savenotes}

\index{ElevatedError@\spxentry{ElevatedError}}

\begin{savenotes}\begin{fulllineitems}
\phantomsection\label{\detokenize{code/lezargus.library.logging:lezargus.library.logging.ElevatedError}}
\pysigstartsignatures
\pysigline{\sphinxbfcode{\sphinxupquote{exception\DUrole{w,w}{  }}}\sphinxcode{\sphinxupquote{lezargus.library.logging.}}\sphinxbfcode{\sphinxupquote{ElevatedError}}}
\pysigstopsignatures
\sphinxAtStartPar
Bases: {\hyperref[\detokenize{code/lezargus.library.logging:lezargus.library.logging.LezargusError}]{\sphinxcrossref{\sphinxcode{\sphinxupquote{LezargusError}}}}}

\sphinxAtStartPar
An error used when elevating warnings or errors to critical level.

\sphinxAtStartPar
Only to be used when elevating via the configuration property.

\end{fulllineitems}\end{savenotes}

\index{ExpectedCaughtError@\spxentry{ExpectedCaughtError}}

\begin{savenotes}\begin{fulllineitems}
\phantomsection\label{\detokenize{code/lezargus.library.logging:lezargus.library.logging.ExpectedCaughtError}}
\pysigstartsignatures
\pysigline{\sphinxbfcode{\sphinxupquote{exception\DUrole{w,w}{  }}}\sphinxcode{\sphinxupquote{lezargus.library.logging.}}\sphinxbfcode{\sphinxupquote{ExpectedCaughtError}}}
\pysigstopsignatures
\sphinxAtStartPar
Bases: {\hyperref[\detokenize{code/lezargus.library.logging:lezargus.library.logging.LezargusBaseError}]{\sphinxcrossref{\sphinxcode{\sphinxupquote{LezargusBaseError}}}}}

\sphinxAtStartPar
An error used when raising an error to be caught later is needed.

\sphinxAtStartPar
This error should only be used when an error is needed to be raised which
will be caught later. The user should not see this error at all as
any time it is used, it should be caught. This name also conveniently
provides an obvious and unique error name.

\end{fulllineitems}\end{savenotes}

\index{FileError@\spxentry{FileError}}

\begin{savenotes}\begin{fulllineitems}
\phantomsection\label{\detokenize{code/lezargus.library.logging:lezargus.library.logging.FileError}}
\pysigstartsignatures
\pysigline{\sphinxbfcode{\sphinxupquote{exception\DUrole{w,w}{  }}}\sphinxcode{\sphinxupquote{lezargus.library.logging.}}\sphinxbfcode{\sphinxupquote{FileError}}}
\pysigstopsignatures
\sphinxAtStartPar
Bases: {\hyperref[\detokenize{code/lezargus.library.logging:lezargus.library.logging.LezargusError}]{\sphinxcrossref{\sphinxcode{\sphinxupquote{LezargusError}}}}}

\sphinxAtStartPar
An error used for file issues.

\sphinxAtStartPar
If there are issues with files, use this error. This error should not be
used in cases where the problem is because of an incorrect format of the
file (other than corruption).

\end{fulllineitems}\end{savenotes}

\index{FileWarning@\spxentry{FileWarning}}

\begin{savenotes}\begin{fulllineitems}
\phantomsection\label{\detokenize{code/lezargus.library.logging:lezargus.library.logging.FileWarning}}
\pysigstartsignatures
\pysigline{\sphinxbfcode{\sphinxupquote{exception\DUrole{w,w}{  }}}\sphinxcode{\sphinxupquote{lezargus.library.logging.}}\sphinxbfcode{\sphinxupquote{FileWarning}}}
\pysigstopsignatures
\sphinxAtStartPar
Bases: {\hyperref[\detokenize{code/lezargus.library.logging:lezargus.library.logging.LezargusWarning}]{\sphinxcrossref{\sphinxcode{\sphinxupquote{LezargusWarning}}}}}

\sphinxAtStartPar
A warning used for file and permission issues which are not fatal.

\sphinxAtStartPar
If there are issues with files, use this warning. However, unlike the
error version FileError, this should be used when the file issue is
a case considered and is recoverable. This warning should not be
used in cases where the problem is because of an incorrect format of the
file (other than corruption).

\end{fulllineitems}\end{savenotes}

\index{InputError@\spxentry{InputError}}

\begin{savenotes}\begin{fulllineitems}
\phantomsection\label{\detokenize{code/lezargus.library.logging:lezargus.library.logging.InputError}}
\pysigstartsignatures
\pysigline{\sphinxbfcode{\sphinxupquote{exception\DUrole{w,w}{  }}}\sphinxcode{\sphinxupquote{lezargus.library.logging.}}\sphinxbfcode{\sphinxupquote{InputError}}}
\pysigstopsignatures
\sphinxAtStartPar
Bases: {\hyperref[\detokenize{code/lezargus.library.logging:lezargus.library.logging.LezargusError}]{\sphinxcrossref{\sphinxcode{\sphinxupquote{LezargusError}}}}}

\sphinxAtStartPar
An error used for issues with input parameters or data.

\sphinxAtStartPar
This is the error to be used when the inputs to a function are not valid
and do not match the expectations of that function.

\end{fulllineitems}\end{savenotes}

\index{InputWarning@\spxentry{InputWarning}}

\begin{savenotes}\begin{fulllineitems}
\phantomsection\label{\detokenize{code/lezargus.library.logging:lezargus.library.logging.InputWarning}}
\pysigstartsignatures
\pysigline{\sphinxbfcode{\sphinxupquote{exception\DUrole{w,w}{  }}}\sphinxcode{\sphinxupquote{lezargus.library.logging.}}\sphinxbfcode{\sphinxupquote{InputWarning}}}
\pysigstopsignatures
\sphinxAtStartPar
Bases: {\hyperref[\detokenize{code/lezargus.library.logging:lezargus.library.logging.LezargusWarning}]{\sphinxcrossref{\sphinxcode{\sphinxupquote{LezargusWarning}}}}}

\sphinxAtStartPar
A warning for a weird input.

\sphinxAtStartPar
This warning is used when the input of a function or a field is not
expected, but may be able to be handled.

\end{fulllineitems}\end{savenotes}

\index{LezargusBaseError@\spxentry{LezargusBaseError}}

\begin{savenotes}\begin{fulllineitems}
\phantomsection\label{\detokenize{code/lezargus.library.logging:lezargus.library.logging.LezargusBaseError}}
\pysigstartsignatures
\pysigline{\sphinxbfcode{\sphinxupquote{exception\DUrole{w,w}{  }}}\sphinxcode{\sphinxupquote{lezargus.library.logging.}}\sphinxbfcode{\sphinxupquote{LezargusBaseError}}}
\pysigstopsignatures
\sphinxAtStartPar
Bases: \sphinxcode{\sphinxupquote{BaseException}}

\sphinxAtStartPar
The base inheriting class which for all Lezargus errors.

\sphinxAtStartPar
This is for exceptions that should never be caught and should bring
everything to a halt.

\end{fulllineitems}\end{savenotes}

\index{LezargusError@\spxentry{LezargusError}}

\begin{savenotes}\begin{fulllineitems}
\phantomsection\label{\detokenize{code/lezargus.library.logging:lezargus.library.logging.LezargusError}}
\pysigstartsignatures
\pysigline{\sphinxbfcode{\sphinxupquote{exception\DUrole{w,w}{  }}}\sphinxcode{\sphinxupquote{lezargus.library.logging.}}\sphinxbfcode{\sphinxupquote{LezargusError}}}
\pysigstopsignatures
\sphinxAtStartPar
Bases: \sphinxcode{\sphinxupquote{Exception}}

\sphinxAtStartPar
The main inheriting class which all Lezargus errors use as their base.

\sphinxAtStartPar
This is done for ease of error handling and is something that can and
should be managed.

\end{fulllineitems}\end{savenotes}

\index{LezargusWarning@\spxentry{LezargusWarning}}

\begin{savenotes}\begin{fulllineitems}
\phantomsection\label{\detokenize{code/lezargus.library.logging:lezargus.library.logging.LezargusWarning}}
\pysigstartsignatures
\pysigline{\sphinxbfcode{\sphinxupquote{exception\DUrole{w,w}{  }}}\sphinxcode{\sphinxupquote{lezargus.library.logging.}}\sphinxbfcode{\sphinxupquote{LezargusWarning}}}
\pysigstopsignatures
\sphinxAtStartPar
Bases: \sphinxcode{\sphinxupquote{UserWarning}}

\sphinxAtStartPar
The main inheriting class which all Lezargus warnings use as their base.

\sphinxAtStartPar
The base warning class which all of the other Lezargus warnings
are derived from.

\end{fulllineitems}\end{savenotes}

\index{LogicFlowError@\spxentry{LogicFlowError}}

\begin{savenotes}\begin{fulllineitems}
\phantomsection\label{\detokenize{code/lezargus.library.logging:lezargus.library.logging.LogicFlowError}}
\pysigstartsignatures
\pysigline{\sphinxbfcode{\sphinxupquote{exception\DUrole{w,w}{  }}}\sphinxcode{\sphinxupquote{lezargus.library.logging.}}\sphinxbfcode{\sphinxupquote{LogicFlowError}}}
\pysigstopsignatures
\sphinxAtStartPar
Bases: {\hyperref[\detokenize{code/lezargus.library.logging:lezargus.library.logging.LezargusBaseError}]{\sphinxcrossref{\sphinxcode{\sphinxupquote{LezargusBaseError}}}}}

\sphinxAtStartPar
An error used for an error in the flow of program logic.

\sphinxAtStartPar
This is an error to ensure that the logic does not flow to a point to a
place where it is not supposed to. This is helpful in making sure changes
to the code do not screw up the logical flow of the program.

\end{fulllineitems}\end{savenotes}

\index{MemoryFullWarning@\spxentry{MemoryFullWarning}}

\begin{savenotes}\begin{fulllineitems}
\phantomsection\label{\detokenize{code/lezargus.library.logging:lezargus.library.logging.MemoryFullWarning}}
\pysigstartsignatures
\pysigline{\sphinxbfcode{\sphinxupquote{exception\DUrole{w,w}{  }}}\sphinxcode{\sphinxupquote{lezargus.library.logging.}}\sphinxbfcode{\sphinxupquote{MemoryFullWarning}}}
\pysigstopsignatures
\sphinxAtStartPar
Bases: {\hyperref[\detokenize{code/lezargus.library.logging:lezargus.library.logging.LezargusWarning}]{\sphinxcrossref{\sphinxcode{\sphinxupquote{LezargusWarning}}}}}

\sphinxAtStartPar
A warning for when there is not enough volatile memory.

\sphinxAtStartPar
This warning is used when the program detects that the machine does not
have enough memory to proceed with a given process and so it tries an
alternative method to do a similar calculation. We use the name
MemoryFullWarning to avoid a name collision with MemoryWarning and to be a
little more specific about what the issue is.

\end{fulllineitems}\end{savenotes}

\index{NotSupportedError@\spxentry{NotSupportedError}}

\begin{savenotes}\begin{fulllineitems}
\phantomsection\label{\detokenize{code/lezargus.library.logging:lezargus.library.logging.NotSupportedError}}
\pysigstartsignatures
\pysigline{\sphinxbfcode{\sphinxupquote{exception\DUrole{w,w}{  }}}\sphinxcode{\sphinxupquote{lezargus.library.logging.}}\sphinxbfcode{\sphinxupquote{NotSupportedError}}}
\pysigstopsignatures
\sphinxAtStartPar
Bases: {\hyperref[\detokenize{code/lezargus.library.logging:lezargus.library.logging.LezargusBaseError}]{\sphinxcrossref{\sphinxcode{\sphinxupquote{LezargusBaseError}}}}}

\sphinxAtStartPar
An error used for something which is beyond the scope of work.

\sphinxAtStartPar
This is an error to be used when what is trying to be done does not
seem reasonable. Usually warnings are the better thing for this but
this error is used when the assumptions for reasonability guided
development and what the user is trying to do is not currently supported
by the software.

\end{fulllineitems}\end{savenotes}

\index{OutOfOrderError@\spxentry{OutOfOrderError}}

\begin{savenotes}\begin{fulllineitems}
\phantomsection\label{\detokenize{code/lezargus.library.logging:lezargus.library.logging.OutOfOrderError}}
\pysigstartsignatures
\pysigline{\sphinxbfcode{\sphinxupquote{exception\DUrole{w,w}{  }}}\sphinxcode{\sphinxupquote{lezargus.library.logging.}}\sphinxbfcode{\sphinxupquote{OutOfOrderError}}}
\pysigstopsignatures
\sphinxAtStartPar
Bases: {\hyperref[\detokenize{code/lezargus.library.logging:lezargus.library.logging.LezargusError}]{\sphinxcrossref{\sphinxcode{\sphinxupquote{LezargusError}}}}}

\sphinxAtStartPar
An error used when things are done out\sphinxhyphen{}of\sphinxhyphen{}order.

\sphinxAtStartPar
This error is used when something is happening out of the expected required
order. This order being in place for specific publicly communicated
reasons.

\end{fulllineitems}\end{savenotes}

\index{ReadOnlyError@\spxentry{ReadOnlyError}}

\begin{savenotes}\begin{fulllineitems}
\phantomsection\label{\detokenize{code/lezargus.library.logging:lezargus.library.logging.ReadOnlyError}}
\pysigstartsignatures
\pysigline{\sphinxbfcode{\sphinxupquote{exception\DUrole{w,w}{  }}}\sphinxcode{\sphinxupquote{lezargus.library.logging.}}\sphinxbfcode{\sphinxupquote{ReadOnlyError}}}
\pysigstopsignatures
\sphinxAtStartPar
Bases: {\hyperref[\detokenize{code/lezargus.library.logging:lezargus.library.logging.LezargusError}]{\sphinxcrossref{\sphinxcode{\sphinxupquote{LezargusError}}}}}

\sphinxAtStartPar
An error used for problems with read\sphinxhyphen{}only files.

\sphinxAtStartPar
If the file is read\sphinxhyphen{}only and it needs to be read, use FileError. This
error is to be used only when variables or files are assumed to be read
only, this error should be used to enforce that notion.

\end{fulllineitems}\end{savenotes}

\index{UndiscoveredError@\spxentry{UndiscoveredError}}

\begin{savenotes}\begin{fulllineitems}
\phantomsection\label{\detokenize{code/lezargus.library.logging:lezargus.library.logging.UndiscoveredError}}
\pysigstartsignatures
\pysigline{\sphinxbfcode{\sphinxupquote{exception\DUrole{w,w}{  }}}\sphinxcode{\sphinxupquote{lezargus.library.logging.}}\sphinxbfcode{\sphinxupquote{UndiscoveredError}}}
\pysigstopsignatures
\sphinxAtStartPar
Bases: {\hyperref[\detokenize{code/lezargus.library.logging:lezargus.library.logging.LezargusBaseError}]{\sphinxcrossref{\sphinxcode{\sphinxupquote{LezargusBaseError}}}}}

\sphinxAtStartPar
An error used for an unknown error.

\sphinxAtStartPar
This is an error used in cases where the source of the error has not
been determined and so a more helpful error message or mitigation strategy
cannot be devised.

\end{fulllineitems}\end{savenotes}

\index{add\_file\_logging\_handler() (in module lezargus.library.logging)@\spxentry{add\_file\_logging\_handler()}\spxextra{in module lezargus.library.logging}}

\begin{savenotes}\begin{fulllineitems}
\phantomsection\label{\detokenize{code/lezargus.library.logging:lezargus.library.logging.add_file_logging_handler}}
\pysigstartsignatures
\pysiglinewithargsret{\sphinxcode{\sphinxupquote{lezargus.library.logging.}}\sphinxbfcode{\sphinxupquote{add\_file\_logging\_handler}}}{\sphinxparam{\DUrole{n,n}{filename}\DUrole{p,p}{:}\DUrole{w,w}{  }\DUrole{n,n}{str}}, \sphinxparam{\DUrole{n,n}{log\_level}\DUrole{p,p}{:}\DUrole{w,w}{  }\DUrole{n,n}{int}\DUrole{w,w}{  }\DUrole{o,o}{=}\DUrole{w,w}{  }\DUrole{default_value}{10}}}{{ $\rightarrow$ None}}
\pysigstopsignatures
\sphinxAtStartPar
Add a stream handler to the logging infrastructure.
\begin{quote}\begin{description}
\sphinxlineitem{Parameters}\begin{itemize}
\item {} 
\sphinxAtStartPar
\sphinxstyleliteralstrong{\sphinxupquote{filename}} (\sphinxstyleliteralemphasis{\sphinxupquote{str}}) – The filename path where the log file will be saved to.

\item {} 
\sphinxAtStartPar
\sphinxstyleliteralstrong{\sphinxupquote{log\_level}} (\sphinxstyleliteralemphasis{\sphinxupquote{int}}) – The logging level for this handler.

\end{itemize}

\sphinxlineitem{Return type}
\sphinxAtStartPar
None

\end{description}\end{quote}

\end{fulllineitems}\end{savenotes}

\index{add\_stream\_logging\_handler() (in module lezargus.library.logging)@\spxentry{add\_stream\_logging\_handler()}\spxextra{in module lezargus.library.logging}}

\begin{savenotes}\begin{fulllineitems}
\phantomsection\label{\detokenize{code/lezargus.library.logging:lezargus.library.logging.add_stream_logging_handler}}
\pysigstartsignatures
\pysiglinewithargsret{\sphinxcode{\sphinxupquote{lezargus.library.logging.}}\sphinxbfcode{\sphinxupquote{add\_stream\_logging\_handler}}}{\sphinxparam{\DUrole{n,n}{stream}\DUrole{p,p}{:}\DUrole{w,w}{  }\DUrole{n,n}{object}}, \sphinxparam{\DUrole{n,n}{log\_level}\DUrole{p,p}{:}\DUrole{w,w}{  }\DUrole{n,n}{int}\DUrole{w,w}{  }\DUrole{o,o}{=}\DUrole{w,w}{  }\DUrole{default_value}{10}}, \sphinxparam{\DUrole{n,n}{use\_color}\DUrole{p,p}{:}\DUrole{w,w}{  }\DUrole{n,n}{bool}\DUrole{w,w}{  }\DUrole{o,o}{=}\DUrole{w,w}{  }\DUrole{default_value}{True}}}{{ $\rightarrow$ None}}
\pysigstopsignatures
\sphinxAtStartPar
Add a stream handler to the logging infrastructure.
\begin{quote}\begin{description}
\sphinxlineitem{Parameters}\begin{itemize}
\item {} 
\sphinxAtStartPar
\sphinxstyleliteralstrong{\sphinxupquote{stream}} (\sphinxstyleliteralemphasis{\sphinxupquote{Any}}) – The stream where the logs will write to.

\item {} 
\sphinxAtStartPar
\sphinxstyleliteralstrong{\sphinxupquote{log\_level}} (\sphinxstyleliteralemphasis{\sphinxupquote{int}}) – The logging level for this handler.

\item {} 
\sphinxAtStartPar
\sphinxstyleliteralstrong{\sphinxupquote{use\_color}} (\sphinxstyleliteralemphasis{\sphinxupquote{bool}}) – If True, use colored log messaged based on the configuration file
parameters.

\end{itemize}

\sphinxlineitem{Return type}
\sphinxAtStartPar
None

\end{description}\end{quote}

\end{fulllineitems}\end{savenotes}

\index{critical() (in module lezargus.library.logging)@\spxentry{critical()}\spxextra{in module lezargus.library.logging}}

\begin{savenotes}\begin{fulllineitems}
\phantomsection\label{\detokenize{code/lezargus.library.logging:lezargus.library.logging.critical}}
\pysigstartsignatures
\pysiglinewithargsret{\sphinxcode{\sphinxupquote{lezargus.library.logging.}}\sphinxbfcode{\sphinxupquote{critical}}}{\sphinxparam{\DUrole{n,n}{critical\_type}\DUrole{p,p}{:}\DUrole{w,w}{  }\DUrole{n,n}{{\hyperref[\detokenize{code/lezargus.library.logging:lezargus.library.logging.LezargusError}]{\sphinxcrossref{LezargusError}}}}}, \sphinxparam{\DUrole{n,n}{message}\DUrole{p,p}{:}\DUrole{w,w}{  }\DUrole{n,n}{str}}}{{ $\rightarrow$ None}}
\pysigstopsignatures
\sphinxAtStartPar
Log a critical error and raise.

\sphinxAtStartPar
Use this for issues which are more serious than warnings and should
raise/throw an exception. The main difference between critical and error
for logging is that critical will also raise the exception as error will
not and the program will attempt to continue.

\sphinxAtStartPar
This is a wrapper around the critical function to standardize it for
Lezargus.
\begin{quote}\begin{description}
\sphinxlineitem{Parameters}\begin{itemize}
\item {} 
\sphinxAtStartPar
\sphinxstyleliteralstrong{\sphinxupquote{critical\_type}} ({\hyperref[\detokenize{code/lezargus.library.logging:lezargus.library.logging.LezargusError}]{\sphinxcrossref{\sphinxstyleliteralemphasis{\sphinxupquote{LezargusError}}}}}) – The class of the critical exception error which will be used and
raised.

\item {} 
\sphinxAtStartPar
\sphinxstyleliteralstrong{\sphinxupquote{message}} (\sphinxstyleliteralemphasis{\sphinxupquote{str}}) – The critical error message.

\end{itemize}

\sphinxlineitem{Return type}
\sphinxAtStartPar
None

\end{description}\end{quote}

\end{fulllineitems}\end{savenotes}

\index{debug() (in module lezargus.library.logging)@\spxentry{debug()}\spxextra{in module lezargus.library.logging}}

\begin{savenotes}\begin{fulllineitems}
\phantomsection\label{\detokenize{code/lezargus.library.logging:lezargus.library.logging.debug}}
\pysigstartsignatures
\pysiglinewithargsret{\sphinxcode{\sphinxupquote{lezargus.library.logging.}}\sphinxbfcode{\sphinxupquote{debug}}}{\sphinxparam{\DUrole{n,n}{message}\DUrole{p,p}{:}\DUrole{w,w}{  }\DUrole{n,n}{str}}}{{ $\rightarrow$ None}}
\pysigstopsignatures
\sphinxAtStartPar
Log a debug message.

\sphinxAtStartPar
This is a wrapper around the debug function to standardize it for Lezargus.
\begin{quote}\begin{description}
\sphinxlineitem{Parameters}
\sphinxAtStartPar
\sphinxstyleliteralstrong{\sphinxupquote{message}} (\sphinxstyleliteralemphasis{\sphinxupquote{str}}) – The debugging message.

\sphinxlineitem{Return type}
\sphinxAtStartPar
None

\end{description}\end{quote}

\end{fulllineitems}\end{savenotes}

\index{error() (in module lezargus.library.logging)@\spxentry{error()}\spxextra{in module lezargus.library.logging}}

\begin{savenotes}\begin{fulllineitems}
\phantomsection\label{\detokenize{code/lezargus.library.logging:lezargus.library.logging.error}}
\pysigstartsignatures
\pysiglinewithargsret{\sphinxcode{\sphinxupquote{lezargus.library.logging.}}\sphinxbfcode{\sphinxupquote{error}}}{\sphinxparam{\DUrole{n,n}{error\_type}\DUrole{p,p}{:}\DUrole{w,w}{  }\DUrole{n,n}{{\hyperref[\detokenize{code/lezargus.library.logging:lezargus.library.logging.LezargusError}]{\sphinxcrossref{LezargusError}}}}}, \sphinxparam{\DUrole{n,n}{message}\DUrole{p,p}{:}\DUrole{w,w}{  }\DUrole{n,n}{str}}, \sphinxparam{\DUrole{n,n}{elevate}\DUrole{p,p}{:}\DUrole{w,w}{  }\DUrole{n,n}{bool\DUrole{w,w}{  }\DUrole{p,p}{|}\DUrole{w,w}{  }None}\DUrole{w,w}{  }\DUrole{o,o}{=}\DUrole{w,w}{  }\DUrole{default_value}{None}}}{{ $\rightarrow$ None}}
\pysigstopsignatures
\sphinxAtStartPar
Log an error message, do not raise.

\sphinxAtStartPar
Use this for issues which are more serious than warnings but do not result
in a raised exception.

\sphinxAtStartPar
This is a wrapper around the error function to standardize it for Lezargus.
\begin{quote}\begin{description}
\sphinxlineitem{Parameters}\begin{itemize}
\item {} 
\sphinxAtStartPar
\sphinxstyleliteralstrong{\sphinxupquote{error\_type}} ({\hyperref[\detokenize{code/lezargus.library.logging:lezargus.library.logging.LezargusError}]{\sphinxcrossref{\sphinxstyleliteralemphasis{\sphinxupquote{LezargusError}}}}}) – The class of the error which will be used.

\item {} 
\sphinxAtStartPar
\sphinxstyleliteralstrong{\sphinxupquote{message}} (\sphinxstyleliteralemphasis{\sphinxupquote{str}}) – The error message.

\item {} 
\sphinxAtStartPar
\sphinxstyleliteralstrong{\sphinxupquote{elevate}} (\sphinxstyleliteralemphasis{\sphinxupquote{bool}}\sphinxstyleliteralemphasis{\sphinxupquote{, }}\sphinxstyleliteralemphasis{\sphinxupquote{default = None}}) – If True, always elevate the error to a critical issue. By default,
use the configuration value.

\end{itemize}

\sphinxlineitem{Return type}
\sphinxAtStartPar
None

\end{description}\end{quote}

\end{fulllineitems}\end{savenotes}

\index{info() (in module lezargus.library.logging)@\spxentry{info()}\spxextra{in module lezargus.library.logging}}

\begin{savenotes}\begin{fulllineitems}
\phantomsection\label{\detokenize{code/lezargus.library.logging:lezargus.library.logging.info}}
\pysigstartsignatures
\pysiglinewithargsret{\sphinxcode{\sphinxupquote{lezargus.library.logging.}}\sphinxbfcode{\sphinxupquote{info}}}{\sphinxparam{\DUrole{n,n}{message}\DUrole{p,p}{:}\DUrole{w,w}{  }\DUrole{n,n}{str}}}{{ $\rightarrow$ None}}
\pysigstopsignatures
\sphinxAtStartPar
Log an informational message.

\sphinxAtStartPar
This is a wrapper around the info function to standardize it for Lezargus.
\begin{quote}\begin{description}
\sphinxlineitem{Parameters}
\sphinxAtStartPar
\sphinxstyleliteralstrong{\sphinxupquote{message}} (\sphinxstyleliteralemphasis{\sphinxupquote{str}}) – The informational message.

\sphinxlineitem{Return type}
\sphinxAtStartPar
None

\end{description}\end{quote}

\end{fulllineitems}\end{savenotes}

\index{terminal() (in module lezargus.library.logging)@\spxentry{terminal()}\spxextra{in module lezargus.library.logging}}

\begin{savenotes}\begin{fulllineitems}
\phantomsection\label{\detokenize{code/lezargus.library.logging:lezargus.library.logging.terminal}}
\pysigstartsignatures
\pysiglinewithargsret{\sphinxcode{\sphinxupquote{lezargus.library.logging.}}\sphinxbfcode{\sphinxupquote{terminal}}}{}{{ $\rightarrow$ None}}
\pysigstopsignatures
\sphinxAtStartPar
Terminal error function which is used to stop everything.
\begin{quote}\begin{description}
\sphinxlineitem{Parameters}
\sphinxAtStartPar
\sphinxstyleliteralstrong{\sphinxupquote{None}} – 

\sphinxlineitem{Return type}
\sphinxAtStartPar
None

\end{description}\end{quote}

\end{fulllineitems}\end{savenotes}

\index{update\_global\_minimum\_logging\_level() (in module lezargus.library.logging)@\spxentry{update\_global\_minimum\_logging\_level()}\spxextra{in module lezargus.library.logging}}

\begin{savenotes}\begin{fulllineitems}
\phantomsection\label{\detokenize{code/lezargus.library.logging:lezargus.library.logging.update_global_minimum_logging_level}}
\pysigstartsignatures
\pysiglinewithargsret{\sphinxcode{\sphinxupquote{lezargus.library.logging.}}\sphinxbfcode{\sphinxupquote{update\_global\_minimum\_logging\_level}}}{\sphinxparam{\DUrole{n,n}{log\_level}\DUrole{p,p}{:}\DUrole{w,w}{  }\DUrole{n,n}{int}\DUrole{w,w}{  }\DUrole{o,o}{=}\DUrole{w,w}{  }\DUrole{default_value}{10}}}{{ $\rightarrow$ None}}
\pysigstopsignatures
\sphinxAtStartPar
Update the logging level of this module.

\sphinxAtStartPar
This function updates the minimum logging level which is required for
a log record to be recorded. Handling each single logger handler is really
unnecessary.
\begin{quote}\begin{description}
\sphinxlineitem{Parameters}
\sphinxAtStartPar
\sphinxstyleliteralstrong{\sphinxupquote{log\_level}} (\sphinxstyleliteralemphasis{\sphinxupquote{int}}\sphinxstyleliteralemphasis{\sphinxupquote{, }}\sphinxstyleliteralemphasis{\sphinxupquote{default = logging.DEBUG}}) – The log level which will be set as the minimum level.

\sphinxlineitem{Return type}
\sphinxAtStartPar
None

\end{description}\end{quote}

\end{fulllineitems}\end{savenotes}

\index{warning() (in module lezargus.library.logging)@\spxentry{warning()}\spxextra{in module lezargus.library.logging}}

\begin{savenotes}\begin{fulllineitems}
\phantomsection\label{\detokenize{code/lezargus.library.logging:lezargus.library.logging.warning}}
\pysigstartsignatures
\pysiglinewithargsret{\sphinxcode{\sphinxupquote{lezargus.library.logging.}}\sphinxbfcode{\sphinxupquote{warning}}}{\sphinxparam{\DUrole{n,n}{warning\_type}\DUrole{p,p}{:}\DUrole{w,w}{  }\DUrole{n,n}{{\hyperref[\detokenize{code/lezargus.library.logging:lezargus.library.logging.LezargusWarning}]{\sphinxcrossref{LezargusWarning}}}}}, \sphinxparam{\DUrole{n,n}{message}\DUrole{p,p}{:}\DUrole{w,w}{  }\DUrole{n,n}{str}}, \sphinxparam{\DUrole{n,n}{elevate}\DUrole{p,p}{:}\DUrole{w,w}{  }\DUrole{n,n}{bool\DUrole{w,w}{  }\DUrole{p,p}{|}\DUrole{w,w}{  }None}\DUrole{w,w}{  }\DUrole{o,o}{=}\DUrole{w,w}{  }\DUrole{default_value}{None}}}{{ $\rightarrow$ None}}
\pysigstopsignatures
\sphinxAtStartPar
Log a warning message.

\sphinxAtStartPar
This is a wrapper around the warning function to standardize it for
Lezargus.
\begin{quote}\begin{description}
\sphinxlineitem{Parameters}\begin{itemize}
\item {} 
\sphinxAtStartPar
\sphinxstyleliteralstrong{\sphinxupquote{warning\_type}} ({\hyperref[\detokenize{code/lezargus.library.logging:lezargus.library.logging.LezargusWarning}]{\sphinxcrossref{\sphinxstyleliteralemphasis{\sphinxupquote{LezargusWarning}}}}}) – The class of the warning which will be used.

\item {} 
\sphinxAtStartPar
\sphinxstyleliteralstrong{\sphinxupquote{message}} (\sphinxstyleliteralemphasis{\sphinxupquote{str}}) – The warning message.

\item {} 
\sphinxAtStartPar
\sphinxstyleliteralstrong{\sphinxupquote{elevate}} (\sphinxstyleliteralemphasis{\sphinxupquote{bool}}\sphinxstyleliteralemphasis{\sphinxupquote{, }}\sphinxstyleliteralemphasis{\sphinxupquote{default = None}}) – If True, always elevate the warning to a critical issue. By default,
use the configuration value.

\end{itemize}

\sphinxlineitem{Return type}
\sphinxAtStartPar
None

\end{description}\end{quote}

\end{fulllineitems}\end{savenotes}


\sphinxstepscope


\subparagraph{lezargus.library.path module}
\label{\detokenize{code/lezargus.library.path:module-lezargus.library.path}}\label{\detokenize{code/lezargus.library.path:lezargus-library-path-module}}\label{\detokenize{code/lezargus.library.path::doc}}\index{module@\spxentry{module}!lezargus.library.path@\spxentry{lezargus.library.path}}\index{lezargus.library.path@\spxentry{lezargus.library.path}!module@\spxentry{module}}
\sphinxAtStartPar
Functions to deal with different common pathname manipulations.

\sphinxAtStartPar
As Lezargus is going to be cross platform, this is a nice abstraction.
\index{get\_directory() (in module lezargus.library.path)@\spxentry{get\_directory()}\spxextra{in module lezargus.library.path}}

\begin{savenotes}\begin{fulllineitems}
\phantomsection\label{\detokenize{code/lezargus.library.path:lezargus.library.path.get_directory}}
\pysigstartsignatures
\pysiglinewithargsret{\sphinxcode{\sphinxupquote{lezargus.library.path.}}\sphinxbfcode{\sphinxupquote{get\_directory}}}{\sphinxparam{\DUrole{n,n}{pathname}\DUrole{p,p}{:}\DUrole{w,w}{  }\DUrole{n,n}{str}}}{{ $\rightarrow$ str}}
\pysigstopsignatures
\sphinxAtStartPar
Get the directory from the pathname without the file or the extension.
\begin{quote}\begin{description}
\sphinxlineitem{Parameters}
\sphinxAtStartPar
\sphinxstyleliteralstrong{\sphinxupquote{pathname}} (\sphinxstyleliteralemphasis{\sphinxupquote{str}}) – The pathname which the directory will be extracted.

\sphinxlineitem{Returns}
\sphinxAtStartPar
\sphinxstylestrong{directory} – The directory which belongs to the pathname.

\sphinxlineitem{Return type}
\sphinxAtStartPar
str

\end{description}\end{quote}

\end{fulllineitems}\end{savenotes}

\index{get\_file\_extension() (in module lezargus.library.path)@\spxentry{get\_file\_extension()}\spxextra{in module lezargus.library.path}}

\begin{savenotes}\begin{fulllineitems}
\phantomsection\label{\detokenize{code/lezargus.library.path:lezargus.library.path.get_file_extension}}
\pysigstartsignatures
\pysiglinewithargsret{\sphinxcode{\sphinxupquote{lezargus.library.path.}}\sphinxbfcode{\sphinxupquote{get\_file\_extension}}}{\sphinxparam{\DUrole{n,n}{pathname}\DUrole{p,p}{:}\DUrole{w,w}{  }\DUrole{n,n}{str}}}{{ $\rightarrow$ str}}
\pysigstopsignatures
\sphinxAtStartPar
Get the file extension only from the pathname.
\begin{quote}\begin{description}
\sphinxlineitem{Parameters}
\sphinxAtStartPar
\sphinxstyleliteralstrong{\sphinxupquote{pathname}} (\sphinxstyleliteralemphasis{\sphinxupquote{str}}) – The pathname which the file extension will be extracted.

\sphinxlineitem{Returns}
\sphinxAtStartPar
\sphinxstylestrong{extension} – The file extension only.

\sphinxlineitem{Return type}
\sphinxAtStartPar
str

\end{description}\end{quote}

\end{fulllineitems}\end{savenotes}

\index{get\_filename\_with\_extension() (in module lezargus.library.path)@\spxentry{get\_filename\_with\_extension()}\spxextra{in module lezargus.library.path}}

\begin{savenotes}\begin{fulllineitems}
\phantomsection\label{\detokenize{code/lezargus.library.path:lezargus.library.path.get_filename_with_extension}}
\pysigstartsignatures
\pysiglinewithargsret{\sphinxcode{\sphinxupquote{lezargus.library.path.}}\sphinxbfcode{\sphinxupquote{get\_filename\_with\_extension}}}{\sphinxparam{\DUrole{n,n}{pathname}\DUrole{p,p}{:}\DUrole{w,w}{  }\DUrole{n,n}{str}}}{{ $\rightarrow$ str}}
\pysigstopsignatures
\sphinxAtStartPar
Get the filename from the pathname with the file extension.
\begin{quote}\begin{description}
\sphinxlineitem{Parameters}
\sphinxAtStartPar
\sphinxstyleliteralstrong{\sphinxupquote{pathname}} (\sphinxstyleliteralemphasis{\sphinxupquote{str}}) – The pathname which the filename will be extracted.

\sphinxlineitem{Returns}
\sphinxAtStartPar
\sphinxstylestrong{filename} – The filename with the file extension.

\sphinxlineitem{Return type}
\sphinxAtStartPar
str

\end{description}\end{quote}

\end{fulllineitems}\end{savenotes}

\index{get\_filename\_without\_extension() (in module lezargus.library.path)@\spxentry{get\_filename\_without\_extension()}\spxextra{in module lezargus.library.path}}

\begin{savenotes}\begin{fulllineitems}
\phantomsection\label{\detokenize{code/lezargus.library.path:lezargus.library.path.get_filename_without_extension}}
\pysigstartsignatures
\pysiglinewithargsret{\sphinxcode{\sphinxupquote{lezargus.library.path.}}\sphinxbfcode{\sphinxupquote{get\_filename\_without\_extension}}}{\sphinxparam{\DUrole{n,n}{pathname}\DUrole{p,p}{:}\DUrole{w,w}{  }\DUrole{n,n}{str}}}{{ $\rightarrow$ str}}
\pysigstopsignatures
\sphinxAtStartPar
Get the filename from the pathname without the file extension.
\begin{quote}\begin{description}
\sphinxlineitem{Parameters}
\sphinxAtStartPar
\sphinxstyleliteralstrong{\sphinxupquote{pathname}} (\sphinxstyleliteralemphasis{\sphinxupquote{str}}) – The pathname which the filename will be extracted.

\sphinxlineitem{Returns}
\sphinxAtStartPar
\sphinxstylestrong{filename} – The filename without the file extension.

\sphinxlineitem{Return type}
\sphinxAtStartPar
str

\end{description}\end{quote}

\end{fulllineitems}\end{savenotes}

\index{get\_most\_recent\_filename\_in\_directory() (in module lezargus.library.path)@\spxentry{get\_most\_recent\_filename\_in\_directory()}\spxextra{in module lezargus.library.path}}

\begin{savenotes}\begin{fulllineitems}
\phantomsection\label{\detokenize{code/lezargus.library.path:lezargus.library.path.get_most_recent_filename_in_directory}}
\pysigstartsignatures
\pysiglinewithargsret{\sphinxcode{\sphinxupquote{lezargus.library.path.}}\sphinxbfcode{\sphinxupquote{get\_most\_recent\_filename\_in\_directory}}}{\sphinxparam{\DUrole{n,n}{directory}\DUrole{p,p}{:}\DUrole{w,w}{  }\DUrole{n,n}{str}}, \sphinxparam{\DUrole{n,n}{extension}\DUrole{p,p}{:}\DUrole{w,w}{  }\DUrole{n,n}{str\DUrole{w,w}{  }\DUrole{p,p}{|}\DUrole{w,w}{  }list}\DUrole{w,w}{  }\DUrole{o,o}{=}\DUrole{w,w}{  }\DUrole{default_value}{None}}, \sphinxparam{\DUrole{n,n}{recursive}\DUrole{p,p}{:}\DUrole{w,w}{  }\DUrole{n,n}{bool}\DUrole{w,w}{  }\DUrole{o,o}{=}\DUrole{w,w}{  }\DUrole{default_value}{False}}, \sphinxparam{\DUrole{n,n}{recency\_function}\DUrole{p,p}{:}\DUrole{w,w}{  }\DUrole{n,n}{Callable\DUrole{p,p}{{[}}\DUrole{p,p}{{[}}str\DUrole{p,p}{{]}}\DUrole{p,p}{,}\DUrole{w,w}{  }float\DUrole{p,p}{{]}}}\DUrole{w,w}{  }\DUrole{o,o}{=}\DUrole{w,w}{  }\DUrole{default_value}{None}}}{{ $\rightarrow$ str}}
\pysigstopsignatures
\sphinxAtStartPar
Get the most recent filename from a directory.

\sphinxAtStartPar
Because of issues with different operating systems having differing
issues with storing the creation time of a file, this function sorts based
off of modification time.
\begin{quote}\begin{description}
\sphinxlineitem{Parameters}\begin{itemize}
\item {} 
\sphinxAtStartPar
\sphinxstyleliteralstrong{\sphinxupquote{directory}} (\sphinxstyleliteralemphasis{\sphinxupquote{str}}) – The directory by which the most recent file will be derived from.

\item {} 
\sphinxAtStartPar
\sphinxstyleliteralstrong{\sphinxupquote{extension}} (\sphinxstyleliteralemphasis{\sphinxupquote{str}}\sphinxstyleliteralemphasis{\sphinxupquote{ or }}\sphinxstyleliteralemphasis{\sphinxupquote{list}}\sphinxstyleliteralemphasis{\sphinxupquote{, }}\sphinxstyleliteralemphasis{\sphinxupquote{default = None}}) – The extension by which to filter for. It is often the case that some
files are created but the most recent file of some type is desired.
Only files which match the included extensions will be considered.

\item {} 
\sphinxAtStartPar
\sphinxstyleliteralstrong{\sphinxupquote{recursive}} (\sphinxstyleliteralemphasis{\sphinxupquote{bool}}\sphinxstyleliteralemphasis{\sphinxupquote{, }}\sphinxstyleliteralemphasis{\sphinxupquote{default = False}}) – If True, the directory is searched recursively for the most recent file
based on the recency function.

\item {} 
\sphinxAtStartPar
\sphinxstyleliteralstrong{\sphinxupquote{recency\_function}} (\sphinxstyleliteralemphasis{\sphinxupquote{callable}}\sphinxstyleliteralemphasis{\sphinxupquote{, }}\sphinxstyleliteralemphasis{\sphinxupquote{default = None}}) – A function which, when provided, provides a sorting index for a given
filename. This is used when the default sorting method (modification
time) is not desired and a custom function can be provided here. The
larger the value returned by this function, the more “recent” a
given file will be considered to be.

\end{itemize}

\sphinxlineitem{Returns}
\sphinxAtStartPar
\sphinxstylestrong{recent\_filename} – The filename of the most recent file, by modification time, in the
directory.

\sphinxlineitem{Return type}
\sphinxAtStartPar
str

\end{description}\end{quote}

\end{fulllineitems}\end{savenotes}

\index{merge\_pathname() (in module lezargus.library.path)@\spxentry{merge\_pathname()}\spxextra{in module lezargus.library.path}}

\begin{savenotes}\begin{fulllineitems}
\phantomsection\label{\detokenize{code/lezargus.library.path:lezargus.library.path.merge_pathname}}
\pysigstartsignatures
\pysiglinewithargsret{\sphinxcode{\sphinxupquote{lezargus.library.path.}}\sphinxbfcode{\sphinxupquote{merge\_pathname}}}{\sphinxparam{\DUrole{n,n}{directory}\DUrole{p,p}{:}\DUrole{w,w}{  }\DUrole{n,n}{str\DUrole{w,w}{  }\DUrole{p,p}{|}\DUrole{w,w}{  }list}\DUrole{w,w}{  }\DUrole{o,o}{=}\DUrole{w,w}{  }\DUrole{default_value}{None}}, \sphinxparam{\DUrole{n,n}{filename}\DUrole{p,p}{:}\DUrole{w,w}{  }\DUrole{n,n}{str\DUrole{w,w}{  }\DUrole{p,p}{|}\DUrole{w,w}{  }None}\DUrole{w,w}{  }\DUrole{o,o}{=}\DUrole{w,w}{  }\DUrole{default_value}{None}}, \sphinxparam{\DUrole{n,n}{extension}\DUrole{p,p}{:}\DUrole{w,w}{  }\DUrole{n,n}{str\DUrole{w,w}{  }\DUrole{p,p}{|}\DUrole{w,w}{  }None}\DUrole{w,w}{  }\DUrole{o,o}{=}\DUrole{w,w}{  }\DUrole{default_value}{None}}}{{ $\rightarrow$ str}}
\pysigstopsignatures
\sphinxAtStartPar
Join the directories, filenames, and file extensions into one pathname.
\begin{quote}\begin{description}
\sphinxlineitem{Parameters}\begin{itemize}
\item {} 
\sphinxAtStartPar
\sphinxstyleliteralstrong{\sphinxupquote{directory}} (\sphinxstyleliteralemphasis{\sphinxupquote{str}}\sphinxstyleliteralemphasis{\sphinxupquote{ or }}\sphinxstyleliteralemphasis{\sphinxupquote{list}}\sphinxstyleliteralemphasis{\sphinxupquote{, }}\sphinxstyleliteralemphasis{\sphinxupquote{default = None}}) – The directory(s) which is going to be used. If it is a list,
then the paths within it are combined.

\item {} 
\sphinxAtStartPar
\sphinxstyleliteralstrong{\sphinxupquote{filename}} (\sphinxstyleliteralemphasis{\sphinxupquote{str}}\sphinxstyleliteralemphasis{\sphinxupquote{, }}\sphinxstyleliteralemphasis{\sphinxupquote{default = None}}) – The filename that is going to be used for path construction.

\item {} 
\sphinxAtStartPar
\sphinxstyleliteralstrong{\sphinxupquote{extension}} (\sphinxstyleliteralemphasis{\sphinxupquote{str}}\sphinxstyleliteralemphasis{\sphinxupquote{, }}\sphinxstyleliteralemphasis{\sphinxupquote{default = None}}) – The filename extension that is going to be used.

\end{itemize}

\sphinxlineitem{Returns}
\sphinxAtStartPar
\sphinxstylestrong{pathname} – The combined pathname.

\sphinxlineitem{Return type}
\sphinxAtStartPar
str

\end{description}\end{quote}

\end{fulllineitems}\end{savenotes}

\index{split\_pathname() (in module lezargus.library.path)@\spxentry{split\_pathname()}\spxextra{in module lezargus.library.path}}

\begin{savenotes}\begin{fulllineitems}
\phantomsection\label{\detokenize{code/lezargus.library.path:lezargus.library.path.split_pathname}}
\pysigstartsignatures
\pysiglinewithargsret{\sphinxcode{\sphinxupquote{lezargus.library.path.}}\sphinxbfcode{\sphinxupquote{split\_pathname}}}{\sphinxparam{\DUrole{n,n}{pathname}\DUrole{p,p}{:}\DUrole{w,w}{  }\DUrole{n,n}{str}}}{{ $\rightarrow$ tuple\DUrole{p,p}{{[}}str\DUrole{p,p}{,}\DUrole{w,w}{  }str\DUrole{p,p}{,}\DUrole{w,w}{  }str\DUrole{p,p}{{]}}}}
\pysigstopsignatures
\sphinxAtStartPar
Return a pathname split into its components.

\sphinxAtStartPar
This is a wrapper function around the more elementary functions
\sphinxtitleref{get\_directory}, \sphinxtitleref{get\_filename\_without\_extension}, and
\sphinxtitleref{get\_file\_extension}.
\begin{quote}\begin{description}
\sphinxlineitem{Parameters}
\sphinxAtStartPar
\sphinxstyleliteralstrong{\sphinxupquote{pathname}} (\sphinxstyleliteralemphasis{\sphinxupquote{str}}) – The combined pathname which to be split.

\sphinxlineitem{Returns}
\sphinxAtStartPar
\begin{itemize}
\item {} 
\sphinxAtStartPar
\sphinxstylestrong{directory} (\sphinxstyleemphasis{str}) – The directory which was split from the pathname.

\item {} 
\sphinxAtStartPar
\sphinxstylestrong{filename} (\sphinxstyleemphasis{str}) – The filename which was split from the pathname.

\item {} 
\sphinxAtStartPar
\sphinxstylestrong{extension} (\sphinxstyleemphasis{str}) – The filename extension which was split from the pathname.

\end{itemize}


\end{description}\end{quote}

\end{fulllineitems}\end{savenotes}


\sphinxstepscope


\subparagraph{lezargus.library.wrapper module}
\label{\detokenize{code/lezargus.library.wrapper:module-lezargus.library.wrapper}}\label{\detokenize{code/lezargus.library.wrapper:lezargus-library-wrapper-module}}\label{\detokenize{code/lezargus.library.wrapper::doc}}\index{module@\spxentry{module}!lezargus.library.wrapper@\spxentry{lezargus.library.wrapper}}\index{lezargus.library.wrapper@\spxentry{lezargus.library.wrapper}!module@\spxentry{module}}
\sphinxAtStartPar
Function wrappers.

\sphinxAtStartPar
We borrow a lot of functions from different packages; however, for a lot of
them, we build wrappers around them to better integrate them into our
package provided its own idiosyncrasies.
\index{blackbody\_function() (in module lezargus.library.wrapper)@\spxentry{blackbody\_function()}\spxextra{in module lezargus.library.wrapper}}

\begin{savenotes}\begin{fulllineitems}
\phantomsection\label{\detokenize{code/lezargus.library.wrapper:lezargus.library.wrapper.blackbody_function}}
\pysigstartsignatures
\pysiglinewithargsret{\sphinxcode{\sphinxupquote{lezargus.library.wrapper.}}\sphinxbfcode{\sphinxupquote{blackbody\_function}}}{\sphinxparam{\DUrole{n,n}{temperature}\DUrole{p,p}{:}\DUrole{w,w}{  }\DUrole{n,n}{float}}}{{ $\rightarrow$ Callable\DUrole{p,p}{{[}}\DUrole{p,p}{{[}}\sphinxhref{https://numpy.org/doc/stable/reference/generated/numpy.ndarray.html\#numpy.ndarray}{ndarray}%
\begin{footnote}[117]\sphinxAtStartFootnote
\sphinxnolinkurl{https://numpy.org/doc/stable/reference/generated/numpy.ndarray.html\#numpy.ndarray}
%
\end{footnote}\DUrole{p,p}{{]}}\DUrole{p,p}{,}\DUrole{w,w}{  }\sphinxhref{https://numpy.org/doc/stable/reference/generated/numpy.ndarray.html\#numpy.ndarray}{ndarray}%
\begin{footnote}[118]\sphinxAtStartFootnote
\sphinxnolinkurl{https://numpy.org/doc/stable/reference/generated/numpy.ndarray.html\#numpy.ndarray}
%
\end{footnote}\DUrole{p,p}{{]}}}}
\pysigstopsignatures
\sphinxAtStartPar
Return a callable blackbody function for a given temperature.

\sphinxAtStartPar
This function is a wrapper around the Astropy blackbody model. This wrapper
exists to remove the unit baggage of the original Astropy blackbody
model so that we can stick to the convention of Lezargus.
\begin{quote}\begin{description}
\sphinxlineitem{Parameters}
\sphinxAtStartPar
\sphinxstyleliteralstrong{\sphinxupquote{temperature}} (\sphinxstyleliteralemphasis{\sphinxupquote{float}}) – The blackbody temperature, in Kelvin.

\sphinxlineitem{Returns}
\sphinxAtStartPar
\sphinxstylestrong{blackbody} – The blackbody function, the wavelength callable is in microns. The
return units are in FLAM/sr.

\sphinxlineitem{Return type}
\sphinxAtStartPar
Callable

\end{description}\end{quote}

\end{fulllineitems}\end{savenotes}

\index{cubic\_interpolate\_1d\_function() (in module lezargus.library.wrapper)@\spxentry{cubic\_interpolate\_1d\_function()}\spxextra{in module lezargus.library.wrapper}}

\begin{savenotes}\begin{fulllineitems}
\phantomsection\label{\detokenize{code/lezargus.library.wrapper:lezargus.library.wrapper.cubic_interpolate_1d_function}}
\pysigstartsignatures
\pysiglinewithargsret{\sphinxcode{\sphinxupquote{lezargus.library.wrapper.}}\sphinxbfcode{\sphinxupquote{cubic\_interpolate\_1d\_function}}}{\sphinxparam{\DUrole{n,n}{x}\DUrole{p,p}{:}\DUrole{w,w}{  }\DUrole{n,n}{\sphinxhref{https://numpy.org/doc/stable/reference/generated/numpy.ndarray.html\#numpy.ndarray}{ndarray}%
\begin{footnote}[119]\sphinxAtStartFootnote
\sphinxnolinkurl{https://numpy.org/doc/stable/reference/generated/numpy.ndarray.html\#numpy.ndarray}
%
\end{footnote}}}, \sphinxparam{\DUrole{n,n}{y}\DUrole{p,p}{:}\DUrole{w,w}{  }\DUrole{n,n}{\sphinxhref{https://numpy.org/doc/stable/reference/generated/numpy.ndarray.html\#numpy.ndarray}{ndarray}%
\begin{footnote}[120]\sphinxAtStartFootnote
\sphinxnolinkurl{https://numpy.org/doc/stable/reference/generated/numpy.ndarray.html\#numpy.ndarray}
%
\end{footnote}}}}{{ $\rightarrow$ Callable\DUrole{p,p}{{[}}\DUrole{p,p}{{[}}\sphinxhref{https://numpy.org/doc/stable/reference/generated/numpy.ndarray.html\#numpy.ndarray}{ndarray}%
\begin{footnote}[121]\sphinxAtStartFootnote
\sphinxnolinkurl{https://numpy.org/doc/stable/reference/generated/numpy.ndarray.html\#numpy.ndarray}
%
\end{footnote}\DUrole{p,p}{{]}}\DUrole{p,p}{,}\DUrole{w,w}{  }\sphinxhref{https://numpy.org/doc/stable/reference/generated/numpy.ndarray.html\#numpy.ndarray}{ndarray}%
\begin{footnote}[122]\sphinxAtStartFootnote
\sphinxnolinkurl{https://numpy.org/doc/stable/reference/generated/numpy.ndarray.html\#numpy.ndarray}
%
\end{footnote}\DUrole{p,p}{{]}}}}
\pysigstopsignatures
\sphinxAtStartPar
Return a wrapper around Scipy’s Cubic interpolation.
\begin{quote}\begin{description}
\sphinxlineitem{Parameters}\begin{itemize}
\item {} 
\sphinxAtStartPar
\sphinxstyleliteralstrong{\sphinxupquote{x}} (\sphinxstyleliteralemphasis{\sphinxupquote{ndarray}}) – The x data to interpolate over.

\item {} 
\sphinxAtStartPar
\sphinxstyleliteralstrong{\sphinxupquote{y}} (\sphinxstyleliteralemphasis{\sphinxupquote{ndarray}}) – The y data to interpolate over.

\end{itemize}

\sphinxlineitem{Returns}
\sphinxAtStartPar
\sphinxstylestrong{interpolate\_function} – The interpolation function of the data.

\sphinxlineitem{Return type}
\sphinxAtStartPar
Callable

\end{description}\end{quote}

\end{fulllineitems}\end{savenotes}



\subparagraph{Module contents}
\label{\detokenize{code/lezargus.library:module-lezargus.library}}\label{\detokenize{code/lezargus.library:module-contents}}\index{module@\spxentry{module}!lezargus.library@\spxentry{lezargus.library}}\index{lezargus.library@\spxentry{lezargus.library}!module@\spxentry{module}}
\sphinxAtStartPar
Common routines which are important functions of Lezargus.


\subsubsection{Submodules}
\label{\detokenize{code/lezargus:submodules}}
\sphinxstepscope


\paragraph{lezargus.\_\_main\_\_ module}
\label{\detokenize{code/lezargus.__main__:module-lezargus.__main__}}\label{\detokenize{code/lezargus.__main__:lezargus-main-module}}\label{\detokenize{code/lezargus.__main__::doc}}\index{module@\spxentry{module}!lezargus.\_\_main\_\_@\spxentry{lezargus.\_\_main\_\_}}\index{lezargus.\_\_main\_\_@\spxentry{lezargus.\_\_main\_\_}!module@\spxentry{module}}
\sphinxAtStartPar
Just a small hook for the main execution.

\sphinxAtStartPar
This section parses arguments which is then passed to execution to do exactly
as expected by the commands.

\sphinxstepscope


\paragraph{lezargus.\_\_version\_\_ module}
\label{\detokenize{code/lezargus.__version__:module-lezargus.__version__}}\label{\detokenize{code/lezargus.__version__:lezargus-version-module}}\label{\detokenize{code/lezargus.__version__::doc}}\index{module@\spxentry{module}!lezargus.\_\_version\_\_@\spxentry{lezargus.\_\_version\_\_}}\index{lezargus.\_\_version\_\_@\spxentry{lezargus.\_\_version\_\_}!module@\spxentry{module}}
\sphinxAtStartPar
The version of the software. Use \sphinxtitleref{hatch} to upgrade this version number.

\sphinxAtStartPar
Please do not manually edit this value.


\subsubsection{Module contents}
\label{\detokenize{code/lezargus:module-lezargus}}\label{\detokenize{code/lezargus:module-contents}}\index{module@\spxentry{module}!lezargus@\spxentry{lezargus}}\index{lezargus@\spxentry{lezargus}!module@\spxentry{module}}
\sphinxAtStartPar
Lezargus: The software package related to IRTF SPECTRE.


\chapter{Indices and tables}
\label{\detokenize{index:indices-and-tables}}\begin{itemize}
\item {} 
\sphinxAtStartPar
\DUrole{xref,std,std-ref}{genindex}

\item {} 
\sphinxAtStartPar
\DUrole{xref,std,std-ref}{modindex}

\item {} 
\sphinxAtStartPar
\DUrole{xref,std,std-ref}{search}

\end{itemize}


\renewcommand{\indexname}{Python Module Index}
\begin{sphinxtheindex}
\let\bigletter\sphinxstyleindexlettergroup
\bigletter{l}
\item\relax\sphinxstyleindexentry{lezargus}\sphinxstyleindexpageref{code/lezargus:\detokenize{module-lezargus}}
\item\relax\sphinxstyleindexentry{lezargus.\_\_main\_\_}\sphinxstyleindexpageref{code/lezargus.__main__:\detokenize{module-lezargus.__main__}}
\item\relax\sphinxstyleindexentry{lezargus.\_\_version\_\_}\sphinxstyleindexpageref{code/lezargus.__version__:\detokenize{module-lezargus.__version__}}
\item\relax\sphinxstyleindexentry{lezargus.container}\sphinxstyleindexpageref{code/lezargus.container:\detokenize{module-lezargus.container}}
\item\relax\sphinxstyleindexentry{lezargus.container.cube}\sphinxstyleindexpageref{code/lezargus.container.cube:\detokenize{module-lezargus.container.cube}}
\item\relax\sphinxstyleindexentry{lezargus.container.image}\sphinxstyleindexpageref{code/lezargus.container.image:\detokenize{module-lezargus.container.image}}
\item\relax\sphinxstyleindexentry{lezargus.container.mosaic}\sphinxstyleindexpageref{code/lezargus.container.mosaic:\detokenize{module-lezargus.container.mosaic}}
\item\relax\sphinxstyleindexentry{lezargus.container.parent}\sphinxstyleindexpageref{code/lezargus.container.parent:\detokenize{module-lezargus.container.parent}}
\item\relax\sphinxstyleindexentry{lezargus.container.spectra}\sphinxstyleindexpageref{code/lezargus.container.spectra:\detokenize{module-lezargus.container.spectra}}
\item\relax\sphinxstyleindexentry{lezargus.data}\sphinxstyleindexpageref{code/lezargus.data:\detokenize{module-lezargus.data}}
\item\relax\sphinxstyleindexentry{lezargus.library}\sphinxstyleindexpageref{code/lezargus.library:\detokenize{module-lezargus.library}}
\item\relax\sphinxstyleindexentry{lezargus.library.array}\sphinxstyleindexpageref{code/lezargus.library.array:\detokenize{module-lezargus.library.array}}
\item\relax\sphinxstyleindexentry{lezargus.library.atmosphere}\sphinxstyleindexpageref{code/lezargus.library.atmosphere:\detokenize{module-lezargus.library.atmosphere}}
\item\relax\sphinxstyleindexentry{lezargus.library.config}\sphinxstyleindexpageref{code/lezargus.library.config:\detokenize{module-lezargus.library.config}}
\item\relax\sphinxstyleindexentry{lezargus.library.conversion}\sphinxstyleindexpageref{code/lezargus.library.conversion:\detokenize{module-lezargus.library.conversion}}
\item\relax\sphinxstyleindexentry{lezargus.library.data}\sphinxstyleindexpageref{code/lezargus.library.data:\detokenize{module-lezargus.library.data}}
\item\relax\sphinxstyleindexentry{lezargus.library.fits}\sphinxstyleindexpageref{code/lezargus.library.fits:\detokenize{module-lezargus.library.fits}}
\item\relax\sphinxstyleindexentry{lezargus.library.flags}\sphinxstyleindexpageref{code/lezargus.library.flags:\detokenize{module-lezargus.library.flags}}
\item\relax\sphinxstyleindexentry{lezargus.library.hint}\sphinxstyleindexpageref{code/lezargus.library.hint:\detokenize{module-lezargus.library.hint}}
\item\relax\sphinxstyleindexentry{lezargus.library.logging}\sphinxstyleindexpageref{code/lezargus.library.logging:\detokenize{module-lezargus.library.logging}}
\item\relax\sphinxstyleindexentry{lezargus.library.path}\sphinxstyleindexpageref{code/lezargus.library.path:\detokenize{module-lezargus.library.path}}
\item\relax\sphinxstyleindexentry{lezargus.library.wrapper}\sphinxstyleindexpageref{code/lezargus.library.wrapper:\detokenize{module-lezargus.library.wrapper}}
\end{sphinxtheindex}

\renewcommand{\indexname}{Index}
\printindex
\end{document}